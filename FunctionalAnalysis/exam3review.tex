\documentclass[letterpaper,11pt,intlimits,sumlimits]{amsart}

% Packages
\usepackage{amsfonts}
\usepackage{amsmath}
\usepackage{amssymb}
\usepackage{enumerate}
\usepackage{fancyhdr}
\usepackage{ifthen}
\usepackage{lastpage}
\usepackage{latexsym}
\usepackage{setspace}
\usepackage{txfonts}
\usepackage{xspace}

% Expanded Margins
\setlength{\textheight}{8.65in}
\setlength{\textwidth}{6.5in}
\setlength{\oddsidemargin}{0in}
\setlength{\evensidemargin}{0in}
\setlength{\topmargin}{.200in}

% Line Spacing
\singlespacing
%\onehalfspacing
%\doublespacing

% Set appropriate header/footer information on each page
\fancypagestyle{plain}{
    \fancyhf{}
    \renewcommand{\headheight}{20pt}
    \renewcommand{\headrulewidth}{1pt}
    \renewcommand{\footrulewidth}{0pt}
    \lhead{
        CAM 386M Exam 3 Review
    }
    \rhead{
        Page \thepage{} of \pageref{LastPage}
    }
}
\pagestyle{plain}

% Paragraph spacing
\setlength{\parindent}{0em}
\setlength{\parskip}{1.25ex plus 0.5ex minus 0.5ex}

\newtheorem{lemma}{Lemma}
\newtheorem{theorem}{Theorem}

\newcommand{\norm}[1]{\left|\left|{#1}\right|\right|}
\newcommand{\normone}[1]{\norm{#1}_{1}}
\newcommand{\normtwo}[1]{\norm{#1}_{2}}
\newcommand{\norminfty}[1]{\norm{#1}_{\infty}}
\newcommand{\normF}[1]{\norm{#1}_{\textrm{F}}}
\newcommand{\abs}[1]{\left|{#1}\right|}
\newcommand{\del}[1]{\delta\!{#1}}
\newcommand{\T}{\textrm{T}}
\newcommand{\dualpair}[3][]{{\langle{}{#2},{#3}\rangle{}}_{#1}\xspace{}}
\newcommand{\innerprod}[3][]{{\left({#2},{#3}\right)}_{#1}\xspace{}}
\newcommand{\dist}[2]{\textrm{d}\left( {#1}, {#2} \right)}
\newcommand{\ballo}[2]{B\left( {#1}, {#2} \right)}
\newcommand{\ballc}[2]{\bar{B}\left( {#1}, {#2} \right)}
\newcommand{\closure}[1]{\overline{#1}}
\newcommand{\interior}[1]{\,\textrm{int}\,{#1}\,}
\newcommand{\naturals}{\mathbb{N}}
\newcommand{\powerset}[1]{\mathcal{P}\left({#1}\right)}
\newcommand{\Borel}[1]{\mathcal{B}\left({#1}\right)}
\newcommand{\reals}{\mathbb{R}}
\newcommand{\keyphrase}[1]{\textbf{#1}}
\newcommand{\almosteverywhere}{\,\textrm{a.e.}\,}
\DeclareMathOperator{\LD}{LD}
\DeclareMathOperator{\LI}{LI}
\DeclareMathOperator{\linear}{linear}
\DeclareMathOperator{\nullity}{nullity}
\DeclareMathOperator{\rank}{rank}
\DeclareMathOperator{\open}{open}
\DeclareMathOperator{\dom}{dom}
\DeclareMathOperator{\closed}{closed}
\DeclareMathOperator{\compact}{compact}
\DeclareMathOperator{\continuous}{continuous}
\DeclareMathOperator{\measurable}{measurable}
\DeclareMathOperator{\sign}{sign}

\begin{document}

\section*{Properties of a $\sigma$-algebra (Prop. 3.1.1)}

Given $X$, $S\subset\powerset{X}$ is a
\keyphrase{$\boldsymbol{\sigma}$-algebra} iff
\begin{align*}
    &S \neq \emptyset \tag{i} \\
    &A \in S \implies A' = X - A \in S \tag{ii} \\
    &A_{1}, A_{2},\ldots \in S \implies
    \bigcup_{i=1}^{\infty} A_{i} \in S \tag{iii}
\end{align*}

The $\sigma$-algebra definition implies all of
\begin{align*}
    &A_{1}, \ldots,A_{n} \in S \implies
        \bigcup_{i=1}^{n} A_{i} \in S
    \tag{3.1.1 i} \\
    &\emptyset, X \in S
    \tag{3.1.1 ii} \\
    &A_{1}, A_{2},\ldots \in S \implies
        \bigcap_{i=1}^{\infty} A_{i} \in S
    \tag{3.1.1 iii} \\
    &A_{1},\ldots,A_{n} \in S \implies
        \bigcap_{i=1}^{n} A_{i} \in S
    \tag{3.1.1 iv} \\
    &A,B \in S \implies A-B \in S
    \tag{3.1.1 v}
\end{align*}

If $K\subset\powerset{X}$ then \keyphrase{K generates S}
\[
    S\!\left( K \right) \coloneqq
    \bigcap \left\{ S, \sigma-\text{algebra} : K\subset{}S \right\}
\]
using that
\[
    S_{\iota} \in \powerset{X}, \iota \in I
    \text{ are $\sigma$-algebras}
    \implies
    \bigcap_{\iota\in{}I} S_{\iota}
    \text{is a $\sigma$-algebra}
\]

If $f:X \to Y$ and $S\subset\powerset{X}$ is a $\sigma$-algebra
then
    $R \coloneqq \left\{
        E\in\powerset{Y} : f^{-1}\left( E \right) \in S
    \right\}$
is a $\sigma$-algebra in Y.

If $f:X \to Y$ is bijective
and $S\subset\powerset{X}$ is a $\sigma$-algebra
then both
\begin{align*}
    &f\left( S \right) \coloneqq \left\{ f(A) : A \in S \right\}
    \text{ is a $\sigma$-algebra in $Y$ }
    \tag{a} \\
    &S\!\left( K \right) = S\subset\powerset{X}
        \implies S\!\left( f\left(K\right) \right)
        = f\left(S\right)\in\powerset{Y}
    \tag{b}
\end{align*}

\section*{Properties of Borel sets (Prop. 3.1.4, 3.1.5 combined)}

$\boldsymbol{\Borel{\reals^{n}}} \coloneqq
S\!\left( \text{open sets in $\reals^{n}$} \right)$

(3.1.4) 
If $f:\reals^{n}\to\reals^{m}$ is continuous and $B\in\Borel{\reals^{m}}$
then $f^{-1}\left( B \right)\in\Borel{\reals^{n}}$.
Additionally, if $f$ is a bijection with a continuous $f^{-1}$ then
$f\left( \Borel{\reals^{n}} \right) = \Borel{\reals^{n}}$.

(3.1.5) $E\in\Borel{\reals^{n}} \wedge F\in\Borel{\reals^{m}}
\implies E\times{}F \in \Borel{\reals^{n}\times\reals^{m}}$

\section*{Properties of an (abstract) measure (Prop. 3.1.6)}

Function $\mu:S\to\left[ 0,\infty \right] \in \bar{\reals}$ is a
\keyphrase{measure} iff
\begin{align*}
    &\mu \neq \infty
    \tag{i} \\
    &\mu\left( \bigcup_{i=1}^{\infty} E_{i} \right)
    = \sum_{i=1}^{\infty} \mu\left( E_{i} \right)
    \coloneqq \lim_{n\to\infty}\sum_{i=1}^{n} \mu\left( E_{i} \right)
    & &\text{$E_{i}$ pairwise disjoint}
    \tag{ii}
\end{align*}

The measure definition implies all of
\begin{align*}
    & \mu\left( \emptyset \right) = 0
    \tag{3.1.6 i} \\
    &\mu\left( \bigcup_{i=1}^{n} E_{i} \right)
    = \sum_{i=1}^{n} \mu\left( E_{i} \right)
    & &\text{$E_{i}$ pairwise disjoint}
    \tag{3.1.6 ii} \\
    &E\subset{}F \implies \mu\left( E \right)\leq\mu\left( F \right)
    \tag{3.1.6 iii} \\
    &\mu\left( \bigcup_{i=1}^{\infty} E_{i} \right)
    \leq \sum_{i=1}^{\infty} \mu\left( E_{i} \right)
    & &\text{$E_{i}\in{}S, i=1,2,\ldots$}
    \tag{3.1.6 iv} \\
    &\ldots \subset E_{i} \subset E_{i+1} \subset \ldots \implies
    \mu\left( \bigcup_{i=1}^{\infty} \right)
    = \lim_{n\to\infty} \mu\left( E_{n} \right)
    \tag{3.1.6 v} \\
    &\ldots \supset E_{i} \supset E_{i+1} \subset \ldots \implies
    \mu\left( \bigcap_{i=1}^{\infty} \right)
    = \lim_{n\to\infty} \mu\left( E_{n} \right)
    \tag{3.1.6 vi}
\end{align*}

\section*{Construction of the Lebesgue measure}

For a ``mesh size'' $k\in\naturals$, $\reals^{n}$ can be partitioned by a set
of half-open, half-closed ``cubes'', i.e.
\[
\mathcal{S}^{k}\left(\reals^{n}\right) \coloneqq \left\{
    \left[\frac{\nu_{i}}{2^{k}},\frac{\nu_{i+1}}{2^{k}}\right)
    \times \cdots \times
    \left[\frac{\nu_{i}}{2^{k}},\frac{\nu_{i+1}}{2^{k}}\right)
    : \nu\in\mathbf{Z}^{n}
\right\}.
\]
Any open set $G$ can be similarly partitioned
\mbox{$\mathcal{S}^{k}\left( G \right) \coloneqq \left\{
\sigma\in\mathcal{S}^{k}\left( \reals^{n} \right):\closure{\sigma}\subset{}G
\right\}$}.
Define
\mbox{$S^{k}\left( G \right) \coloneqq
\bigcup\limits_{\sigma\in\mathcal{S}^{k}\left( G \right)}\sigma$}, which is
monotone in $k$ (i.e. \mbox{
$S^{k}\left( G \right)\subset{}S^{k+1}\left( G \right)$}) and allows
recovery of the original set (i.e.
\mbox{$G = \cup_{k=0}^{\infty} S^{k}\left( G \right)$}). Define the
\keyphrase{prototype for the Lebesgue measure}.
\[
m(G) \coloneqq
\lim_{k\to\infty} \frac{1}{2^{kn}} \cdot \#\mathcal{S}^{k}\left( G \right).
\]

From the prototype Lebesgue measure definition follows all of
\begin{align*}
    G\subset{}H \open &\implies m(G) \leq m(H)
    \tag{i} \\
    G,H \open,G\cap{}H = \emptyset &\implies m(G\cap{}H) = m(G)+m(H)
    \tag{ii} \\
    G\subset\reals^{n}, H\subset\reals^{m} \open &\implies
    m(G\times{}H) = m(G)\cdot{}m(H)
    \tag{iii} \\
    m\left( (a,b) \right) &= b - a
    \tag{iv} \\
    m\left( (a_{1},b_{1}) \times\cdots\times (a_{n},b_{n}) \right)
    &= \prod_{i=1}^{n} (b_{i} - a_{i})
    \tag{v}
\end{align*}

\[
    G\subset\reals^{n} \open \implies m(G) = \sup \left\{
        m(H), H \open, \closure{H} \compact \subset G
    \right\}
\]

Subadditivity and $\sigma$-subadditivity of the measure for open sets:
\begin{align*}
    m\left( G_{1}\cup{}G_{2} \right) &\leq m(G_{1}) + m(G_{2})
    \tag{i} \\
    m\left(\bigcup_{i=1}^{n} G_{i}\right)
    &\leq
    \sum_{i=1}^{n}m\left(G_{i}\right)
    \tag{ii} \\
    m\left(\bigcup_{i=1}^{\infty}G_{i}\right)
    &\leq
    \sum_{i=1}^{\infty}m\left(G_{i}\right)
    \tag{iii}
\end{align*}

\[
    G \open \implies
    \inf_{F} \left\{ m\left( G-F) : F \closed \subset G \right) \right\} = 0
\]
\[
    F_{1}, F_{2} \closed, F_{1}\cap{}F_{2}=\emptyset
    \implies
    \exists G_{1}\open\supset{}F_{1},
    \exists G_{2}\open\supset{}F_{2},
    G_{1}\cap{}G_{2}=\emptyset
\]

\section*{Characterization of Lebesgue measurable sets (Prop. 3.2.3, Thm 3.2.1)}

Obtain the \keyphrase{Lebesgue measure} by extending the prototype measure:
\[
    m^{*}\left( E \right) \coloneqq \inf \left\{
        m\left( G \right): G \open \supset E
        \right\} \; \forall E \in \reals^{n}
\]
from which it follows that
\begin{align*}
    E \open &\implies m^{*}\left( E \right) = m\left( E \right)
    \tag{i} \\
    E \subset F &\implies m^{*}\left( E \right)\leq{}m^{*}\left( F \right)
    \tag{ii} \\
    m^{*}\left( \bigcup_{i=1}^{\infty} E_{i} \right)
    &\leq
    \sum_{i=1}^{\infty} m^{*} \left( E_{i} \right)
    \tag{iii}
\end{align*}

The following three families of sets coincide with each other
\begin{align*}
    &\left\{E\subset\reals^{n}:
    \inf_{G \open\supset{}E\phantom{\supset{}F\closed}}
    \;m^{*}\left( G-E \right) = 0
    \right\}
    \tag{3.2.3 a} \\
    &\left\{E\subset\reals^{n}:
    \inf_{G\open\supset{}E\supset{}F\closed}
    \;m^{\phantom{*}}\left( G-F \right) = 0
    \right\}
    \tag{3.2.3 b} \\
    &\left\{E\subset\reals^{n}:
    \inf_{\phantom{G\open\supset}E\supset{}F\closed}
    \;m^{*}\left( E-F \right) = 0
    \right\}
    \tag{3.2.3 c}
\end{align*}
and are exactly the \keyphrase{Lebesgue measurable sets},
$\boldsymbol{\mathcal{L}\left( \reals^{n} \right)}$.  The following hold:
\begin{align*}
    \mathcal{L} \text{ is a $\sigma$-algebra, }
    \Borel{\reals^{n}}\subset\mathcal{L}\left( \reals^{n} \right).
    \tag{3.2.1 i} \\
    m \coloneqq \left.m^{*}\right|_{\mathcal{L}} \text{ is a measure. }
    \tag{3.2.1 ii} \\
    Z\subset\reals^{n}, m^{*}\left( Z \right) = 0
    \implies Z \in \mathcal{L}\left( \reals^{n} \right)
    \tag{Corollary}
\end{align*}

Lebesgue measurable sets, $\mathcal{L}\left( \reals^{n} \right)$
can also be characterized as
\begin{align*}
    &\left\{ H - Z
        : \text{$H$ is $G_{\delta}$-type },
        m^{*}\left( Z \right) = 0
    \right\}
    & &\text{where $H=\bigcap_{i=1}^{\infty} G_{i}$ for $G_{i}\open$}
    \tag{a} \\
    &\left\{ J \cup Z
        : \text{$J$ is $F_{\sigma}$-type },
        m^{*}\left( Z \right) = 0
    \right\}
    & &\text{where $J=\bigcup_{i=1}^{\infty} F_{i}$ for $F_{i}\closed$}
    \tag{b} \\
    &S\left(
        \Borel{\reals^{N}}
        \cup
        \left\{ Z :  m^{*}\left( Z \right) = 0 \right\}
    \right)
    & &\text{generation of a $\sigma$-algebra}
    \tag{c}
\end{align*}

\[
    F\subset\reals^{m}, Z\subset\reals^{n}, m^{*}_{n}\left( Z \right) = 0
    \implies m^{*}_{m+n} \left( F\times{}Z \right) = 0
\]
\[
    E_{1}\in\mathcal{L}\left( \reals^{n} \right),
    E_{2}\in\mathcal{L}\left( \reals^{m} \right)
    \implies
    E_{1}\times{}E_{2}\in\mathcal{L}\left( \reals^{n+m} \right)
\]
\[
    m_{n+m}\left( E_{1}\times{}E_{2} \right) =
    m_{n}\left( E_{1} \right)\cdot m_{m}\left( E_{2} \right)
\]

\section*{Properties of measurable (Borel) functions (Prop. 3.4.1)}

$f:\reals^{n}\supset{}E\to\closure{\reals}$ is \keyphrase{measurable}
iff both
\begin{align*}
    &\text{$E$ is measurable} \tag{a} \\
    &\boldsymbol{\left\{ y < f(x) \right\}} \coloneqq
    \left\{ (x,y)\in\reals^{n}\times\reals:x\in{}E, y < f(x) \right\}
    \text{ is measurable. } \tag{b}
\end{align*}
from which it follows that
\begin{align*}
    E\subset\dom{}f \measurable{}, f \measurable
    &\implies \left.f\right|_{E} \measurable
    \tag{3.4.1 i} \\
    f_{i}:E_{i}\to\closure{\reals}\;\measurable,E_{i}\text{ pairwise disjoint}
    &\implies \bigcup_{i=1}^{\infty} f_{i} 
    : \bigcup_{i=1}^{E_{i}} \to \closure{\reals} \measurable
    \tag{3.4.1 ii} \\
    f \measurable, \lambda\in\reals
    &\implies \lambda{}f \measurable
    \tag{3.4.1 iii} \\
    f_{i}:E\to\closure{\reals}\,\measurable &\implies 
    \sup_{i} f_{i},\, \inf_{i} f_{i},\, \limsup_{i} f_{i},\, \liminf_{i} f_{i}
    \;\measurable
    \tag{3.4.1 iv}
\end{align*}
Similar properties hold for Borel sets.

\[
    \dom f = E \open, f \continuous
    \implies f \textrm{ Borel } \implies f \measurable
\]
\[
    g:\reals^{n}\to\reals^{n} \text{ an affine isomorphism },
    f \measurable \iff f\circ{}g \measurable
\]

Property $P\left( x \right)$ is satisfied \keyphrase{almost everywhere}
($\almosteverywhere$) in $\reals^{n}$ iff 
$m\left(\left\{ x\in\reals^{n} : \neg{}P\left( x \right) \right\}\right) = 0$.
\[
    f_{1} = f_{2} \,\almosteverywhere \text{on } E\subset\reals^{n}
    \implies \left( f_{1} \measurable \iff f_{2} \measurable \right)
\]

\section*{Properties of Lebesgue integral (Prop. 3.5.1)}

For $f \measurable, f\geq{}0,$ the 
\keyphrase{Lebesgue integral} is defined as
\[
    \int f \, dm \coloneqq m_{n+1}\left(S\left(f\right)\right) = 
    m_{n+1}\left(\left\{ 
        (x,y)\in\reals^{n}\times\reals : x\in\dom f, 0 < y < f(x)
    \right\}\right)
\]
Every nonnegative
measurable function is Lebesgue integrable.
For $f \measurable$ with arbitrary sign, define
\begin{align*}
    f^{+}(x) &\coloneqq \left\{ 
        \max\left( f(x), 0 \right) : x\in\dom{}f, f(x)\geq{}0 
    \right\}
    \\
    f^{-}(x) &\coloneqq \left\{ 
        \max\left( -f(x), 0 \right) : x\in\dom{}f, f(x)<0 
    \right\}
\end{align*}
which allows definition of the integral as
\[
    \int f \, dm = \int f^{+} \, dm - \int f^{-} \, dm
\]
provided that $\int f \, dm \neq \infty - \infty$.  

For measurable, nonnegative functions the following properties hold
\begin{align*}
    m(E) = 0 &\implies \int_{E} \varphi \, dm = 0
    \tag{3.5.1 i} \\
    \varphi:E\to\closure{\reals}, E_{i}\subset{}E \text{ pairwise disjoint }
    &\implies 
    \int_{\bigcup{}E_{i}}\varphi\,dm 
    = \sum_{1}^{\infty} \int_{E_{i}} \varphi \, dm
    \tag{3.5.1 ii} \\
    \varphi,\psi:E\to\closure{\reals}, \varphi=\psi \almosteverywhere \in E
    &\implies
    \int_{E}\varphi\,dm = \int_{E}\psi\,dm
    \tag{3.5.1 iii} \\
    c\geq{}0, E \measurable 
    &\implies
    \int_{E}c\,dm = c\,m(E)
    \tag{3.5.1 iv} \\
    \varphi,\psi:E\to\closure{\reals}, \varphi\leq\psi \almosteverywhere \in E
    &\implies
    \int_{E}\varphi\,dm \leq \int_{E}\psi\,dm
    \tag{3.5.1 v} \\
    \lambda\geq{}0
    &\implies
    \int\left( \lambda \varphi \right) \, dm = \lambda \int \varphi \, dm
    \tag{3.5.1 vi}
\end{align*}


\section*{Fatou's Lemma}

\section*{Lebesgue Dominated Convergence Theorem (for non-negative functions, Thm. 3.5.2)}

\section*{H\"older and Minkowski inequalities}

\keyphrase{H\"older Inequality}: 
If $\Omega\subset\reals^{n}\,\measurable,
f,g:\Omega\to\closure{\reals}\,\measurable$ with
$\int_{\Omega}\abs{f}^{p}\,dm, \int_{\Omega}\abs{g}^{q}\,dm< \infty$ where
$p,q>1 : \frac{1}{p}+\frac{1}{q}=1$ then $\int_{\Omega}fg\,dm < \infty$ and
\[
    \abs{\int_{\Omega}fg\,dm}
    \leq 
    \left( \int_{\Omega} \abs{f}^{p}\,dm \right)^{\frac{1}{p}}
    \left( \int_{\Omega} \abs{g}^{q}\,dm \right)^{\frac{1}{q}}.
\]

For $1\leq{}p<\infty$ define $\boldsymbol{\norm{f}_{p} 
\coloneqq \left( \int_{\Omega} \abs{f}^{p}\,dm \right)^{\frac{1}{p}}}$

\keyphrase{Minkowski Inequality}:
If $\Omega\subset\reals^{n}\,\measurable$ and
$f,g:\Omega\to\closure{\reals}\,\measurable$ with 
$\norm{f}_{p}, \norm{g}_{p}< \infty$ where $p>1$ then
$\norm{f+g}_{p} < \infty$ and
\[
    \norm{f+g}_{p} \leq \norm{f}_{p}+\norm{g}_{p}.
\]

\section*{Properties of open sets, properties of closed sets, properties of the operations of interior and closure (all in context of general topological spaces)}

\section*{Characterization of open and closed sets in a topological subspace}

\section*{Characterization of (globally) continuous functions (Prop. 4.3.2)}

\section*{Properties of compact sets}

\section*{The Heine-Borel Theorem}

\section*{The Weierstrass Theorem}

\section*{Properties of sequentially compact sets (Prop. 4.4.5)}

\end{document}
