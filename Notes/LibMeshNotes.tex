%        File: LibMeshNotes.tex
%     Created: Fri Aug 05 02:00 PM 2011 C
% Last Change: Fri Aug 05 02:00 PM 2011 C
%
\documentclass[a4paper]{article}
\usepackage[fleqn]{amsmath}
\usepackage{amsfonts,amssymb,amsthm}

\providecommand{\abs}[1]{\left\lvert#1\right\rvert}
\providecommand{\norm}[1]{\left\lVert#1\right\rVert}

\def\d{\,\mathrm{d}}
\def\dx{\d x}
\def\ds{\d s}
\def\btau{\boldsymbol\tau}
\def\bsigma{\boldsymbol\sigma}
\def\bbeta{\boldsymbol\beta}
\def\bpsi{\boldsymbol\psi}
\def\bn{\mathbf{n}}
\def\dtau{\boldsymbol{\delta\tau}}
\def\dv{\delta v}

\def\div{\nabla\cdot}

\begin{document}
\section*{Strong Form}
\begin{align*}
\frac{1}{\epsilon}\bsigma-\nabla u=0 & \quad\text{ in }\Omega\\
-\nabla\cdot(\bsigma-\bbeta u)=f & \quad\text{ in }\Omega\\
u=u_0 & \quad\text{ in }\partial\Omega\\
\end{align*}
\section*{Weak Form (Residual Form)}
\subsection*{CG}
\begin{align*}
F_u&=\int_\Omega(\bsigma-\bbeta u)\cdot\nabla v-fv\dx
- \int_{\partial\Omega }(\sigma_n-\beta_nu)v\ds
&=0 & \quad\forall v\\
F_\sigma&=\int_\Omega\frac{1}{\epsilon}\bsigma\cdot\btau+u\nabla\cdot\btau\dx
-\int_{\partial \Omega}u\btau\cdot n\ds&=0 & \quad\forall\btau\\
\end{align*}
Now let
\[
u=\sum_iu_i\phi_i\,,\quad\quad
\bsigma=\sum_i\bsigma_i\bpsi_i
\]
To calculate jacobians, differentiate with respect to $u_i$ and $\bsigma_i$
\begin{align*}
K_{uu}=\int_\Omega-\bbeta\phi\cdot\nabla v\dx-\int_{\partial\Omega}-\beta_n\phi v\ds
\end{align*}
\begin{align*}
K_{u\sigma}=\int_\Omega\bpsi\cdot\nabla v\dx-\int_{\partial\Omega}\psi_nv\ds
\end{align*}
\begin{align*}
K_{\sigma
u}=\int_\Omega\phi\nabla\cdot\btau\dx-\int_{\partial\Omega}\phi\btau\cdot\bn\ds
\end{align*}
\begin{align*}
K_{\sigma\sigma}=\int_\Omega\frac{1}{\epsilon}\bpsi\cdot\tau\dx
\end{align*}
\subsection*{DPG}
\begin{align*}
F_u&=\int_K(\bsigma-\bbeta u)\cdot\nabla v-fv\dx
- \int_{\partial K}\widehat{(\sigma_n-\beta_nu)}\text{sgn}(n)v\ds
&=0 & \quad\forall v\\
F_\sigma&=\int_K\frac{1}{\epsilon}\bsigma\cdot\btau+u\nabla\cdot\btau\dx
-\int_{\partial K}\hat u\btau\cdot n\ds&=0 & \quad\forall\btau\\
\end{align*}
% I'm not sure how to calculated the jacobians for DPG. If I follow the same
% strategy, I get $K_{uu}\,,K_{u\sigma}\,,K_{u\hat
% u}\,,K_{u\widehat{(\sigma_n-\beta_nu)}}\,,K_{\sigma
% u}\,,K_{\sigma\sigma}\,,K_{\sigma\hat
% u}\,,K_{\sigma\widehat{(\sigma_n-\beta_nu)}}$, but no
% $K_{\hat u\dots}\,,\text{ or }K_{\widehat{(\sigma_n-\beta_nu)}\dots}$ terms.
% Furthermore, I know that the jacobian should be symmetric and positive-definite,
% something that this approach is not producing. So how do I derive the
% residuals/jacobians for DPG?
% Let
% \[
% \hat u=\sum_i\hat
% u_i\Phi_i\,,\quad\quad\widehat{(\sigma_n-\beta_nu)}=\sum_i\widehat{(\sigma_n-\beta_nu)}_i\Psi_i
% \]
% \begin{align*}
% K_{uu}&=\left[\int_K-\bbeta\phi\cdot\nabla v\dx\,,\quad0\right]\\
% K_{u\sigma}&=\left[\int_K\bpsi\cdot\nabla v\dx\,,\quad-\int_{\partial
% K}\Psi\text{sgn}(n)v\ds\right]\\
% K_{\sigma u}&=\left[\phi\nabla\cdot\dx\,,\quad-\int_{\partial
% K}\Phi\btau\cdot\bn\ds\right]\\
% K_{\sigma\sigma}&=\left[\int_K\frac{1}{\epsilon}\bpsi\cdot\btau\dx\,,\quad0\right]
% \end{align*}
\end{document}


