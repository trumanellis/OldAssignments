\documentclass[letterpaper]{article}
\usepackage[top=.5in, bottom=.5in, left=.5in, right=.5in]{geometry}
\usepackage{amsmath,amsfonts,amssymb,amsthm}
\usepackage{multicol}
\usepackage{cancel}
\usepackage{SmallPar}

\setlength{\parindent}{0in}
\setlength{\parskip}{0ex}
\setlength{\jot}{.5ex} 

\providecommand{\abs}[1]{\left\lvert#1\right\rvert}
\providecommand{\norm}[1]{\left\lVert#1\right\rVert}
\providecommand{\inner}[1]{\left(#1\right)}
\providecommand{\ainner}[1]{\left\langle#1\right\rangle}

\def\d{\mathrm{d}}

\pagestyle{empty}
\begin{document}
\begin{multicols}{2}
\footnotesize
\paragraph{Relations}
\[
\Delta U=\Delta Q-\Delta W
\]
\[
dU=TdS-pdV+\mu dN
\]
\[
F=U-TS
\]
\[
P=-\left(\frac{\partial F}{\partial V}\right)_{T,N}
\]
\[
S=-\left(\frac{\partial F}{\partial T}\right)_{V,N}
\]
Microcanonical
\[
\Omega(E)=\int_{x,p}\delta(E-H(x,p))\frac{\d x\d p}{h}
\]
$P_E$ is constant since each state has same probability
\[
S=\log(\Omega(E)\Delta E)=-\log(P_E)+\log(\Delta E)=-\sum P_E\log(P_E)
\]
Canonical
\[
T(E)=\frac{\Omega(E)}{\frac{\d\Omega}{\d E}}=\left(\frac{\d S}{\d
E}\right)^{-1}
\]
\[
Q=\sum_ne^{-\beta E_n}
\]
\[
\ainner{E}=-\frac{\partial}{\partial\beta}\log(Q)
\]
\[
p_n=\frac{e^{-\beta E_n}}{Q}\Rightarrow
E_n=-\frac{1}{\beta}\log(p_n)-\frac{1}{\beta}\log(Q)
\]
\begin{align*}
\d\ainner{E}&=\sum_n\d p_n E_n\\
&=\sum_n\d p_n\left[-\frac{1}{\beta}\log(p_n)-\frac{1}{\beta}\log(Q)\right]\\
&=-\frac{1}{\beta}\sum_n\d p_n\log(p_n)
-\frac{1}{\beta}\log(Q)\d\left(\sum_n p_n\right)\\
&=-\frac{1}{\beta}\sum_n\d p_n\log(p_n)\\
&=-\frac{1}{\beta}\d\left[\sum_n p_n\log(p_n)\right]\\
\Rightarrow&S=-\sum_n p_n\log(p_n)
\end{align*}
\[
F=E-TS=-T\log(Q)
\]

\problem{Derivation of Grand Canonical}
\[
\log[P(E_T,N_T)]\propto\log[\Omega(E-E_T,N-N_T)]
\]
\begin{align*}
&\log[\Omega(E-E_T,N-N_T)]\approx\log\left[\Omega(E,N)
-\frac{\d\Omega}{\d E}E_t-\frac{\d\Omega}{\d N}N_T\right]\\
&=\log\left[\Omega(E,N)\left(1-\frac{1}{\Omega(E,N)}\frac{\d\Omega}{\d E}E_T
-\frac{1}{\Omega(E,N)}\frac{\d\Omega}{\d N}N_T\right)\right]\\
&=\log[\Omega(E,N)]+\log\left(1-\frac{1}{\Omega(E,N)}\frac{\d\Omega}{\d E}E_T
-\frac{1}{\Omega(E,N)}\frac{\d\Omega}{\d N}N_T\right)\\
&=\log[\Omega(E,N)]+\log(1-\beta E_T-\mu N_T)
\end{align*}
\[
\ainner{N}=-\frac{\partial}{\partial\gamma}\log(Q)
\]

\problem{Particle in a Box}
\[
\phi_n=\sqrt{\frac{2}{L}}\sin\left(\frac{n\pi x}{L}\right)
\]
\[
E_n=\frac{n^2\pi^2\hbar^2}{2mL^2}
\]

\problem{Harmonic Oscillator}
\[
E_n=\hbar\sqrt{\frac{k}{m}}\left(n+\frac{1}{2}\right)
=\hbar\omega\left(n+\frac{1}{2}\right)
\]
\[
x(t)=\cos(\omega t+\phi)
\]

\problem{Poisson Equation}
\[
\nabla^2\phi=-\frac{4\pi\rho}{\epsilon}
\]

\problem{Ideal Gas Law}
Show that if the canonical partition function $Q$ takes the form $Q=f(T)V^N$,
the ideal gas law can be obtained.
\solution{Solution}
\begin{align*}
F&= -kT\log Q=-kT\log(f(T)V^N)\\
P&=-\frac{\partial F}{\partial V}=kT\frac{f(T)NV^{N-1}}{f(T)V^N}=\frac{kTN}{V}
\end{align*}

\problem{Heat Capacity}
Show that the heat capacity
\[
C=\frac{\partial\ainner{E}}{\partial T}
\]
is given in the canonical ensemble by
\[
C=\frac{\ainner{E^2}-\ainner{E}^2}{kT^2}
\]
\solution{Solution}
\[
\frac{\partial\ainner{E}}{\partial T}=\frac{\partial\beta}{\partial T}
\frac{\partial\ainner{E}}{\partial\beta}
\]
\[
\frac{\partial}{\partial T}\left(\frac{1}{kT}\right)
=-\frac{1}{kT^2}
\]
\begin{align*}
\frac{\partial\ainner{E}}{\partial\beta}&=\frac{\partial}{\partial\beta}\left(
\frac{\sum_nE_ne^{-\beta E_n}}{\sum_ne^{-\beta E_n}}\right)\\
&=\frac{\sum_n-E_n^2e^{-\beta E_n}}{\sum_ne^{-\beta E_n}}
+\left(\frac{\sum_nE_ne^{-\beta E_n}}{\sum_ne^{-\beta E_n}}\right)^2\\
&=-\ainner{E^2}+\ainner{E}^2
\end{align*}
\begin{align*}
C_v&=-\frac{1}{kT^2}(-\ainner{E^2}+\ainner{E}^2)\\
&=\frac{1}{kT^2}(\ainner{E^2}-\ainner{E}^2)
\end{align*}

\problem{Enthalpy}
The constant pressure ensemble is defined by the enthalpy $H=E+PV$, where $H$ is
the enthalpy, $E$ is the energy, $P$ is the pressure (which is a parameter like
the temperature in the canonical ensemble), and $V$ is the volume, which is a
variable (not constant). The weight of a configuration is given by $e^{-\beta
E-\beta PV}$. Find an expression for the fluctuations of the enthalpy in terms
of the derivative of the average enthalpy with respect to the temperature.
\solution{Solution}
Partition Function:
\[
Q=\sum_je^{-\beta E_j-\beta PV_j}=\sum_je^{-\beta H_j}
\]
Average enthalpy:
\[
\ainner{H}=\frac{\sum_j H_j e^{-\beta H_j}}{\sum_j e^{-\beta H_j}}
=\frac{-\frac{\partial}{\partial\beta}(Q)}{Q}
=-\frac{\partial}{\partial\beta}(\log Q)
\]
\[
\frac{\partial Q}{\partial\beta}=\sum_j-H_je^{-\beta H_j}
\]
\[
\frac{\partial^2 Q}{\partial\beta^2}=\sum_j H_j^2 e^{-\beta H_j}
\]
\[
\ainner{H^2}=\frac{\frac{\partial^2 Q}{\partial\beta^2}}{Q}
\]
\begin{align*}
\frac{\partial\ainner{H}}{\partial\beta}
&=-\frac{\partial}{\partial\beta}
\left(\frac{\frac{\partial Q}{\partial\beta}}{Q}\right)
=-\frac{Q\frac{\partial^2Q}{\partial\beta^2}-\left(\frac{\partial
Q}{\partial\beta}\right)^2}{Q^2}\\
&=\left(\frac{\frac{\partial
Q}{\partial\beta}}{Q}\right)^2-\frac{\frac{\partial^2Q}{\partial\beta^2}}{Q}
=\ainner{H}^2-\ainner{H^2}
\end{align*}
Now $\beta=\frac{1}{kT}$, so $T=\frac{1}{k\beta}$
\[
\frac{\partial T}{\partial\beta}=-\frac{1}{k\beta^2}=-\frac{(kT)^2}{k}=-kT^2
\]
\[
\sigma^2_H=\ainner{H^2}-\ainner{H}^2=-\frac{\partial\ainner{H}}{\partial\beta}
=-\frac{\partial T}{\partial\beta}\frac{\partial\ainner{H}}{\partial T}
=kT^2\frac{\partial\ainner{H}}{\partial T}
\]

\problem{Master Equation} Given a kinetic model for the probability at state n:
\[
\frac{\d p_n}{\d t}=\sum_m k(p_m-p_n)
\]
show that the change in entropy $S=-\sum_n p_n\log(p_n)$ as a function of time
is always nonnegative, $\frac{\d S}{\d t}\ge0$.
\solution{Solution}
\begin{align*}
\frac{\d S}{\d t}&=-\sum_n\left[\frac{\d p_n}{\d t}+\frac{\d p_n}{\d
t}\log(p_n)\right]\\
&=-\frac{\d}{\d t}\sum_n p_n
-\sum_n\frac{\d p_n}{\d t}\log(p_n)\\
&=-\sum_n\frac{\d p_n}{\d t}\log(p_n)\\
&=-\sum_n\log(p_n)\sum_mk(p_m-p_n)\\
&=-\sum_m\log(p_m)\sum_nk(p_n-p_m)\\
&=-\frac{1}{2}k\sum_{m,n}[\log(p_n)(p_m-p_n)+\log(p_m)(p_n-p_m)]\\
&=-\frac{1}{2}k\sum_{m,n}[(\log(p_n)-\log(p_m))(p_m-p_n)]\\
&=\frac{1}{2}k\sum_{m,n}[(\log(p_m)-\log(p_n))(p_m-p_n)]\\
\end{align*}

\problem{Liouville Theorem} 
The probability density of phase space in classical mechanics is $\rho(x,p,t)$.
Write down the complete derivative of the density with respect to time. Use the
Hamilton equations. Integrate overa  phase space volume element $\Delta x\Delta
p$ and use the conservation of probability and Gauss therorem to prove the
Liouville theorem.
\solution{Solution}
\begin{align*}
\dot x&=\frac{\partial H}{\partial p}\\
\dot p&=-\frac{\partial H}{\partial x}
\end{align*}
\begin{align*}
\frac{\d\rho(x,p,t)}{\d t}&=\frac{\partial\rho}{\partial
t}+\frac{\partial\rho}{\partial x}\frac{\partial x}{\partial t}
+\frac{\partial\rho}{\partial p}\frac{\partial p}{\partial t}\\
&=\frac{\partial\rho}{\partial t}
+\frac{\partial\rho}{\partial x}\frac{\partial H}{\partial p}
-\frac{\partial\rho}{\partial p}\frac{\partial H}{\partial x}\\
\end{align*}
\[
P=\int_{\Delta x\Delta p}\rho\d x\d p
\]
\begin{align*}
\frac{\partial P}{\partial t}
&=\int_{\Delta x\Delta p}\frac{\partial\rho}{\partial t}\d x\d p\\
&=-\int_S\rho(x,p,t)(\dot x\,,\dot p)\cdot\left(\begin{array}{c}
\d A_x \\ \d A_p \end{array}\right)
\intertext{By Gauss Theorem}
&=-\int_{\Delta x\Delta p}\frac{\partial\rho}{\partial x}\dot x
+\frac{\partial\rho}{\partial p}\dot p
+\rho\left(\cancel{\frac{\partial^2H}{\partial x\partial p}
-\frac{\partial^2H}{\partial p\partial x}}\right)\d x\d p\\
&=\int_{\Delta x\Delta p}\frac{\partial\rho}{\partial p}
\frac{\partial H}{\partial x}-\frac{\partial\rho}{\partial x}
\frac{\partial H}{\partial p}
\end{align*}
\[
\frac{\partial\rho}{\partial t}=\frac{\partial\rho}{\partial p}
\frac{\partial H}{\partial x}-\frac{\partial\rho}{\partial x}
\frac{\partial H}{\partial p}
\]
Substituting this into the original equation, everything cancels, and we get
\[
\frac{\d\rho}{\d t}=0
\]

\problem{Lagrange Multipliers}
The weight of a particular arrangement from an ensemble is given by
\[
W=\frac{L!}{\prod_jn_j!}
\]
where L is the number of subsystems and $n_j$ is the number of times a system
with energy $E_j$ is observed in the arrangement. Find the most probable
arrangement under the constraints that the total number of systems and total
energy are fixed. Determine the probability to sample energy $E_j$.
\solution{Solution}
\[
L=\sum_in_i
\]
\[
E=\sum_in_iE_i
\]
We will use Lagrange multipliers and Sterlings approx: $\log(x!)=x\log x-x$
Maximize $\log W$:
\begin{align*}
\log W-\alpha L-\beta E&=\log\left(\frac{L!}{\prod_jn_j!}\right)
-\alpha\sum_in_i-\beta\sum_in_iE_i\\
&=\log(L!)-\sum_i\log(n_i!)-\alpha\sum_in_i-\beta\sum_in_iE_i\\
&=\left(\sum_in_i\right)\log\left(\sum_in_i\right)-\sum_in_i\\
&-\sum_i\left(n_i\log n_i-n_i+\alpha n_i+\beta n_iE_i\right)
\end{align*}
\[
\frac{\partial}{\partial n_i}(\log W-\alpha L-\beta E)
=-\log n_i-\alpha-\beta E_i=0
\]
\[
n_i=e^{-\alpha-\beta E_i}
\]
\[
P_{E_j}=\frac{e^{-\alpha-\beta E_j}}{\sum_ne^{-\alpha-\beta E_n}}
=\frac{e^{-\beta E_j}}{\sum_ne^{-\beta E_n}}
\]
\[
\Omega(E)\propto E^N
\]
\[
P(E)\propto\Omega(E)e^{-\beta E}\propto E^Ne^{-\beta E}
\]
\begin{align*}
\frac{\d}{\d E}\log P(E)&=\frac{\d}{\d E}\left[\log(E^N)-\beta E\right]\\
&=\frac{NE^{N-1}}{E^N}-\beta=0
\end{align*}
Most probable energy
\[
E^*=\frac{N}{\beta}
\]

\problem{Equilibrium} Use the Liouvill equation to show that if the phase space
probability density in classical mechanics is a function of the Hamiltonian
only, then the system must be in equilibrium.
\solution{Solution} Liouville Equation:
\[
\frac{\partial\rho}{\partial t}=\frac{\partial\rho}{\partial x}
\frac{\partial H}{\partial p}-\frac{\partial\rho}{\partial p}
\frac{\partial H}{\partial x}
\]
\begin{align*}
\frac{\partial\rho}{\partial x}&=\frac{\partial\rho}{\partial H}
\frac{\partial H}{\partial x}\\
\frac{\partial\rho}{\partial p}&=\frac{\partial\rho}{\partial H}
\frac{\partial H}{\partial p}\\
\end{align*}
\begin{align*}
\frac{\partial\rho}{\partial t}&=\frac{\partial\rho}{\partial H}
\frac{\partial H}{\partial x}\frac{\partial H}{\partial p}
-\frac{\partial\rho}{\partial H}
\frac{\partial H}{\partial p}\frac{\partial H}{\partial x}\\
&=\frac{\partial\rho}{\partial H}\left(
\frac{\partial H}{\partial x}\frac{\partial H}{\partial p}
-\frac{\partial H}{\partial p}\frac{\partial H}{\partial x}\right)\\
&=0
\end{align*}

\problem{Harmonic Oscillator}
Consider a quantum harmomic oscillator with Hamiltonian
\[
H=-\frac{\hbar^2}{2m}\frac{\d^2}{\d x^2}+\frac{1}{2}m\omega^2x^2
\]
Write down the energy levels and derive the partition function in the canonical
ensemble.
\solution{Solution}
\[
E_n=\hbar\omega\left(n+\frac{1}{2}\right)
\]
\begin{align*}
Q&=\sum_ne^{-\beta\hbar\omega\left(n+\frac{1}{2}\right)}\\
&=e^{-\frac{\beta\hbar\omega}{2}}\sum_ne^{-\beta\hbar\omega n}\\
&=\frac{e^{-\frac{\beta\hbar\omega}{2}}}
{1-e^{-\beta\hbar\omega}}
\end{align*}
As $\beta\rightarrow0$
\[
Q\rightarrow\frac{1-\beta\hbar\omega/2}{1-1+\beta\hbar\omega}
=\frac{1}{\beta\hbar\omega}-\frac{1}{2}\rightarrow\frac{2\pi}{\beta h\omega}
\]
Classical:
\begin{align*}
\int e^{-\beta H}\d x\d p&=\int e^{-\frac{\beta p^2}{2m}}\d p
\int e^{-\frac{\beta m\omega^2x^2}{2}}\d x\\
&=\sqrt{\frac{2m\pi}{\beta}}\sqrt{\frac{2\pi}{\beta m\omega^2}}\\
&=\frac{2\pi}{\beta\omega}
\end{align*}

\problem{Ergodic Hypothesis} Consider the classical harmonic oscillator with energy 
\[
E=\frac{1}{2}m\dot x^2+\frac{1}{2}\underbrace{k}_{m\omega^2}x^2
\]
Compute the average of the kinetic energy over time and over the microcanonical
ensemble. Show that both averages are the same.
\solution{Solution}
Temporal average:
\begin{align*}
x(t)&=A\cos(\omega t+\phi)\\
v(t)&=-\omega A\sin(\omega t+\phi)
\end{align*}
\[
E=\frac{1}{2}m\omega^2A^2
\]
\[
K=E\sin^2(\omega t+\phi)
\]
\begin{align*}
\ainner{K}_t&=\frac{1}{t}\int_{t_0}^{t_0+t}E\sin^2(\omega\tau+\phi)\d\tau\\
&=\frac{E}{\omega t}\left[-\frac{1}{4}\sin(2\omega\tau+2\phi)
+\frac{\omega\tau+\phi}{2}\right]_{\omega t_0}^{\omega t_0+\omega t}\\
&=\frac{E}{\omega t}\left[-\frac{1}{4}\sin(2\omega t+\Delta)
+\frac{1}{4}\sin(\Delta)+\frac{\omega t}{2}\right]\\
&=\frac{E}{2}-\frac{E}{4\omega t}\left[\sin(2\omega t+\Delta)-\sin(\Delta)\right]
\end{align*}
\[
\lim_{t\rightarrow\infty}\ainner{K}_t=\frac{E}{2}
\]
Phase space average:
\[
\ainner{K}_S=\frac{\int\frac{1}{2}mv^2
\delta\left(E-\frac{1}{2}mv^2-\frac{1}{2}kx^2\right)\d x\d v}
{\int\delta\left(E-\frac{1}{2}mv^2-\frac{1}{2}kx^2\right)\d x\d v}
\]
Let $Z_1=\sqrt{\frac{m}{2}}v$, $Z_2=\sqrt{\frac{k}{2}}x$.
\[
\int\delta\left(E-\frac{1}{2}mv^2-\frac{1}{2}kx^2\right)\d x\d v
=\frac{2}{\sqrt{mk}}\int\delta\left(E-Z_1^2-Z_2^2\right)\d Z_1\d Z_2
\]
Let $R^2=Z_1^2+Z_2^2$ and $\cos(\theta)=Z_1/R$.
\begin{align*}
\frac{2}{\sqrt{mk}}\int\delta\left(E-Z_1^2-Z_2^2\right)\d Z_1\d Z_2
&=\frac{2}{\sqrt{mk}}\int\delta\left(E-R^2\right)\d\theta R\d R\\
&=\frac{2\pi}{\sqrt{mk}}\int\delta(E-R^2)\d R^2\\
&=\frac{2\pi}{\sqrt{mk}}
\end{align*}
\begin{align*}
\int\frac{1}{2}mv^2 \delta\left(E-\frac{1}{2}mv^2-\frac{1}{2}kx^2\right)\d x\d v
&=\frac{1}{\sqrt{mk}}\int R^2\cos^2\theta\delta(E-R^2)\d\theta\d R^2\\
&=\frac{E\pi}{\sqrt{mk}}
\end{align*}
\[
\ainner{K}_S=\frac{E}{2}
\]

\problem{Imperfect Gas}
\[
Z=\int e^{-\beta U(r_1,r_2,\dots,r_N)}\prod_i\d r_i
\]
\[
U=\frac{1}{2}\sum_{i\neq j}u_{ij}(r_{ij})
\]
\[
f_{ij}=1-e^{-\beta u_{ij}}
\]
\begin{align*}
Z&\approx\int e^{-\beta\sum_{j,k}u_{jk}(r_{jk})}\prod_i\d r_i\\
&=\int\prod e^{-\beta u_{jk}(r_{jk})}\prod_i\d r_i\\
&=\int\prod(1-f_{jk}(r_{jk}))\prod_i\d r_i\\
&\approx\int\left(1-\frac{1}{2}\sum_{i\neq j}f_{ij}\right)\prod_i\d r_i\\
&=V^N-\frac{V^{N-2}}{2}\sum_{i\neq j}\int f_{ij}(r_{ij})\d r_i\d r_j\\
&=V^N\left(1-\frac{N(N-1)}{2V}\int\left(1-e^{-\beta u(r_{12})}\right)\d
r_{12}\right)
\end{align*}
\begin{align*}
p&=-\frac{\partial F}{\partial V}=\beta^{-1}\frac{\partial}{\partial V}\log(Q)
=\beta^{-1}\frac{\partial}{\partial V}\log(Z)\\
&=\beta^{-1}\frac{\partial}{\partial V}\left[
V^N\left(1-\frac{N(N-1)}{2V}\int4\pi r^2\left(1-e^{-\beta u(r)}\right)\d
r\right)\right]\\
&\approx\beta^{-1}\left[NV^{-1}-\frac{\partial}{\partial V}\frac{N^2}{V}
\int2\pi r^2\left(1-e^{-\beta u(r)}\right)\d r\right]\\
&=\beta^{-1}\frac{N}{V}+\beta^{-1}\frac{N^2}{V^2}
\int2\pi r^2\left(1-e^{-\beta u(r)}\right)\d r\\
&=kT\rho+kT\rho^2 B_2(T)
\end{align*}

\problem{Liquids}
\[
p=kT\frac{\partial}{\partial V}\log(Z)=kT\frac{\frac{\partial Z}{\partial V}}{Z}
\]
\[
r_k=V\bar r_k
\]
\[
Z=\int^V\cdots\int^Ve^{-\beta U}\prod_i\d r_i
=V^N\int^1\cdots\int^1e^{-\beta U}\prod_i\d\bar r_i
\]
\[
\frac{\partial Z}{\partial V}=NV^{N-1}\int^1e^{-\beta U}\prod_i\d\bar r_i
-\beta V^N\int^1\frac{\partial U}{\partial V}e^{-\beta U}\prod_i\d\bar r_i
\]
\[
\frac{\partial U}{\partial V}=\sum_{i>j}\frac{\partial u(r_{ij})}{\partial V}
=\sum_{i>j}\frac{\partial u(r_{ij})}{\partial r_{ij}}\frac{\partial
r_{ij}}{\partial V}
\]
\begin{align*}
r_{ij}&=\sqrt{(x_i-x_j)^2+(y_i-y_j)^2+(z_i-z_j)^2}=V^{1/3}\bar r_{ij}
\end{align*}
\[
\frac{\partial r_{ij}}{\partial V}
=\frac{\partial}{\partial V}V^{1/3}\bar r_{ij}
=\frac{\bar r_{ij}}{3V^{2/3}}
=\frac{r_{ij}}{3V}
\]
\begin{align*}
\frac{\partial}{\partial V}\log(Z)&=\frac{N}{V}-\frac{N(N-1)}{6VkT}
\int_V r_{12}\frac{\d u}{\d r_{12}}e^{-\beta\sum_{i\neq j}u_{ij}}
\d r_1\dots\d r_N\\
&=\frac{N}{V}-\frac{N(N-1)}{6VkTZ}\int_Vr_{12}\frac{\d u}{\d r_{12}}
\left[\int_Ve^{-\beta U}\d r_3\dots\d r_N\right]\d r_1\d r_2
\end{align*}

\problem{Particle in a Box}
\begin{align*}
Q&=\sum_{n_x,n_y,n_z}e^{-\beta\frac{\hbar^2\pi^2}{2MV^{2/3}}(n_x^2+n_y^2+n_z^2)}\\
&\approx\int_0^\infty e^{-\beta\frac{\hbar^2\pi^2}{2MV^{2/3}}(n_x^2+n_y^2+n_z^2)}
\d n_x\d n_y\d n_z\\
&=\frac{1}{8}\left(\frac{2MV^{2/3}}{\beta\hbar^2\pi}\right)^{3/2}
=\left(\frac{M}{2\beta\hbar^2\pi}\right)^{3/2}V
\end{align*}
N indistinguishable particles:
\[
Q=\frac{1}{N!}\left(\frac{M}{2\beta\hbar^2\pi}\right)^{3N/2}V^N
\]

\problem{Reaction}
\[
K=\frac{P_{Na_2}}{P_{Na}^2}=\frac{V}{kT}\frac{N_{Na_2}}{N_{Na}^2}
\]
\[
dN=dN_{Na_2}=-1/2dN_{Na}
\]
\[
dF=-\mu_{Na}dN_{Na}-\mu_{Na_2}dN_{Na_2}=(2\mu_{Na}-\mu_{Na_2})dN=0
\]
$2\mu_{Na}=\mu_{Na_2}$
\begin{align*}
\mu_{Na}&=-kT\frac{\partial}{\partial
N_{Na}}\log\left[\frac{Q_{Na}^{N_{Na}}Q_{Na_2}^{N_{Na_2}}}{N_{Na}!N_{Na_2}!}
\right]\\
&\approx-kT\frac{\partial}{\partial N_{Na}}[N_{Na}\log(Q_{Na})]
+kT\frac{\partial}{\partial N_{Na}}[N_{Na}\log(N_{Na})-N_{Na}]\\
&=-kT\log(Q_{Na})+kT\log(N_{Na})\\
&=kT\log\left(\frac{N_{Na}}{Q_{Na}}\right)
\end{align*}
\[
\frac{Q_{Na_2}}{Q_{Na}^2}=\frac{N_{Na_2}}{N_{Na}^2}
\]

\problem{Poisson-Boltzman Equation}
\begin{align*}
\nabla^2\phi(r)&=-\frac{4\pi}{\epsilon}\sum_sc_sq_se^{-\beta q_s\phi(r)}\\
&\approx-\frac{4\pi}{\epsilon}\sum_sc_sq_s[1-\beta q_s\phi(r)]\\
&=\frac{4\pi\beta}{\epsilon}\sum_sc_sq_s^2\phi(r)=\kappa^2\phi(r)
\end{align*}
\[
\frac{1}{r^2}\frac{\d}{\d r}\left(r^2\frac{\d\phi}{\d r}\right)=\kappa^2\phi(r)
\]
\[
\phi(r)=A\frac{e^{-\kappa r}}{r}+\cancel{B\frac{e^{\kappa r}}{r}}
\]
\[
\phi(r)=\frac{q}{\epsilon r(1+\kappa a)}e^{-\kappa(r-a)}
\]
\[
Q=\int_a^\infty\rho(r)4\pi r^2\d r
=-\frac{\epsilon}{4\pi}\int_a^\infty(\nabla^2\phi)4\pi r^2\d r
\]
Probability to find ion within $r+\d r$ is
\[
\rho(r)4\pi r^2\d r=\frac{q\kappa^2r}{1+\kappa a}e^{-\kappa(r-a)}\d r
\]

\problem{Boltzmann Equation}
\[
\frac{\partial f_i}{\partial t}+v_i\frac{\partial f_i}{\partial
r}+\frac{F_i}{m_i}\frac{\partial f_i}{\partial v_i}
=\sum_j\int[f_i'f_j'-f_if_j]g_{ij}2\pi b\d b\d v_i
\]
\[
H(t)=\int f(r,v,t)\log[f(r,v,t)]\d r\d v
\]
\begin{align*}
\frac{\d H}{\d t}&=\int\frac{\partial f}{\partial t}\log(f)\d r\d v
+\cancel{\int\frac{\partial f}{\partial t}\d r\d v}\\
&=\cancel{-\int\log(f)v\frac{\partial f}{\partial r}\d r\d v
-\int\log(f)\frac{F}{m}\frac{\partial f}{\partial v}\d r\d v}\\
&+\int\log(f)[f'f_1'-ff_1]g2\pi b\d b\d v\d r\\
&=\int\log(f)[f'f_1'-ff_1]g2\pi b\d b\d v\d r\\
&=\int\log(f_1)[f'f_1'-ff_1]g2\pi b\d b\d v\d r\\
&=-\int\log(f')[f'f_1'-ff_1]g2\pi b\d b\d v\d r\\
&=-\int\log(f_1')[f'f_1'-ff_1]g2\pi b\d b\d v\d r\\
&=\frac{1}{4}\int\log\left(\frac{ff_1}{f'f_1'}\right)[f'f_1'-ff_1]2\pi b\d b\d
r\d v
\end{align*}
Work out Verlet Algorithm on paper

\problem{Umbrella Sampling}
\begin{align*}
P(q_0)&=\frac{\ainner{\delta(q(X)-q_0)e^{\beta
V(q)}}_{U+V}}{\ainner{e^{\beta V(q)}}_{U+V}}\\
&=\frac{e^{\beta V(q_0)}\ainner{\delta(q(X)-q_0)}_{U+V}}{\ainner{e^{\beta
V(q)}}_{U+V}}
\end{align*}

\begin{description}
\item[Prokaryotic cell] Primitive cell without a nucleus
\item[Cytoplasm] Internal cell fluid - no compartments
\item[Cell wall] Rigid layer surrounding some cells. Prevents over expansion due
to osmotic pressure
\item[Cell membrane] Collection of phospholipids that form a bilayer and define
the boundary between inside and outside of all cells
\item[Capsule] External layer on prokaryotic cells. Made of sugar, extra
protection
\item[Ribosome] Compartment in which protein synthesis takes place from
information provided by RNA
\item[Golgi apparatus] Organelle that packages biomolecules for delivery.
\item[Golgi vesicles] Vesicles that transport enzymes and other material through
ER to Golgi body
\item[Smooth ER] Synthesizes lipids and steroids
\item[Rough ER] Synthesizes proteins
\item[Nucleus] Where genetic material is stored
\item[Mitochondrion] Organelle that generates ATP
\item[Lysosome] Organelle that breaks up waste materails
\end{description}
\end{multicols}
\end{document}

