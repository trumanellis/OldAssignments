\documentclass[letterpaper,10pt]{article}
\usepackage{amsfonts,amsmath,amssymb,amsthm}
\usepackage{cancel}
\usepackage{graphicx}
\usepackage{multirow}

\setlength{\parindent}{0in}
\setlength{\parskip}{.4ex}

\DeclareMathOperator*{\tgrad}{grad}
\DeclareMathOperator*{\tGrad}{Grad}
\DeclareMathOperator*{\tdiv}{div}
\DeclareMathOperator*{\tDiv}{Div}
\DeclareMathOperator*{\tcurl}{curl}
\DeclareMathOperator*{\Cof}{Cof}
\DeclareMathOperator*{\tr}{tr}

\providecommand{\abs}[1]{\lvert#1\rvert}
\providecommand{\norm}[1]{\lVert#1\rVert}

\def\mathbi#1{\textbf{\em #1}}
\def\expect#1{\left\langle #1\right\rangle}
\def\d{\mathrm{d}}
\def\e{\mathrm{e}}

\def\mbf{\mathbf}
\def\mbb{\mathbb}

% Document Specific Shortcuts
\renewcommand*\arraystretch{1.5}
\def\enoo{$\frac{\hbar^2\pi^2}{ma^2}$}
\def\enot{$2.5\frac{\hbar^2\pi^2}{ma^2}$}
\def\enoth{$5\frac{\hbar^2\pi^2}{ma^2}$}
\def\enof{$8.5\frac{\hbar^2\pi^2}{ma^2}$}
\def\entt{$4\frac{\hbar^2\pi^2}{ma^2}$}
\def\entth{$6.5\frac{\hbar^2\pi^2}{ma^2}$}
\def\entf{$10\frac{\hbar^2\pi^2}{ma^2}$}
\def\enthth{$9\frac{\hbar^2\pi^2}{ma^2}$}
\def\enthf{$12.5\frac{\hbar^2\pi^2}{ma^2}$}
\def\enff{$16\frac{\hbar^2\pi^2}{ma^2}$}

\def\R{\mathbf{R}}
\def\r{\mathbf{r}}


%opening
\title{CAM 389C Exercise Set II.3}
\author{Truman Ellis}

\begin{document}

\maketitle

\subsection*{Problem II.2.7}
Which of the following operators are Hermitian?\\[.5ex]

\paragraph*{a)} $\displaystyle \e^{ix}$

\paragraph*{b)} $\displaystyle \frac{\d^2}{\d x^2}$

\paragraph*{c)} $\displaystyle x\frac{\d}{\d x}$

\subsubsection*{Solution}
\paragraph*{a)} An operator $Q$ is Hermitian if
\[
\expect{u,Qv}=\expect{Qu,v}\,.
\]
Also, in order to evaluate the inner product, the functions $u$ and $v$ and
their derivatives must vanish at $\pm\infty$. Then
\begin{align*}
\expect{u,\e^{ix}v}=\int_{-\infty}^\infty u^*\e^{ix}v\,\d x\,,
\end{align*}
and
\begin{align*}
\expect{\e^{ix}u,v}=\int_{-\infty}^\infty \e^{-ix}u^*v\,\d x\,,
\end{align*}
Clearly these are not going to evaluate identically for all functions $u(x)$
and $v(x)$. Thus, $\e^{ix}$ is not a Hermitian operator.

\paragraph*{b)}
Integrating by parts twice,
\begin{align*}
\expect{u,\frac{\d^2}{\d x^2}v}&=\int_\mathbb{R}u^*\frac{\d^2v}{\d x^2}\,\d x\\
&=\left.\cancel{u^*}\frac{\d v}{\d x}\right|_0^a
-\int_\mathbb{R}\frac{\d v}{\d x}\frac{\d u^*}{\d x}\,\d x\\
&=-\left.\frac{\d u^*}{\d x}\cancel{v}\right|_0^a
+\int_\mathbb{R}v\frac{\d^2u^*}{\d x^2}\,\d x\\
&=\expect{\frac{\d^2}{\d x^2}u,v}\,.
\end{align*}
Therefore $\frac{\d^2}{\d x^2}$ is a Hermitian operator.

\paragraph*{c)}
\begin{align*}
\expect{u,x\frac{\d}{\d x}v}&=\int_\mbb{R}u^*x\frac{\d v}{\d x}\,\d x &
\text{integrate by parts}\\
&=\left.\cancel{u^*xv}\right|_{-\infty}^\infty
-\int_\mbb{R}x\frac{\d u^*}{\d x}v\,\d x
-\int_\mbb{R}u^*v\,\d x\\
&=-\expect{x\frac{\d}{\d x}u,v}-\expect{u,v}
\neq\expect{x\frac{\d}{\d x}u,v}\,.
\end{align*}
Therefore,$x\frac{\d}{\d x}$ is not a Hermitian operator.

\subsection*{Problem II.2.8}
Ehrenfest's theorem relates the time derivative of the expected value
$\expect{Q}$ of an operator $\tilde Q$ to the commutator $[\tilde Q,H]$ of the
operator with the Hamiltonian of the system, as follows:
\[
\frac{\d\expect{Q}}{\d t}=
\frac{1}{i\hbar}\expect{[\tilde Q,H]}+\expect{\frac{\partial Q}{\partial t}}\,.
\]
Derive the following intermediate results:

\paragraph*{a)}
\[
\frac{\d\expect{Q}}{\d t}=\int_{\mbb{R}^3}\frac{\partial\Psi^*}{\partial t}
\tilde Q\Psi\d^3 x+\expect{\frac{\partial Q}{\partial t}}
+\int_{\mbb{R}^3}\Psi^*\tilde Q\frac{\partial \Psi}{\partial t}\d^3 x\,.
\]
\subsubsection*{Solution}
We can just use the product rule as follows
\begin{align*}
\frac{\d\expect{Q}}{\d t}&=\frac{\d}{\d t}\int_{\mbb{R}^3}\Psi^*Q\Psi\,\d^3 x\\
&=\int_{\mbb{R}^3}\frac{\partial\Psi^*}{\partial t}\tilde Q\Psi
+\Psi^*\frac{\partial\tilde Q}{\partial t}\Psi
+\Psi\tilde Q\frac{\partial\Psi}{\partial t}\,\d^3x\\
&=\int_{\mbb{R}^3}\frac{\partial\Psi^*}{\partial t}\tilde Q\Psi\,\d^3x
+\int_{\mbb{R}^3}\Psi^*\frac{\partial\tilde Q}{\partial t}\Psi\,\d^3x
+\int_{\mbb{R}^3}\Psi\tilde Q\frac{\partial\Psi}{\partial t}\,\d^3x\\
&=\int_{\mbb{R}^3}\frac{\partial\Psi^*}{\partial t}\tilde Q\Psi\,\d^3x
+\expect{\frac{\partial\tilde Q}{\partial t}}
+\int_{\mbb{R}^3}\Psi\tilde Q\frac{\partial\Psi}{\partial t}\,\d^3x\,.
\end{align*}
\paragraph*{b)}
\[
\frac{\partial\Psi^*}{\partial t}=-\frac{1}{i\hbar}H\Psi^*\,.
\]
\subsubsection*{Solution}
The next result is somewhat trivial. According to the Schrodinger equation,
\[
H\Psi=i\hbar\frac{\partial\Psi}{\partial t}\,.
\]
From this we get
\[
\frac{1}{i\hbar}H\Psi=\frac{\partial\Psi}{\partial t}\,.
\]

Taking the complex conjugate of both sides and noting that $H$ is a Hermitian
operator,
\[
-\frac{1}{i\hbar}H\Psi^*=\frac{\partial\Psi^*}{\partial t}\,.
\]
\paragraph*{c)} Complete the proof of Ehrenfest's theorem.

Substituting our results from \textbf{b)} into \textbf{a)},
\begin{align*}
\frac{\d\expect{Q}}{\d t}&=\frac{1}{i\hbar}\int_{\mbb{R}^3}
-H\Psi^*\tilde Q\Psi+\Psi^*\tilde QH\Psi\,\d^3x
+\expect{\frac{\partial Q}{\partial t}}\\
&=\frac{1}{i\hbar}\int_{\mbb{R}^3}
-\Psi^*\tilde HQ\Psi+\Psi^*\tilde QH\Psi\,\d^3x
+\expect{\frac{\partial Q}{\partial t}}&H\text{ is Hermitian}\\
&=\frac{1}{i\hbar}\int_{\mbb{R}^3}\Psi^*(\tilde QH-H\tilde Q)\Psi\,\d^3x
+\expect{\frac{\partial Q}{\partial t}}\\
&=\frac{1}{i\hbar}\int_{\mbb{R}^3}\Psi^*[\tilde Q,H]\Psi\,\d^3x
+\expect{\frac{\partial Q}{\partial t}}\\
&=\frac{1}{i\hbar}\expect{[\tilde Q,H]}
+\expect{\frac{\partial Q}{\partial t}}\,.
\end{align*}

\subsection*{Problem II.3.1}
This exercise is designed to carry through the classical method of separation
of variables for the solution of partial differential equations. The problem is
the two-dimensional \emph{particle in a box}. The physical situation is that of
a single particle in a square box $\overline{\Omega}=[0,a]\times[0,b]$ in the
$xy$-plane in a quantum system for which the potential $V=V(x,y)$ is
\[
V(x,y)=
\begin{cases}
0\,,      & \text{for }0\leq x\leq a\,,\,0\leq y\leq b\,,\\
\infty\,, & \text{otherwise}\,.
\end{cases}
\]
The Hamiltonian is thus
\[
H=
\begin{cases}
(-\hbar^2/2m)\Delta
& \text{in the box }\left((x,y)\in\overline{\Omega})\right)\,,\\
+\infty
&\text{outside the box,}
\end{cases}
\]
where $\Delta$ is teh two-dimensional Laplacian,
\[
\Delta=\frac{\partial^2}{\partial x^2}+\frac{\partial^2}{\partial y^y}\,.
\]
Thus, the wave function $\Psi=0$ outside $\overline{\Omega}$. Considering the
time-independent Schrodinger equatio, we wish to find $\psi=\psi(x,y)$
satisfying
\begin{align*}
H\psi&=E\psi & \text{in }&\overline{\Omega}\,,\\
\psi&=0      & \text{on }&\partial\overline{\Omega}\,.
\end{align*}
\paragraph*{a)} Use the standard trick of the method of separation of variables:
assume $\psi$ is a product of a function $X(x)$ and a function
$Y(y):\:\psi(x,y)=X(x)Y(y)$, and derive two ordinary differential equations,
one for $X$ and one for $Y$ under the assumption that $E$ is the sum,
$E=e_X+e_Y\,,\:e_X=\text{constant, }e_Y=\text{constant}$. Show that the
solutions are of the form
\[
\psi_{nn'}(x,y)=\sqrt{\frac{4}{ab}}\sin\frac{n\pi x}{a}
\sin{\frac{n'\pi y}{b}}\,.
\]

\subsubsection*{Solution}
Let us start by plugging $\psi=X(x)Y(y)$ into our governing equation,
\begin{align*}
H(XY)-EXY&=-\frac{\hbar^2}{2m}\Delta X(x)Y(y)\\
&=-\frac{\hbar^2}{2m}\left(\frac{\d^2 X(x)}{\d x^2}Y(y)
+X(x)\frac{\d^2 Y(y)}{\d y^2}\right)-E(X(x)Y(y))=0\,.
\end{align*}
Now divide by $-\frac{\hbar^2}{2m}XY$,
\begin{align*}
\frac{1}{X}\frac{\d^2X}{\d x^2}+\frac{1}{Y}\frac{\d^2Y}{\d y^2}
+\frac{2m}{\hbar^2}E=0\,.
\end{align*}

Therefore,
\[
-\frac{1}{X}\frac{\d^2X}{\d x^2}=\frac{1}{Y}\frac{\d^2Y}{\d y^2}
+\frac{2m}{\hbar^2}E\,.
\]
Since one side of the equation is strictly a function of $x$ and the other side
is strictly a function of $y$, they must both be equal to a constant $e_X$.
Solving the $X$ equation first:
\begin{align*}
\frac{\d^2X}{\d x^2}+e_XX=0\,.
\end{align*}
The zero boundary conditions eliminate the possibility of a polynomial solution
to this ODE and stipulate that $e_X$ must be positive, otherwise the solution
would involve terms of the form $X=A\e^{\sqrt{-e_X}x}$ which obviously can not
be
zero at the boundaries (neglecting the trivial case of $A=0$). Therefore our
solution must be of the form
\[
X(x)=A\cos(\sqrt{e_X}x)+B\sin(\sqrt{e_X}x)\,.
\]
Applying the boundary conditions
\[
X(0)=A=0\,,
\]
and
\[
X(a)=B\sin(\sqrt{e_X}a)=0\,.
\]
Thus either $B=0$ (trivial) or $\sqrt{e_X}a=n\pi$. Taking the non-trivial case,
$e_X=\frac{n^2\pi^2}{a^2}$.

Therefore,
\[
X(x)=B\sin\left(\frac{n\pi x}{a}\right)\,.
\]
Turning to the $Y$ equation,
\[
\frac{1}{Y}\frac{\d^2Y}{\d y^2}+\frac{2m}{\hbar^2}E=\frac{n^2\pi^2}{a^2}\,.
\]
Rearranging terms,
\[
\frac{\d^2Y}{\d y^2}=\left(\frac{n^2\pi^2}{a^2}-\frac{2m}{\hbar^2}E\right)Y
=e_YY
\]
By similar argument from before, $e_Y$ must be positive and $Y(y)$ must be of
the form
\[
Y(y)=C\cos(\sqrt{e_Y}y)+D\sin(\sqrt{e_Y}y)\,.
\]
Applying the boundary conditions,
\[
Y(0)=c=0\,,
\]
and
\[
Y(b)=D\sin(\sqrt{e_Y}y)=0\,.
\]
And by similar argument as before,
\[
e_Y=\frac{n'^2\pi^2}{b^2}\,.
\]
So,
\[
Y(y)=D\sin\left(\frac{n'\pi y}{b}\right)\,.
\]
But $\psi_{nn'}(x,y)=X(x)Y(y)=BD\sin\left(\frac{n\pi x}{a}\right)
\sin\left(\frac{n'\pi y}{b}\right)$ must be normalized, so
\begin{multline*}
\expect{BD\sin\left(\frac{n\pi x}{a}\right)
\sin\left(\frac{n'\pi y}{b}\right),BD\sin\left(\frac{p\pi x}{a}\right)
\sin\left(\frac{p'\pi y}{b}\right)}\\=
\begin{cases}
B^2D^2\frac{ab}{4}\,, & n=p\text{ and }n'=p'\,,\\
0\,, & \text{else}\,.
\end{cases}
\end{multline*}
Therefore, $BD=\sqrt{\frac{4}{ab}}$, and
\[
\psi_{nn'}(x,y)=\sqrt{\frac{4}{ab}}\sin\left(\frac{n\pi x}{a}\right)
\sin\left(\frac{n'\pi y}{b}\right)\,.
\]
\paragraph*{b)} Show that the energy levels are given by
\[
E=\frac{\hbar^2\pi^2}{2m}\left(\frac{n^2}{a^2}+\frac{{n'}^2}{b^2}\right)\,.
\]
\subsubsection*{Solution}
From before,
\[
e_Y=\frac{n'^2\pi^2}{b^2}=\frac{n^2\pi^2}{a^2}-\frac{2m}{\hbar^2}E\,,
\]
and isolating $E$,
\[
E=\frac{\hbar^2}{2m}\left(\frac{n^2\pi^2}{a^2}-\frac{{n'}^2\pi^2}{b^2}\right)
=\frac{\hbar^2\pi^2}{2m}\left(\frac{n^2}{a^2}-\frac{{n'}^2}{b^2}\right)\,.
\]
\paragraph*{c)} For the case of a square box, $(a=b)$, determine the energy
levels
$\displaystyle\mu=\frac{\hbar^2\pi^2}{2ma^2}(n^2+{n'}^2)$ for values of $n$ and
$n'$ up to 4. What is the lowest energy level of the system?
\subsubsection*{Solution}
The calculation is straightforward, just plug in values for $n$ and $n'$ into
the formula for the energy levels. The table below demonstrates the possible
energy levels.
\begin{table*}[h]
\begin{center}
\caption{Possible energy levels for square box}
\begin{tabular}{cc|c|c|c|c|l}
\cline{3-6}
& & \multicolumn{4}{|c|}{$n$} \\ \cline{3-6}
& & 1 & 2 & 3 & 4 \\ \cline{1-6}
\multicolumn{1}{|c|}{\multirow{4}{*}{$n'$}} &
\multicolumn{1}{|c|}{1} & \enoo & \enot & \enoth & \enof &     \\ \cline{2-6}
\multicolumn{1}{|c|}{}                        &
\multicolumn{1}{|c|}{2} & \enot & \entt & \entth & \entf &     \\ \cline{2-6}
\multicolumn{1}{|c|}{}                        &
\multicolumn{1}{|c|}{3} & \enoth & \entth & \enthth & \enthf &  \\ \cline{2-6}
\multicolumn{1}{|c|}{}                        &
\multicolumn{1}{|c|}{4} & \enof & \entf & \enthf & \enff &     \\ \cline{1-6}
\end{tabular}
\end{center}
\end{table*}
Clearly the lowest energy level corresponds to $n=n'=1$,
$\displaystyle
\mu_{11}=\frac{\hbar^2\pi^2}{ma^2}\,.
$

\subsection*{Problem II.3.2}
This exercise concerns a well-known argument for using the nucleus of the
hydrogen atom as the origin of the coordinate system as opposed to the center
of mass. The situation is this: the quantum system of two particles, particle 1
of mass $M$ (corresponding, e.g., to the nucleus) and particle 2 of mass $m$
(e.g. the electron), with origin at a point $O$. The particles are located at
positions $\mbf{r}_1$ and $\mbf{r}_2$, respectively. The center of mass is
located at $\mbf{R}=(\mbf{r}_1M+\mbf{r}_2)/(M+m)$ and the vector connecting 1
to 2 is $\mbf{r}=\mbf{r}_1-\mbf{r}_2$. The goal is to rederive Schrodinger's
equations with a change of variables from $(\mbf{r}_1,\mbf{r}_2)$ to
$(\mbf{R},\mbf{r})$.

\paragraph*{a)} Show that
\[
\mbf{r}_1=\mbf{R}+\frac{m^*}{M}\mbf{r}\quad\text{ and }\quad
\mbf{r}_2=\mbf{R}-\frac{m^*}{M}\mbf{r}\,,
\]
where $m^*=Mm/(M+m)$.
\subsubsection*{Solution}
Substituting $\r_2=\r_1-\r$ into $\R=\frac{\r_1M+\r_2m}{M+m}$,
\[
\R=\frac{\r_1M+(\r_1-\r)m}{M+m}=\frac{(M+m)\r_1-m\r}{M+m}\,.
\]
Then
\begin{align*}
\r_1&=\frac{(M+m)\R+m\r}{M+m}=\R+\frac{m}{M+m}\r=\R+\frac{Mm}{M(M+m)}\r\\
&=\R+\frac{m^*}{m}\r\,.
\end{align*}
From the fact that $\r_2=\r_1-\r$,
\begin{align*}
\r_2&=\R+\frac{m}{M+m}\r-\frac{M+m}{M+m}\r
=\R+\left(\frac{m-M-m}{M+m}\right)\r\\&=\R-\frac{mM}{m(M+m)}\r\\
&=\R-\frac{m^*}{m}\r\,.
\end{align*}
\paragraph*{b)}Using the change of variables, show that
\[
\nabla_{\r_1}=\frac{m^*}{m}\nabla_\R+\nabla_\r \quad\text{ and }\quad
\nabla_{\r_2}=\frac{m^*}{M}\nabla_\R-\nabla_\r\,.
\]
\subsubsection*{Solution}
First note that
\[
\begin{array}{rl}
\frac{\partial \R}{\partial \r_1}=\frac{M}{M+m}=\frac{m^*}{m}\quad
& \frac{\partial \R}{\partial \r_2}=\frac{m}{M+m}=\frac{m^*}{M}\\
\frac{\partial \r}{\partial \r_1}=1 \quad& \frac{\partial \r}{\partial \r_2}=-1
\,.
\end{array}
\]
Then,
\[
\frac{\partial}{\partial\r_1}=
\frac{\partial}{\partial\R}\frac{\partial\R}{\partial\r_1}
+\frac{\partial}{\partial\r}\frac{\partial\r}{\partial\r_1}
=\frac{m^*}{m}\frac{\partial}{\partial\R}+\frac{\partial}{\partial\r}\,.
\]
Similarly
\[
\frac{\partial}{\partial\r_2}=
\frac{\partial}{\partial\R}\frac{\partial\R}{\partial\r_2}
+\frac{\partial}{\partial\r}\frac{\partial\r}{\partial\r_2}
=\frac{m^*}{M}\frac{\partial}{\partial\R}-\frac{\partial}{\partial\r}\,.
\]
But $\nabla_{\r_1}=\mbf{e}_i\frac{\partial}{\partial\r_{1i}}$, and similarly
for $\r_2$, $\r$, and $\R$. Therefore,
\[
\nabla_{\r_1}=\frac{m^*}{m}\nabla_\R+\nabla_\r
\]
and
\[
\nabla_{\r_2}=\frac{m^*}{M}\nabla_\R-\nabla_\r\,.
\]
\paragraph*{c)} Show that, in these new variables, the time-independent
Schrodinger equation is:
\[
\left(-\frac{\hbar^2}{2(M+m)}\Delta_\R-\frac{\hbar^2}{2m^*}\Delta_\r
+V(\r)\right)\psi=E\psi\,.
\]
\subsubsection*{Solution}
We first need to derive the Laplacian operator in our changed coordinates. So,
\begin{align*}
\Delta_{\r_1}&=\frac{\partial}{\partial\r_1}\cdot\frac{\partial}{\partial\r_1}
=\left(\frac{m^*}{m}\right)^2\Delta_\R+2\left(\frac{m^*}{m}\right)\Delta_{\r\R}
+\Delta_\r
\end{align*}
and
\begin{align*}
\Delta_{\r_2}&=\frac{\partial}{\partial\r_2}\cdot\frac{\partial}{\partial\r_2}
=\left(\frac{m^*}{M}\right)^2\Delta_\R-2\left(\frac{m^*}{M}\right)\Delta_{\r\R}
+\Delta_\r
\end{align*}
We can write the time independent Schrodinger's equation as
\[
\left(-\frac{\hbar^2}{2M}\Delta_{\r_1}-\frac{\hbar^2}{2m}\Delta_{\r_2}
+V(\r)\right)\psi=E\psi\,.
\]
Substituting our expressions for $\Delta_{\r_1}$ and $\Delta_{\r_2}$,
\begin{align*}
&\left(-\frac{\hbar^2}{2M}\left(
\left(\frac{m^*}{m}\right)^2\Delta_\R
+2\cancel{\left(\frac{m^*}{m}\right)\Delta_{\r\R}}
+\Delta_\r\right)\right.\\
&\left.-\frac{\hbar^2}{2m}\left(
\left(\frac{m^*}{M}\right)^2\Delta_\R
-2\cancel{\left(\frac{m^*}{M}\right)\Delta_{\r\R}}
+\Delta_\r\right)
+V(\r)\right)\psi\\
&=\left(-\frac{\hbar^2{m^*}^2}{2Mm}\left(\frac{1}{m}+\frac{1}{M}\right)\Delta_\R
-\frac{\hbar^2}{2}\left(\frac{1}{m}+\frac{1}{M}\right)\Delta_\r
+V(\r)\right)\psi\\
&=\left(-\frac{\hbar^2{m^*}^{\cancel{2}}}{2Mm}
\cancel{\left(\frac{M+m}{Mm}\right)}
\Delta_\R-\frac{\hbar^2}{2}\left(\frac{M+m}{mM}\right)\Delta_\r
+V(\r)\right)\psi\\
&=\left(-\frac{\hbar^2}{2(M+m)}\Delta_\R-\frac{\hbar^2}{2m^*}\Delta_\r
+V(\r)\right)\psi=E\psi\,.
\end{align*}
\paragraph*{d)} Now if $M\gg m$ so that $M+m\approx M$ and $1/m\gg 1/(M+m)$,
write down the resulting approximate Schrodinger equation involving only $\r$.
\subsubsection*{Solution}
\begin{align*}
&\left(-\frac{\hbar^2}{2(M+m)}\Delta_\R-\frac{\hbar^2(M+m)}{2Mm}\Delta_\r
+V(\r)\right)\psi\\
&\approx \left(-\frac{\hbar^2}{2(M+m)}\Delta_\R
-\frac{\hbar^2\cancel{M}}{2\cancel{M}m}\Delta_\r
+V(\r)\right)\psi\\
&=\left(-\underbrace{\frac{\hbar^2}{2(M+m)}}_
{\displaystyle\ll \frac{\hbar^2}{2m}}
\Delta_\R
-\frac{\hbar^2}{2m}\Delta_\r
+V(\r)\right)\psi\\
&\approx\left(-\frac{\hbar^2}{2m}\Delta_\r
+V(\r)\right)\psi=E\psi\,.
\end{align*}

\paragraph*{e)} Suppose now that the wave function is
separable:$\psi(\r,\R)=\psi(\r)\chi(\R)$. Show that $\chi$ satisfies the
one-particle Schrodinger equation with mass $M+m$, with potential $V_\R=0$, and
energy $E_\R$, while $\psi$ satisfies the one-particle Schrodinger equation
with mass $m^*$ and potential $V(\r)$, and energy $E_\r$, with the total energy
$E=E_\R+E_\r$.
\subsubsection*{Solution}
Substituting our separated wave function into the Schrodinger equation,
\[
-\frac{\hbar^2}{2(M+m)}\psi(\r)\Delta_\R\chi(\R)
-\frac{\hbar^2}{2m^*}\chi(\R)\Delta_\r\psi(\r)+V(\r)\psi(\r)\chi(\R)
=E\psi(\r)\chi(\R)\,.
\]
Dividing by $\psi(\r)\chi(\R)$ and separating variables,
\[
\frac{\hbar^2}{2(M+m)}\frac{\Delta_\R\chi(\R)}{\chi(\R)}+E_\R
=-\frac{\hbar^2}{2m^*}\frac{\Delta_\r\psi(\r)}{\psi(\r)}+V(\r)-E_\r\,.
\]
Since the left hand side is strictly a function of $\R$ and the right hand side
is strictly a function of $\r$, both sides must be equal to a constant. Also,
assuming we have split split up our energy appropriately, the constant must be
zero. Expanding the left hand side,
\[
-\frac{\hbar^2}{2(M+m)}\Delta_\R\chi(\R)=E_\R\chi(\R)\,,
\]
which is precisely the one-dimensional Schrodinger equation with mass $M+m$,
potential $V_\R=0$, and energy $E_\R$. Similarly with the right hand
side,
\[
-\frac{\hbar^2}{2m^*}\Delta_\r\psi(\r)+V(\r)\psi(\r)=E_\r\psi(\r)\,,
\]
which is the one-dimensional Schrodinger equation with mass $m^*$, potential
$V(\r)$, and energy $E_\r$.
\end{document}