\documentclass[letterpaper,10pt]{article}
\usepackage{amsfonts,amsmath,amssymb,amsthm}
\usepackage{cancel}
\usepackage{graphicx}

\setlength{\parindent}{0in}
\setlength{\parskip}{.4ex}

\DeclareMathOperator*{\tgrad}{grad}
\DeclareMathOperator*{\tGrad}{Grad}
\DeclareMathOperator*{\tdiv}{div}
\DeclareMathOperator*{\tDiv}{Div}
\DeclareMathOperator*{\tcurl}{curl}
\DeclareMathOperator*{\Cof}{Cof}
\DeclareMathOperator*{\tr}{tr}

\providecommand{\abs}[1]{\lvert#1\rvert}
\providecommand{\norm}[1]{\lVert#1\rVert}

\def\mathbi#1{\textbf{\em #1}}
\def\expect#1{\langle #1\rangle}
\def\d{\mathrm{d}}
\def\e{\mathrm{e}}


%opening
\title{CAM 389C Exercise Set II.2}
\author{Truman Ellis}

\begin{document}

\maketitle

\subsection*{Problem i}
Reproduce the proof of
\[
\expect{p}=m\frac{\d\expect{x}}{\d t}=\int_{-\infty}^\infty\Psi^*p\Psi\,\d x\,.
\]

We can evaluate
\begin{align*}
\frac{\d\expect{x}}{\d t}&=\frac{\d}{\d t}\int_\mathbb{R}x(\Psi^*\Psi)\,\d x\\
&=\int_\mathbb{R}x\frac{\partial}{\partial t}(\Psi^*\Psi)\,\d x\\
&=\frac{i\hbar}{2m}\int_\mathbb{R}x\frac{\partial}{\partial
x}\left(\Psi^*\frac{\partial\Psi}{\partial x}-\frac{\partial\Psi^*}{\partial
x}\Psi\right)\,\d x\\
&=\left.\frac{i\hbar}{2m}x\left(\cancel{\Psi^*}\frac{\partial\Psi}{\partial x}
-\frac{\partial\Psi^*}{\partial x}\cancel{\Psi}\right)\right|_{-\infty}^\infty
-\frac{i\hbar}{2m}\int_\mathbb{R}\left(\Psi^*\frac{\partial\Psi}{\partial
x}-\frac{\partial\Psi^*}{\partial
x}\Psi\right)\,\d x\\
&=-\frac{i\hbar}{2m}\int_\mathbb{R}\Psi^*\frac{\partial\Psi}{\partial x}\,\d x
+\frac{i\hbar}{2m}
\underbrace{\int_\mathbb{R}\frac{\partial\Psi^*}{\partial x}\Psi\,\d x}
_{\left.\cancel{\Psi\Psi^*}\right|_{-\infty}^\infty
-\int_\mathbb{R}\Psi^*\frac{\partial\Psi}{\partial x}\,\d x}\\
&=-\frac{i\hbar}{m}\int_\mathbb{R}\Psi^*\frac{\partial\Psi}{\partial x}\,\d x\\
&=\frac{1}{m}\int_\mathbb{R}\Psi^*
\left(\frac{\hbar}{i}\frac{\partial}{\partial x}\right)\Psi\,\d x\\
&=\frac{1}{m}\int_\mathbb{R}\Psi^*p\Psi\,\d x\\
&=\frac{\expect{p}}{m}\,.
\end{align*}
Therefore
\[
\expect{p}=m\frac{\d\expect{x}}{\d t}=\int_{-\infty}^\infty\Psi^*p\Psi\,\d x\,.
\]

\subsection*{Problem ii}
Reproduce the proof of
\[
\frac{\d\expect{p}}{\d t}=\expect{F}.
\]

Taking the time derivative of $\expect{p}$,
\begin{align*}
\frac{\d\expect{p}}{\d t}&=\frac{\d}{\d t}\int_\mathbb{R}(\Psi^*p\Psi)\,\d x\\
&=\int_\mathbb{R}\underbrace{\frac{\partial\Psi^*}{\partial t}}_
{\displaystyle
-\frac{i\hbar}{2m}\frac{\partial^2\Psi^*}{\partial x^2}-\frac{i}{\hbar}V\Psi^*}
p\Psi\,\d x
+\int_\mathbb{R}\Psi^*p\underbrace{\frac{\partial\Psi}{\partial t}}_
{\displaystyle
\frac{i\hbar}{2m}\frac{\partial^2\Psi}{\partial x^2}+\frac{i}{\hbar}V\Psi}
\,\d x\\
&=-\frac{1}{i\hbar}\int_\mathbb{R}
\left(-\frac{\hbar^2}{2m}\frac{\partial^2\Psi^*}{\partial
x^2}-V\Psi^*\right)p\Psi+\Psi^*p
\left(\frac{\hbar^2}{2m}\frac{\partial^2\Psi}{\partial x^2}+V\Psi
\right)\,\d x\\
&=-\frac{1}{i\hbar}\int_\mathbb{R}
(H-2V)\Psi^*p\Psi-\Psi^*p(H-2V)\Psi\,\d x\\
&=-\frac{1}{i\hbar}\int_\mathbb{R}
\Psi^*(H-\cancel{2V})p\Psi-\Psi^*p(H-\cancel{2V})\Psi\,\d x
& H \text{ is Hermitian} \\
&=-\frac{1}{i\hbar}\int_\mathbb{R}
\Psi^*Hp\Psi-\Psi^*pH\Psi\,\d x\\
&=-\frac{1}{i\hbar}\int_\mathbb{R} \Psi^*\underbrace{(\underbrace{Hp}_
{-\frac{\hbar^2}{2m}\frac{\partial^3}{\partial x^3}
+V\frac{\hbar}{i}\frac{\partial}{\partial x}}
-\underbrace{pH}_
{-\frac{\hbar^2}{2m}\frac{\partial^3}{\partial x^3}
+\frac{\hbar}{i}\frac{\partial}{\partial x}V}
)}_
{\displaystyle V\frac{\hbar}{i}\frac{\partial}{\partial x}
-\frac{\hbar}{i}\frac{\partial}{\partial x}V}
\Psi\,\d x\\
&=\int_\mathbb{R}\Psi^*(\cancel{
V\frac{\partial\Psi}{\partial x}}
-\underbrace{\frac{\partial}{\partial x}(V\Psi)}_
{\displaystyle \cancel{V\frac{\partial\Psi}{\partial x}}
+\frac{\partial V}{\partial x}\Psi}
)\,\d x \\
&=\int_\mathbb{R}\Psi^*\left(-\frac{\partial V}{\partial x}\right)\Psi\,\d x\\
&=\left\langle -\frac{\partial V}{\partial x}\right\rangle\,.
\end{align*}



\subsection*{Problem 1}
The proof of
\[
\sigma_Q^2\sigma_M^2\geq\left(\frac{1}{2i}\langle\left[Q,M\right]
\rangle\right)^2
\]
follows from several algebraic steps and the Cauchy-Schwarz inequality. Let
\[
u=(\tilde Q-\expect{Q})\Psi\quad\text{and}\quad v=(\tilde M-\expect{M})\Psi\,.
\]
\textbf{a)}

\begin{align*}
\sigma_Q^2&=\expect{Q^2}-\expect{Q}^2\\
&=\underbrace{\expect{\Psi,\tilde Q(\tilde Q\Psi)}}_{\text{Expected value of
}Q^2}-\underbrace{\expect{\Psi,\expect{Q}^2\Psi}}_
{\expect{Q}^2\underbrace{\int_{\mathbb{R}}\Psi^*\Psi\,\d x}_1}\\
&=\underbrace{\expect{\tilde Q\Psi,\tilde Q\Psi}}_{Q\text{ is Hermitian}}
-\expect{\expect{Q}\Psi,\expect{Q}\Psi}\\
&=\int_\mathbb{R}(\tilde Q\Psi)^*\tilde Q\Psi
-\expect{Q}\Psi^*\expect{Q}\Psi\,\d x\\
&=\int_\mathbb{R}(\tilde Q\Psi)^*\tilde Q\Psi
-2\expect{Q}\underbrace{\Psi^*\expect{Q}\Psi}_{\tilde Q}
+\expect{Q}\Psi^*\expect{Q}\Psi\,\d x\\
&=\expect{(\tilde Q-\expect{Q})\Psi,(\tilde Q-\expect{Q})\Psi}\\
&=\expect{u,u}\\
&=\norm{u}^2\,,
\end{align*}
and similarly $\sigma_M^2=\norm{v}^2$. Thus, by Cauchy-Schwarz,
\[
\sigma_Q^2\sigma_M^2=\norm{u}^2\norm{v}^2\geq\abs{\expect{u,v}}^2\,.
\]
\textbf{b)} Show that
\[
\expect{u,v}=\expect{QM}-\expect{Q}\expect{M}\,.
\]
Note that
\begin{align*}
\expect{u,v}&=\expect{(\tilde Q-\expect{Q})\Psi,(\tilde M-\expect{M})\Psi}\\
&=\underbrace{\expect{\tilde Q\Psi,\tilde M\Psi}}_{\text{Hermitian}}
-\expect{M}\underbrace{\expect{\tilde Q\Psi,\Psi}}_{\expect{Q}}
-\expect{Q}\underbrace{\expect{\Psi,\tilde M\Psi}}_{\expect{M}}
+\expect{Q}\expect{M}\underbrace{\expect{\Psi,\Psi}}_1\\
&=\expect{\Psi,\tilde Q\tilde M\Psi}
-\expect{M}\expect{Q}-\cancel{\expect{Q}\expect{M}}
+\cancel{\expect{Q}\expect{M}}\\
&=\expect{QM}-\expect{Q}\expect{M}\,.
\end{align*}
\textbf{c)} The number $\expect{u,v}$ is complex. From the fact that any
complex number $z$ satisfies
\[
\abs{z}^2\geq\left(\frac{1}{2i}(z-z^*)\right)^2\,,
\]
where $z^*$ is the complex conjugate of $z$, show that
\[
\sigma_Q^2\sigma_M^2\geq\left(\frac{1}{2i}(\expect{u,v}-\expect{v,u})\right)^2\,
.
\]
From the properties of complex inner products, we know that
\[
\expect{u,v}=\overline{\expect{v,u}}\,.
\]
Therefore,
\[
\abs{\expect{u,v}}^2\geq\left(\frac{1}{2i}(\expect{u,v}-\expect{v,u})\right)^2\,
.
\]
And from Cauchy-Schwarz we know that
\[
\sigma_Q^2\sigma_M^2\geq
\abs{\expect{u,v}}^2\geq
\left(\frac{1}{2i}(\expect{u,v}-\expect{v,u})\right)^2\,.
\]
Therefore,
\[
\sigma_Q^2\sigma_M^2\geq
\left(\frac{1}{2i}(\expect{u,v}-\expect{v,u})\right)^2\,.
\]
\textbf{d)} Now we can show that
\begin{align*}
\sigma_Q^2\sigma_M^2&\geq
\left(\frac{1}{2i}(\expect{u,v}-\expect{v,u})\right)^2\\
&=\left(\frac{1}{2i}\left((\expect{QM}-\expect{Q}\expect{M})
-(\expect{MQ}-\expect{M}\expect{Q})\right)\right)^2\\
&=\left(\frac{1}{2i}(\expect{QM}-\cancel{\expect{Q}\expect{M}}
-\expect{MQ}+\cancel{\expect{M}\expect{Q}})\right)^2\\
&=\left(\frac{1}{2i}\expect{[Q,M]}\right)^2\,,
\end{align*}
where
\[
[Q,M]=\tilde Q\tilde M-\tilde M\tilde Q\,.
\]
This finishes the proof.

\subsection*{Problem 2}
A classical textbook example of the time-independent Schrodinger equation for a
single particle moving along the $x$-axis is the problem of the infinite square
well for which the potential $V$ is of the form
\[
V(x)=
\begin{cases}
0\,,      & \text{if }0\leq x\leq a\,,\\
\infty\,, & \text{otherwise}\,.
\end{cases}
\]
The particle is confined to this ``well'' so $\psi(x)=0$ if $x < 0\text{ and
}x>a$ while $V(x)\equiv 0$ inside the well.

\textbf{a)} Show that Schrodinger's equation reduces to
\[
\frac{\d^2\psi}{\d x^2}+k^2\psi=0\,,\quad\psi(0)=\psi(a)=0\,,
\]
with $k^2=2mE/\hbar^2$.

The time-independent Schrodinger's equation is
\[
-\frac{\hbar^2}{2m}\frac{\d^2\psi}{\d x^2}+V\psi=E\psi\,.
\]
With infinite potential outside the box, the particle is regulated to $0\leq x
\leq a$. So, within these bounds, with zero potential the time-independent
equation becomes
\[
-\frac{\hbar^2}{2m}\frac{\d^2\psi}{\d x^2}=E\psi\,.
\]
Moving everything to one side and dividing by $\hbar^2/2m$,
\[
\frac{\d^2\psi}{\d x^2} + \frac{2mE}{\hbar^2}\psi=0\,.
\]
Let $k^2=2mE/\hbar^2$, then
\[
\frac{\d^2\psi}{\d x^2} + k^2\psi=0\,.
\]
\textbf{b)} Show that the possible values of the energy are
\[
E_n=\frac{n^2\pi^2\hbar^2}{2ma^2}\,,\quad n=1,2,\dots
\]
We know that solutions to this type of ordinary differential equation take the
form
\[
\psi(x)=A\cos(kx)+B\sin(kx)\,.
\]

Our boundary conditions dictate that $\psi=0$ outside the box, and $\psi$ must
be continuous, therefore
\[
\psi(0)=A=0\,,
\]
and
\[
\psi(a)=B\sin(ka)=0\,.
\]
The solution is trivial if $B=0$, then we have a non-trivial solution when
\[
ka=\frac{\sqrt{2mE}}{\hbar}a=n\pi\,,\quad n=1,2,\dots\,.
\]
Therefore, we have a series of solutions for increasing $n$,
\[
E_n=\frac{n^2\pi^2\hbar^2}{2ma^2}\,,\quad n=1,2,\dots\,.
\]
\textbf{c)} Show that the wave function is a superposition of solutions,
\[
\psi_n(x)=\sqrt{\frac{2}{a}}\sin\left(\frac{n\pi x}{a}\right)\,,\quad
0\leq x\leq a\,,
\]
which are orthonormal in $L^2(0,a)$.

First of all, substituting $E$ into $k$, please note that
\[
k=\frac{n\pi}{a}\,.
\]
We wish that $\psi_n(x)$ are orthonormal, then
\[
\expect{\psi_n(x),\psi_m(x)}=\delta_{mn}\,.
\]
But,
\[
\expect{B\sin(\frac{n\pi}{a}x),B\sin(\frac{m\pi}{a}x)}=
\begin{cases}
B^2\frac{a}{2}\,,      & n=m\,,\\
0\,, & n\neq m\,.
\end{cases}
\]
In order to normalize this, we need $B=\sqrt{2/a}$.
Therefore,
\[
\psi_n(x)=\sqrt{\frac{2}{a}}\sin\left(\frac{n\pi x}{a}\right)\,.
\]
If any one $\psi_n(x)$ is a solution, then the superposition of solutions is
also a solution. Thus the wave function can be written as a sum of
$\psi_n(x)\,,\quad n=1,2,\dots$.

\textbf{d)} Given that the functions $\psi_n$ form a complete orthonormal basis
for $L^2(0,a)$, develop a Fourier series representation of the function
$f(x)=x$ as an infinite sum of these functions.

If any one $\psi_n(x)$ satisfies the time-independent Schrodinger equation, then
the infinite sum of these will also satisfy it. In general we can write
\[
f(x)=\sum_{n=1}^\infty b_n\sqrt{\frac{2}{a}}\sin\left(\frac{n\pi x}{a}\right)
\,,\quad 0\leq x\leq a\,,
\]
where
\[
b_n=\expect{\psi_n,x}=\sqrt{\frac{2}{a}}\int_0^a x\sin\left(\frac{n\pi
x}{a}\right)\,\d x\,.
\]
which is in fact a Fourier sine series representation for $f(x)=x$.

\textbf{e)} Let
\[
Q(x,p)\Psi=\frac{\d^2\Psi(x)}{\d x^2}\,,\quad 0\leq x\leq a\,.
\]
Is $Q$ observable?

We can substitute for $Q$ in the reduced Schrodinger equation,
\[
Q\psi=-\frac{n^2\pi^2}{a^2}\psi\,.
\]
Therefore $Q$ has a discrete spectrum of eigenvalues
$\{-\frac{n^2\pi^2}{a^2}\}$. It is only left to prove that $Q$ is Hermitian.
Integrating by parts twice,
\begin{align*}
\expect{\psi,Q\psi}&=\int_\mathbb{R}\psi^*\frac{\d^2\psi}{\d x^2}\,\d x\\
&=\left.\cancel{\psi^*}\frac{\d\psi}{\d x}\right|_0^a
-\int_\mathbb{R}\frac{\d\psi}{\d x}\frac{\d\psi^*}{\d x}\,\d x\\
&=-\left.\frac{\d\psi^*}{\d x}\cancel{\psi}\right|_0^a
+\int_\mathbb{R}\psi\frac{\d^2\psi^*}{\d x^2}\,\d x\\
&=\expect{Q\psi,\psi}\,.
\end{align*}
Therefore $Q$ is Hermitian and has a discrete spectrum of eigenvalues, so $Q$
must be observable.

\subsection*{Problem 3}
Consider a quantum system consisting of a single particle in a straight line
with position $x$ and momentum $p$. Define the following operators:
\begin{align*}
A\phi&=x\psi & (A&=x)\,,\\
B\phi&=\partial\phi/\partial x & (B&=\partial/\partial x)\,.
\end{align*}
Do $A$ and $B$ commute?

The commutator of $A$ and $B$ applied to $\phi$ is
\begin{align*}
[A,B]\phi&=x\frac{\partial}{\partial x}\phi-\frac{\partial}{\partial x}x\phi\\
&=x\frac{\partial}{\partial x}\phi-x\frac{\partial}{\partial x}-\phi\\
&=-\phi\neq 0\,.
\end{align*}
Therefore $A$ and $B$ do not commute.

\subsection*{Problem 4}
Let
\[
Q=-\frac{\hbar^2}{2m}\frac{\partial}{\partial x^2}\quad\text{and}\quad
M=-i\hbar\frac{\partial}{\partial x}\,.
\]
\textbf{a)} Show that these operators commute.

The commutator of $Q$ and $M$ applied to $\phi$ is
\begin{align*}
[Q,M]\phi&=-\frac{\hbar^2}{2m}\frac{\partial^2}{\partial x^2}
\left(-i\hbar\frac{\partial}{\partial x}\right)\phi
-\left(-i\hbar\frac{\partial}{\partial x}\right)
\left(-\frac{\hbar^2}{2m}\frac{\partial^2}{\partial x^2}\right)\phi\\
&=\frac{i\hbar^3}{2m}\frac{\partial^3}{\partial x^3}\phi
-\frac{i\hbar^3}{2m}\frac{\partial^3}{\partial x^3}\phi=0
\end{align*}
\textbf{b)} Show that $\e^{ikx}$ is a simultaneous eigenfunction of these
operators and, indeed, $p^2/2m=E$.

To show that $e^{ikx}$ is a simultaneous eigenfunction of $Q$ and $M$, we need
\[
Q\e^{ikx}=\lambda_Q\e^{ikx}\,,
\]
and
\[
M\e^{ikx}=\lambda_M\e^{ikx}\,.
\]
Then,
\[
Q\e^{ikx}=-\frac{\hbar^2}{2m}\frac{\partial}{\partial
x^2}\e^{ikx}=\underbrace{\frac{\hbar^2k}{2m}}_
{\lambda_Q}
\e^{ikx}\,.
\]
Similarly for $M$,
\[
M\e^{ikx}=-i\hbar\frac{\partial}{\partial
x}\e^{ikx}=\underbrace{\frac{\hbar^2k}{2m}}_
{\lambda_M}
\e^{ikx}\,.
\]
Therefore $\e^{ikx}$ is a simultaneous eigenfunction for $Q$ and $M$.

If we assume the absence of a potential, the Hamiltonian becomes
$H=-\frac{\hbar^2}{2m}\frac{\d^2}{\d x^2}$, and the Schrodinger equation reads
\[
H\Psi=E\Psi\,.
\]
Then
\[
\frac{p^2}{2m}=-\frac{\hbar^2}{2m}\frac{\d^2}{\d x^2}=H=E\,.
\]

\end{document}

% First we notice that
% \[
% \Psi(x,t)=\e^{iEt/\hbar}\psi(x)\,,
% \]
% and
% \[
% \Psi^*\Psi=\bar{\e^{iEt/\hbar}}\psi(x)\e^{iEt/\hbar}\psi(x)
% =\e^{-iEt/\hbar+-iEt/\hbar}\psi(x)^2=\psi(x)^2\,.
% \]
% Also note that
% \[
% Q\Psi=\frac{\d^2}{\d x^2}\Psi=\e^{iEt/\hbar}\frac{\d^2}{\d x^2}\psi\,,
% \]
% where
% \[
% \frac{\d^2}{\d x^2}\psi=\sum_{n=1}^\infty -b_n\sqrt{\frac{2}{a}}
% \left(\frac{n^2\pi^2}{a^2}\right)\sin\left(\frac{n\pi x}{a}\right)
% =-\sum_{n=1}^\infty\left(\frac{n^2\pi^2}{a^2}\right)\psi_n\,.
% \]
% So
% \begin{align*}
% \expect{Q}&=\int_\mathbb{R}\Psi^*Q\Psi\,\d x\\
% &=\int_\mathbb{R}\psi\frac{\d^2}{\d x^2}\psi\,\d x \\
% &=\int_\mathbb{R}\left(\sum_{n=1}^\infty \psi_n\right)
% \left(\sum_{n=1}^\infty-\left(\frac{n^2\pi^2}{a^2}\right)\psi_n\right)\,\d x\\
% &=-\sum_{n=1}^\infty\left(\frac{n^2\pi^2}{a^2}\right)
% & \text{by orthonormality of }\psi_n\,.
% \end{align*}
% In a similar manner, we can calculate
% \begin{align*}
% \expect{Q^2}&=\int_\mathbb{R}\Psi^*Q^2\Psi\,\d x\\
% &=\int_\mathbb{R}\psi\frac{\d^4}{\d x^4}\psi\,\d x \\
% &=\int_\mathbb{R}\left(\sum_{n=1}^\infty \psi_n\right)
% \left(\sum_{n=1}^\infty\left(\frac{n^4\pi^4}{a^4}\right)\psi_n\right)\,\d x\\
% &=\sum_{n=1}^\infty\left(\frac{n^4\pi^4}{a^4}\right)
% & \text{by orthonormality of }\psi_n\,.
% \end{align*}