\documentclass[letterpaper,10pt]{article}
\usepackage{amsfonts,amsmath,amssymb,amsthm}
\usepackage{cancel}

\setlength{\parindent}{0in}
\setlength{\parskip}{.4ex}

\DeclareMathOperator*{\tgrad}{grad}
\DeclareMathOperator*{\tGrad}{Grad}
\DeclareMathOperator*{\tdiv}{div}
\DeclareMathOperator*{\tDiv}{Div}
\DeclareMathOperator*{\tcurl}{curl}
\DeclareMathOperator*{\Cof}{Cof}
\DeclareMathOperator*{\tr}{tr}

\providecommand{\abs}[1]{\lvert#1\rvert}
\providecommand{\norm}[1]{\lVert#1\rVert}

\def\mathbi#1{\textbf{\em #1}}
\def\d{\mathrm{d}}


%opening
\title{CAM 389C Exercise Set I.6}
\author{Truman Ellis}

\begin{document}

\maketitle

\subsection*{Problem 1}
Consider the small deformations and heating of a thermo-elastic solid
constructed of a material characterized by the following constitutive equations:
\begin{align*}
&\text{Free energy: }
\rho_0\psi_0=\frac{1}{2}\lambda(\tr\mathbi{e})^2+\mu\mathbi{e}:\mathbi{e}
+c(\tr\mathbi{e})\theta+\frac{c_0}{2}\theta^2\,,\\
&\text{Heat Flux: }\,\,\,\,
\mathbf{q}_0=k\nabla\theta\,,
\end{align*}
where
\begin{align*}
\mathbi{e}=&\frac{1}{2}(\nabla\mathbi{u}+\nabla\mathbi{u}^T)=\text{the
``infinitesimal'' strain tensor } (\approx\mathbf{E})\,,\\
\mathbi{u}=&\text{ the dispacement field}\,,\\
\theta=&\text{ the temperature field}\,,\\
\lambda,\mu,c,c_0,k=&\text{ material constants.}
\end{align*}
A body $\mathbb{B}$ is constructed of such a material and is subjected to body
forces $\mathbf{f}_0$ and to surface contact forces $\mathbi{g}$ on a portion
$\Gamma_g$ of its boundary $\Gamma_g\subset\partial\Omega_0$. On the remainder
of its boundary, $\Gamma_u=\partial\Omega_0\setminus\Gamma_g$, the
displacements $\mathbi{u}$ are prescribed as zero ($\mathbi{u}=\mathbf{0}$ on
$\Gamma_u$). The mass density of the body is $\rho_0$, and, when in its
reference configuration at time $t=0$,
$\mathbi{u}(\mathbi{x},0)=\mathbi{u}_0(\mathbi{x})$,
$\partial\mathbi{u}(\mathbi{x},0)/\partial t=\mathbi{v}_0(\mathbi{x})$, 
$\mathbi{x}\in\Omega_0$, where $\mathbi{u}_0$ and $\mathbi{v}_0$ are given
functions. A portion $\Gamma_q$ of the boundary is heated, resulting in a
prescribed heat flux $h=\mathbf{q}\cdot\mathbf{n}$, and the complementary
boundary, $\Gamma_\theta=\partial\Omega_0\setminus\Gamma_q$ is subjected to a
prescribed temperature
$\theta(\mathbi{x},t)=\tau(\mathbi{x},t),\,\mathbi{x}\in\Gamma_\theta$.

Develop a mathematical model of this physical phenomena (a set of partial
differential equations, boundary and initial conditions): the dynamic,
thermomechanical behavior of a thermoelastic solid.

\subsubsection*{Solution}
From the given formula for the Helmholtz free energy, we can see that
\[
\psi_0=\Psi(\mathbi{e},\theta)\,.
\]
According to the Coleman-Noll Method, we can rewrite our constraint on the
second law of thermodynamics as
\[
-\rho_0\dot{\psi_0}-\rho_0\eta_0\dot\theta+\mathbf{S}:\dot{\mathbf{E}}-\frac{1}{
\theta}\mathbf{q}_0\cdot\nabla\theta\geq0\,.
\]
Since we are dealing with small deformations, we can assume from here on that
$\mathbf{E}\approx\mathbi{e}$, and the second law constraint becomes
\[
-\rho_0\dot{\psi_0}-\rho_0\eta_0\dot\theta+\mathbf{S}:\dot{\mathbi{e}}-\frac{1}{
\theta}\mathbf{q}_0\cdot\nabla\theta\geq0\,.
\]
When we substitute
\[
\dot\psi_0=\frac{\partial\Psi}{\partial\mathbi{e}}:\dot{\mathbi{e}}+
\frac{\partial\Psi}{\partial\theta}:\dot{\theta}
\]
into the constraint, we get
\[
\left(\mathbf{S}-\rho_0\frac{\partial\Psi}{\partial\mathbi{e}}\right):\dot{
\mathbi{e}}
-\rho_0\left(\frac{\partial\Psi}{\partial\theta}+\eta_0\right)\dot\theta
-\frac{1}{\theta}\mathbf{q}_0\cdot\nabla\theta\geq0\,.
\]
It is sufficient to satisfy this condition that the coefficients of the rates
be zero. Thus
\[
\mathbf{S}=\rho_0\frac{\partial\Psi}{\partial\mathbi{e}}\,,
\text{ an }\quad\eta_0=-\frac{ \partial\Psi}{\partial\theta}\,.
\]

Now,
\begin{align*}
\rho_0\frac{\partial\Psi}{\partial\mathbi{e}}&=\frac{\partial}{\partial\mathbi{e
} } \left(
\frac{1}{2}\lambda(\tr\mathbi{e})^2+\mu\mathbi{e}:\mathbi{e}
+c(\tr\mathbi{e})\theta+\frac{c_0}{2}\theta^2 \right)\\
&=\frac{1}{2}\lambda\frac{\d}{\d\mathbi{e}}(\tr\mathbi{e})^2
+\mu\frac{\d}{\d\mathbi{e}}\tr(\mathbi{e}^T\mathbi{e})
+c\theta\frac{\d}{\d\mathbi{e}}(\tr\mathbi{e})
\end{align*}
% Since trace is a linear operator, $\frac{\d}{\d x}\tr f=\tr\frac{\d}{\d x} f$,
%and
Since $\mathbi{e}$ is square and symmetric,
$\mathbi{e}^T\mathbi{e}=\mathbi{e}^2$.
Therefore,
\begin{align*}
\rho_0\frac{\partial\Psi}{\partial\mathbi{e}}
&=\lambda(\tr\mathbi{e})\frac{\d}{\d\mathbi{e}}\tr(\mathbi{e})
+\mu\frac{\d}{\d\mathbi{e}}\tr(\mathbi{e}^2)
+c\theta\frac{\d}{\d\mathbi{e}}\tr(\mathbi{e})\\
&=\left(\lambda(\tr\mathbi{e})
+2\mu\tr(\mathbi{e})
+c\theta\right)\mathbf{I}\,.
\end{align*}
Next,
\begin{align*}
\rho_0\frac{\partial\Psi}{\partial\theta}&=\frac{\partial}{\partial\theta}\left(
\frac{1}{2}\lambda(\tr\mathbi{e})^2+\mu\mathbi{e}:\mathbi{e}
+c(\tr\mathbi{e})\theta+\frac{c_0}{2}\theta^2 \right)\\
&=c\tr\mathbi{e}+c_0\theta\,.
\end{align*}
Therefore, we can automatically satisfy the second law if we use
\[
\mathbf{S}=\left(\lambda(\tr\mathbi{e})
+2\mu\tr(\mathbi{e})
+c\theta\right)\mathbf{I}\,,
\]
and
\[
\eta_0=-\frac{1}{\rho_0}\left(c\tr\mathbi{e}+c_0\theta\right)\,.
\]
Please note that $\mathbf{F}=\mathbf{I}+\nabla\mathbi{u}$.

Therfore, our system of partial differential equations describing this
thermoelastic solid are
\begin{description}
\item[Conservation of Mass]
\begin{align*}
\rho_0&=\rho\det\mathbf{F}\\
&=\rho\det(\mathbf{I}+\nabla\mathbi{u})\\
&=\mathrm{const}
\end{align*}
\item[Conservation of Linear Momentum]
\begin{align*}
\rho_0\ddot{\mathbi{u}}&=\tDiv{\mathbf{FS}}+\mathbf{f}_0\\
&=\tDiv((\mathbf{I}+\nabla\mathbi{u})\mathbf{S})+\mathbf{f}_0\\
&=\tDiv(\mathbf{S}+\nabla\mathbi{u}\,\mathbf{S})+\mathbf{f}_0
\end{align*}
\item[Conservation of Angular Momentum]
\[
\mathbf{S}=\mathbf{S}^T
\]
\item[Conservation of Energy]
\begin{align*}
\mathbf{S}:\dot{\mathbi{e}}-\tDiv{\mathbf{q}_0}+r_0&=\rho_0\dot{e_0}\\
\mathbf{S}:\dot{\mathbi{e}}-\tDiv{\tGrad{k\theta}}+r_0
&=\rho_0(\dot\psi+\theta\dot\eta+\eta\dot\theta)
\end{align*}
\item[Second Law of Thermodynamics]
\[
\rho_0\dot{\eta_0}+\tDiv{\frac{k\tGrad{\theta}}{\theta}}-\frac{r_0}{\theta}\geq0
\]
This should be satisfied automatically by our choice of $\mathbf{S}$ and
$\eta_0$. Also, according to Fourier's law, $k$ should be negative.
\end{description}
Where
\[
\mathbf{S}=\left(\lambda(\tr\mathbi{e})
+2\mu\tr(\mathbi{e})
+c\theta\right)\mathbf{I}\,,
\]
\[
\eta_0=-\frac{1}{\rho_0}\left(c\tr\mathbi{e}+c_0\theta\right)\,,
\]
and
\[
\rho_0\dot\psi_0=\left(\lambda(\tr\mathbi{e})
+2\mu\tr(\mathbi{e})
+c\theta\right)\mathbf{I}:\dot{\mathbi{e}}
+\left(c\tr\mathbi{e}+c_0\theta\right)\dot\theta\,.
\]

The kinematic boundary conditions are
\begin{align*}
\rho_0\ddot{\mathbi{u}}=\mathbi{g}\quad&\text{on }\Gamma_g\\
\mathbi{u}=0\quad&\text{on }\Gamma_u\,,
\end{align*}
while the thermodynamic boundary conditions are
\begin{center}
% use packages: array
\begin{tabular}{lll}
$\rho_0\dot{e}=h$ & on & $\Gamma_q$ \\ 
$\theta=\tau(\mathbi{x},t)$ & on & $\Gamma_\theta$.
\end{tabular}
\end{center}
The initial kinematic conditions are
\begin{align*}
\mathbi{u}(\mathbi{x},0)=\mathbi{u}_0(\mathbi{x})\\
\frac{\partial\mathbi{u}(\mathbi{x},0)}{\partial t}=\mathbi{v}_0(\mathbi{x})\,.
\end{align*}
Additionally, the initial temperature and entropy must be chosen so
that they satisfy the governing equations. 
\end{document}

