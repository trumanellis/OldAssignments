\documentclass[letterpaper,10pt]{article}
\usepackage[top=.5in, bottom=.5in, left=.5in, right=.5in]{geometry}
\usepackage[fleqn]{amsmath}
\usepackage{amsfonts,amssymb,amsthm}
\usepackage{tabularx}
\usepackage{array}
\usepackage{multicol}

\setlength{\parindent}{0in}
\setlength{\parskip}{0ex}

\makeatletter
\newcommand{\rmnum}[1]{\romannumeral #1}
\newcommand{\Rmnum}[1]{\expandafter\@slowromancap\romannumeral #1@}
\makeatother

\DeclareMathOperator*{\tgrad}{grad}
\DeclareMathOperator*{\tGrad}{Grad}
\DeclareMathOperator*{\tdiv}{div}
\DeclareMathOperator*{\tDiv}{Div}
\DeclareMathOperator*{\tcurl}{curl}
\DeclareMathOperator*{\Cof}{Cof}
\DeclareMathOperator*{\tr}{tr}

\providecommand{\abs}[1]{\lvert#1\rvert}
\providecommand{\norm}[1]{\lVert#1\rVert}

\def\mathbi#1{\textbf{\em #1}}
\def\d{\mathrm{d}}
\def\half{\frac{1}{2}\,}
\def\bphi{\boldsymbol\varphi}
\def\x{\mathbi{x}}
\def\X{\mathbf{X}}
\def\u{\mathbi{u}}
\def\v{\mathbi{v}}
\def\n{\mathbi{n}}
\def\F{\mathbf{F}}
\def\E{\mathbf{E}}
\def\I{\mathbf{I}}
\newenvironment{al}{\begin{align*}}{\end{align*}}


\begin{document}
\begin{multicols}{3}
\begin{align*}
\delta_{ii}&=3\,,\\
\delta_{ij}\delta_{jk}&=\delta_{ik}\,,\\
\epsilon_{ijk}\epsilon_{imn}&=\delta_{jm}\delta_{kn}-\delta_{jn}\delta_{km}\,,\\
\epsilon_{ijk}\epsilon_{ijm}&=2\delta_{km}\,,\\
\epsilon_{ijk}\epsilon_{ijk}&=6\,.
\end{align*}
Cross Product
\begin{equation*}
\mathbf{a}\times\mathbf{b}=\epsilon_{ijk}a_i b_j\mathbf{e}_k
\end{equation*}
Gradient of a Scalar Field
\begin{equation*}
\nabla\phi=\mathbf{e}_i\partial_i\phi
\end{equation*}
Divergence of a Vector Field
\begin{equation*}
 \tdiv\mathbf{v}=\partial_i v_i
\end{equation*}
Curl of a Vector Field
\begin{equation*}
\tcurl\mathbf{v}=\epsilon_{ijk}\partial_i v_j\mathbf{e}_k
\end{equation*}
Gradient of a Vector Field
\begin{equation*}
\tgrad\mathbf{v}=\partial_j v_i\mathbf{e}_i\otimes\mathbf{e}_j
\end{equation*}
Divergence of a Tensor Field
\begin{equation*}
\tdiv\mathbf{A}=\partial_j A_{ij}\mathbf{e}_i
\end{equation*}
\begin{equation*}
\mathbi{u}=\bphi(\X)-\X
\end{equation*}
\begin{equation*}
\F(\X)=\nabla\bphi(\X)=\I+\nabla\u(\X)
\end{equation*}
\begin{equation*}
\Cof\F=\det\F\F^{-T}
\end{equation*}
\begin{equation*}
\mathbf{C}=\F^T\F
\end{equation*}
\begin{align*}
\E&=\half(\mathbf{C}-\I)\\
&=\half(\nabla\u+\nabla\u^T+\nabla\u^T\nabla\u)
\end{align*}
\begin{equation*}
e_1=\sqrt{1+2E_{11}}-1
\end{equation*}
\begin{equation*}
\sin{\gamma_{12}}=\frac{2E_{12}}{\sqrt{1+2E_{11}}\sqrt{1+2E_{22}}}
\end{equation*}
\begin{equation*}
\frac{\d\psi(\x,t)}{\d t}=\frac{\partial\psi(\x,t)}{\partial t}
+\v(\x,t)\cdot\frac{\partial\psi(\x,t)}{\partial\x}
\end{equation*}
\begin{equation*}
\mathbf{L}(\x,t)=\tgrad{\v(\x,t)}
\end{equation*}
\begin{equation*}
\dot\F=\mathbf{L}_m\F
\end{equation*}
\begin{equation*}
\mathbf{L}=\mathbf{D}+\mathbf{W}
\end{equation*}
\begin{equation*}
\mathbf{D}=\half(\mathbf{L}+\mathbf{L}^T)\\
\end{equation*}
\begin{equation*}
\mathbf{W}=\half(\mathbf{L}-\mathbf{L}^T)
\end{equation*}
\begin{equation*}
\mathbf{W}\v=\half\boldsymbol\omega\times\v
\end{equation*}
\begin{equation*}
\dot{\det\F}=\det\F\tdiv\v
\end{equation*}
\parbox{\textwidth}{
Piola Transform
\begin{equation*}
\mathbf{T}_0(\X)=\mathbf{T}(\x)\,\Cof\F(\X)
\end{equation*}
\begin{equation*}
\tDiv\mathbf{T}_0=\det\F\tdiv\mathbf{T}
\end{equation*}
\begin{equation*}
\mathbf{T}_0\n_0\,dA_0=\mathbf{T}\n\,dA
\end{equation*}
\begin{equation*}
dA=\det\F\norm{\F^{-T}\n_0}dA_0
\end{equation*}
\begin{equation*}
\n=\frac{\Cof\F\n_0}{\norm{\Cof\F\n_0}}
\end{equation*}
}
\begin{equation*}
\F=\mathbf{RU}=\mathbf{VR}
\end{equation*}
\begin{equation*}
\mathbf{C}=\mathbf{U}^2
\end{equation*}
\begin{equation*}
\mathbf{B}=\mathbf{V}^2
\end{equation*}
\begin{equation*}
(\mathbf{C}-\lambda\I)\mathbi{m}=\mathbf{0}
\end{equation*}
\begin{multline*}
\det(\mathbf{C}-\lambda\I)=-\lambda^3+\Rmnum{1}(\mathbf{C})\lambda^2\\
-\Rmnum{2}(\mathbf{C})\lambda+\Rmnum{3}(\mathbf{C})
\end{multline*}
\begin{align*}
&\Rmnum{1}(\mathbf{C})&=&\tr\mathbf{C}\\
&&=&\lambda_1+\lambda_2+\lambda_3\\
&\Rmnum{2}(\mathbf{C})&=&\tr\Cof\mathbf{C}\\
&&=&\lambda_1\lambda_2+\lambda_1\lambda_3+\lambda_2\lambda_3\\
&\Rmnum{3}(\mathbf{C})&=&\det\mathbf{C}\\
&&=&\lambda_1\lambda_2\lambda_3
\end{align*}
\begin{equation*}
\frac{\d}{\d t}\int_{\omega} \Psi\,dx
=\int_\omega\frac{\partial\Psi}{\partial t}\,dx
+\int_{\partial\omega}\Psi\v\cdot\n\,dA
\end{equation*}
\begin{equation*}
I(\mathcal{B},t)=\int_{\Omega_t}\rho\v\,dx
\end{equation*}
\begin{equation*}
H(\mathcal{B},t)=\int_{\Omega_t}\x\times\rho\v\,dx
\end{equation*}
\begin{equation*}
\frac{\d I(\mathcal{B},t)}{\d t}=\mathcal{F}(\mathcal{B},t)
\end{equation*}
Cauchy Stress:
\begin{equation*}
\mathbf{T}=\frac{\mathbf{PF^T}}{\det\F}=\frac{\F\mathbf{S}\F^T}{\det\F}
\end{equation*}
First Piola-Kirchhoff Stress:
\begin{equation*}
\mathbf{P}=(\det\F)\mathbf{T}\F^{-T}=\mathbf{T}\Cof\F=\F\mathbf{S}
\end{equation*}
Second Piola-Kirchhoff Stress:
\begin{equation*}
\mathbf{S}=(\det\F)\F^{-1}\mathbf{T}\F^{-T}=\F^{-1}\mathbf{P}
\end{equation*}
\begin{align*}
\mathcal{P}&=\int_{\Omega_t}\mathbf{f}\cdot\v\,dx+\int_{\partial\Omega_t}
\boldsymbol\sigma(\n)\cdot\v\,dA\\
&=\frac{\d \kappa}{\d
t}+\int_{\Omega_t}\mathbf{T}:\mathbf{D}\,dx
\end{align*}
\begin{equation*}
\mathbf{D}=\half(\tgrad\v+\tgrad\v^T)
\end{equation*}
\begin{multline*}
\int_{\Omega_0}\mathbf{f}_0\cdot\dot{\u}\,dX
+\int_{\partial\Omega_0}\mathbf{P}\n_0\cdot\dot{\u}\,dA_0\\
=\int_0\mathbf{P}:\dot{\F}\,dX+\frac{\d}{\d
t}\half\int_{\Omega_0}\rho_0\dot{\u}\cdot\dot{\u}\,dX
\end{multline*}
\begin{multline*}
\int_{\Omega_0}\mathbf{f}_0\cdot\dot{\u}\,dX
+\int_{\partial\Omega_0}\mathbf{FS}\n_0\cdot\dot{\u}\,dA_0\\
=\int_0\mathbf{S}:\dot{\E}\,dX
+\frac{\d}{\d t}\half\int_{\Omega_0}\rho_0\dot{\u}\cdot\dot{\u}\,dX
\end{multline*}
\begin{align*}
Q&=\int_{\partial\Omega_t}-\mathbf{q}\cdot\n\,dA+\int_{\Omega_t}r\,dx\\
&=\int_{\partial\Omega_0}-\mathbf{q}_0\cdot\n_0\,dA_0+\int_{\Omega_0}r_0\,dX
\end{align*}
\begin{equation*}
\frac{\d}{\d t}(\kappa+U)=\mathcal{P}+Q
\end{equation*}
Principle of Determinism

Principle of Material Frame Indifference

Principle of Physical Consistency

Principle of Material Symmetry

Principle of Local Action

Dimensional consistency

Existence, well-posedness

Equipresence
\begin{equation*}
\psi=e-\theta\eta
\end{equation*}
\begin{equation*}
\rho\frac{\d\psi}{\d t}-\rho\eta\frac{\d\theta}{\d t}+\mathbf{T}:\mathbf{D}
-\frac{\mathbf{q}}{\theta}\cdot\tgrad\theta\geq0
\end{equation*}
\begin{equation*}
\rho_0\dot{\psi_0}-\rho_0\eta_0\dot{\theta}+\mathbf{S}:\dot{\mathbf{E}}
-\frac{\mathbf{q}_0}{\theta}\cdot\nabla\theta\geq0
\end{equation*}
\begin{equation*}
\mathbf{S}=\rho_0\frac{\partial\Psi}{\partial\mathbf{E}}
\end{equation*}
\begin{equation*}
\eta_0=-\frac{\partial\Psi}{\partial\theta}
\end{equation*}
\end{multicols}

\renewcommand{\arraystretch}{2.5}
\begin{tabular}{|>{\centering}p{0.3\textwidth}|>{\centering}p{0.3\textwidth}|}
\hline
Lagrangian  & Eulerian
\tabularnewline
\hline
 \multicolumn{2}{|c|}{Conservation of Mass}
\tabularnewline
\hline
 $\rho_0=\rho\det\F$ & $\dfrac{\partial\rho}{\partial t}+\tdiv(\rho\v)=0$
\tabularnewline
\hline
 \multicolumn{2}{|c|}{Conservation of Linear Momentum}
\tabularnewline
\hline
 $\rho_0\ddot{\u}=\tDiv\F\mathbf{S}+\mathbf{f}_0$ &
$\rho\dfrac{\partial\v}{\partial t}
+\rho\v\cdot\tgrad\v=\tdiv\mathbf{T}+\mathbf{f}$
\tabularnewline
\hline
 \multicolumn{2}{|c|}{Conservation of Angular Momentum}
\tabularnewline
\hline
 $\mathbf{S}=\mathbf{S}^T$ & $\mathbf{T}=\mathbf{T}^T$
\tabularnewline
\hline
 \multicolumn{2}{|c|}{Conservation of Energy}
\tabularnewline
\hline
 $\rho_0\dot{e}_0=\mathbf{S}:\dot{\E}-\tDiv\mathbf{q}_0+r_0$ &
$\rho\dfrac{\d e}{\d t}=\mathbf{T}:\mathbf{D}-\tdiv\mathbf{q}+r$
\tabularnewline
\hline
 \multicolumn{2}{|c|}{Second Law of Thermodynamics}
\tabularnewline
\hline
$\rho_0\dot{\eta}_0+\tDiv{\dfrac{\mathbf{q}_0}{\theta}}
-\dfrac{r_0}{\theta}\geq 0$ &
$\rho\dfrac{\partial\eta}{\partial t}+\rho\v\cdot\tgrad\eta
+\tdiv\dfrac{\mathbf{q}}{\theta}-\dfrac{r}{\theta}\geq 0$
\tabularnewline
\hline
\end{tabular}
\end{document}

