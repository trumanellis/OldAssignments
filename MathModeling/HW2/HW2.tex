\documentclass[letterpaper,10pt]{article}
\usepackage{amsfonts,amsmath,amssymb,amsthm}
\usepackage{cancel}

\setlength{\parindent}{0in}
\setlength{\parskip}{.4ex}

%opening
\title{CAM 389C Homework 3}
\author{Truman Ellis}

\begin{document}

\maketitle

\section*{Exercise Set I.2}
\subsection*{Problem 7}
Let $U$ be a normed space and $f: U\rightarrow \mathbb{R}$ be a function from
$U$ to $\mathbb{R}$. The space $U'=L(U,\mathbb{R})$ is called the dual of $U$.
We say that $f$ is differentiable at $\mathbf{u}\in U'$ such that
\[
 \lim_{\theta\rightarrow0}\frac{1}{\theta}\left(f(\mathbf{u}
+\theta\mathbf{v})-f(\mathbf{u})\right)=\langle \mathbf{g},\mathbf{v}\rangle\,,
\]
$\forall \mathbf{v}\in U$. Here $\langle\cdot,\cdot\rangle$ denotes ``duality
pairing'', meaning the number
$\langle\mathbf{g},\mathbf{v}\rangle\in\mathbb{R}$ depends linearly on
$\mathbf{g}$ and on $\mathbf{v}$. In particular,
$\mathbf{g}(\mathbf{v})\equiv\langle\,\mathbf{g},\mathbf{v}\rangle$. We write
\[
 \mathbf{g}=Df(\mathbf{u})\,,
\]
and call $Df(\mathbf{u})$ the derivative of $f$ at $\mathbf{u}$.

Let $\varphi$ be a scalar-valued function defined on the set $\mathcal S$ of
invertible tensors $\mathbf{A}\in L(U,U)$ (i.e.
$\varphi:\mathcal{S}\rightarrow\mathbb{R}$) defined by
\[
 \varphi(\mathbf{A})=\det A\,.
\]

Show that, $\forall\mathbf{V}\in L(U,U)\,,$
\[
 D\varphi(\mathbf{A}):\mathbf{V}=(\det\mathbf{A})\mathbf{V}^T:\mathbf{A}^{-1}\,.
\]
\textbf{Hint:} Note that
\begin{align*}
 \det(\mathbf{A}+\theta\mathbf{V})&=\det((\mathbf{I}+\theta\mathbf{V}\mathbf{A}
^{-1})\mathbf{A})\\
&=\det\mathbf{A}\det(\mathbf{I}+\theta\mathbf{V}\mathbf{A}^{-1})\,,
\end{align*}
and that
\[
 \det(\mathbf{I}+\theta\mathbf{B})=1+\theta\,\mathrm{tr}\,\mathbf{B}+\mathit{O}
(\theta^2)\,.
\]
\subsubsection*{Solution}
By the definition of the derivative,
\begin{align*}
 D\varphi(\mathbf{A}):\mathbf{V}
&=\lim_{\theta\rightarrow 0}
\frac{\det(\mathbf{A}+\theta\mathbf{V})-\det(\mathbf{A} )}{\theta}\\
&=\lim_{\theta\rightarrow 0}
\frac{\det\mathbf{A}\det(\mathbf{I}+\theta\mathbf{V}\mathbf{A}^{-1})
-\det(\mathbf{A})}{\theta}\\
&=\lim_{\theta\rightarrow 0}
\frac{\det\mathbf{A}(1+\theta\,\mathrm{tr}(\mathbf{V}\mathbf{A}^{-1})+\mathit{O}
(\theta^2)))
-\det(\mathbf{A})}{\theta}\\
&=\lim_{\theta\rightarrow 0}
\frac{\det\mathbf{A}}{\theta}+
\frac{\theta(\det\mathbf{A})\mathbf{V}^T:\mathbf{A}^{-1}}{\theta}+
\frac{\mathit{O}(\theta^2)))}{\theta}-\frac{\det\mathbf{A}}{\theta}\\
&=\lim_{\theta\rightarrow 0}
\cancel{\frac{\det\mathbf{A}}{\theta}}+
\frac{\cancel{\theta}(\det\mathbf{A})\mathbf{V}^T:\mathbf{A}^{-1}}{\cancel{
\theta } } +
\frac{\mathit{O}(\theta^2)))}{\theta}-\cancel{\frac{\det\mathbf{A}}{\theta}}\\
&=\lim_{\theta\rightarrow 0}\left(
(\det\mathbf{A})\mathbf{V}^T:\mathbf{A}^{-1}+
\cancelto{0}{\frac{\mathit{O}(\theta^2)))}{\theta}}\right)\\
&=(\det\mathbf{A})\mathbf{V}^T:\mathbf{A}^{-1}
\end{align*}
\[
 \therefore \quad D\varphi(\mathbf{A}):\mathbf{V} =
(\det\mathbf{A})\mathbf{V}^T:\mathbf{A}^{-1}
\]

\subsection*{Problem 8}
Let $\Omega$ be an open set in $\mathbb{R}^3$ and $\varphi$ be a smooth
function mapping $\Omega$ into $\mathbb{R}$. The vector $\mathbf{g}$ with the
property
\[
 D\varphi(\mathbf{x})\cdot\mathbf{v}=\mathbf{g}(\mathbf{x})\cdot\mathbf{v}
\qquad \forall\mathbf{v}\in\mathbb{R}^3\,,
\]
is the gradient of $\mathbf{v}$ at point $\mathbf{x}\in\Omega$. We use the
classical notation,
\[
 \mathbf{g}(\mathbf{x})=\nabla\varphi(\mathbf{x})\,.
\]
Show that
\begin{align*}
\nabla(\varphi\mathbf{v})
&=\varphi\nabla\mathbf{v}+\mathbf{v}\otimes\nabla\varphi\,,\\
\operatorname{div}(\phi\mathbf{v})
&=\varphi\operatorname{div}\mathbf{v}+\mathbf{v}\cdot\nabla\varphi\,,\\
\operatorname{div}(\mathbf{v}\otimes\mathbf{w})
&=\mathbf{v}\operatorname{div}\mathbf{w}+(\nabla\mathbf{v})\mathbf{w}\,.
\end{align*}

\subsubsection*{Solution}
\textbf{Part 1}:\newline
Let's first look at derivative of the vector field $\mathbf{v}(\mathbf{x})$:
\[
D\mathbf{v}(\mathbf{x}):L(\mathbb{R}^3,\mathbb{R}^3)\rightarrow\mathbb{R}
\]
\[
D\mathbf{v}:\mathbb{R}^3\rightarrow L(\mathbb{R}^3,\mathbb{R}
^3)'
\]
for $x\in\mathbb{R}^3$:
\[
D\mathbf{v}(\mathbf{x})(\mathbf{e}_i\otimes\mathbf{e}_j)\equiv
\lim_{\theta\rightarrow
0}\frac{1}{\theta}(v_i(\mathbf{x}+\theta\mathbf{e}_j)-v_i(\mathbf{x}))
\]
\[
D\mathbf{v}(\mathbf{x})(\mathbf{A})
\equiv\sum_{kl}A_{kl}D\mathbf{v}(\mathbf{x})(\mathbf{e}_k\otimes\mathbf
{e}_l)
\]
And the derivative of a scalar field:
\[
D\varphi(x):\mathbb{R}^3\rightarrow\mathbb{R}
\]
\[
D\varphi:\mathbb{R}^3\rightarrow(\mathbb{R}^3)'
\]
\[
D\varphi(\mathbf{x})(\mathbf{e}_i)
\equiv\lim_{\theta\rightarrow
0}\frac{1}{\theta}(\varphi(\mathbf{x}+\theta\mathbf{e}_i)-\varphi(\mathbf{x}))
\]
With these preliminaries taken care of, we can derive our first identity.
\[
\nabla(\varphi\mathbf{v})=\varphi\nabla\mathbf{v}+\mathbf{v}\otimes\nabla\varphi
\]
If this is true, it should hold that
\[
\nabla(\varphi\mathbf{v}):\mathbf{A}=(\varphi\nabla\mathbf{v}+\mathbf{v}
\otimes\nabla\varphi):\mathbf{A} \qquad \forall \mathbf{A}\in
L(\mathbb{R}^3,\mathbb{R}^3)
\]
Starting with the left hand side:
\begin{align*}
\nabla(\varphi\mathbf{v}):\mathbf{A}
&=\nabla(\varphi\mathbf{v})(\mathbf{x})(\mathbf{A})\\
&=\sum_{ij}A_{ij}\lim_{\theta\rightarrow 0}\frac{1}{\theta}((\varphi
v_i)(\mathbf{x}+\theta\mathbf{e}_j)-(\varphi v_i)(\mathbf{x}))\\
&=\sum_{ij}A_{ij}\varphi\frac{\partial v_i}{\partial x_j}
+\sum_{ij}A_{ij}v_i\frac{\partial\varphi}{\partial x_j}\\
&=\sum_{ij}A_{ij}\varphi\lim_{\theta\rightarrow
0}\frac{1}{\theta}(v_i(\mathbf{x}+\theta\mathbf{e}_j)-v_i(\mathbf{x}))
+\sum_{ij}A_{ij}\varphi\lim_{\theta\rightarrow
0}\frac{1}{\theta}(\varphi(\mathbf{x}+\theta\mathbf{e}_j)-\varphi(\mathbf{x}))\\
&=\varphi(\mathbf{x}D\mathbf{v}(\mathbf{x})\left(\sum_{ij}A_{ij}(\mathbf{e}
_i\otimes\mathbf{e}_j)\right)
+A:\left(\sum_{ij}v_iD\varphi(\mathbf{x})(\mathbf{e}_j)(\mathbf{e}
_i\otimes\mathbf{e}_j)\right)\\
&=\varphi(\mathbf{x})D\mathbf{v}(\mathbf{x})(\mathbf{A})
+\mathbf{A}:\left(\sum_{ij}v_i\nabla\varphi\mathbf{e}_j(\mathbf{e}
_i\otimes\mathbf{e}_j)\right)\\
&=\varphi\nabla\mathbf{v}:\mathbf{A}
+\mathbf{A}:\left((\sum_i
v_i\mathbf{e}_i)\otimes(\sum_j(\nabla\varphi\cdot\mathbf{e}_j)\mathbf{e}
_j)\right)\\
&=\left(\varphi\nabla\mathbf{v}+\mathbf{v}\otimes\nabla\varphi\right):\mathbf{A}
\end{align*}
Therefore,
\[
\nabla(\varphi\mathbf{v})=\varphi\nabla\mathbf{v}+\mathbf{v}\otimes\nabla\varphi
\]

\textbf{Part 2}:\newline
We now wish to show that
\[
\operatorname{div}(\varphi\mathbf{v})=\varphi\operatorname{div}\mathbf{v}
+\mathbf{v}\cdot\nabla\varphi
\]
One definition of the divergence is
\[
\operatorname{div}(\mathbf{v})=\operatorname{tr}(\nabla\mathbf{v})
\]
With this definition of the divergence, the identity becomes
\[
\operatorname{div}(\varphi\mathbf{v})=\operatorname{tr}(\nabla(\varphi\mathbf{v}
))\,,
\]
which is just the trace of the identity that we proved in Part 1.
Therefore from Identity 1:
\begin{align*}
\operatorname{div}(\varphi\mathbf{v})
&=\operatorname{tr}(\nabla(\varphi\mathbf{v}))\\
&=\operatorname{tr}(\varphi\nabla\mathbf{v}+\mathbf{v}\otimes\nabla\varphi)\\
&=\varphi\operatorname{tr}(\nabla\mathbf{v})+\operatorname{tr}\left(\sum_{ij}
v_i\mathbf{e}_i\otimes\mathbf{e}_j\partial_j\varphi\right)\\
&=\varphi\operatorname{tr}(\nabla\mathbf{v})+v_i\partial_i\varphi\\
&=\varphi\operatorname{div}\mathbf{v}+\mathbf{v}\cdot\nabla\varphi
\end{align*}
Therefore
\[
\operatorname{div}(\varphi\mathbf{v})=\varphi\operatorname{div}\mathbf{v}
+\mathbf{v}\cdot\nabla\varphi
\]

\textbf{Part 3}:\newline
Finally, we need to show that
\[
\operatorname{div}(\mathbf{v}\otimes\mathbf{w})
=\mathbf{v}\operatorname{div}\mathbf{w}+(\nabla\mathbf{v})\mathbf{w}\,.
\]
If we dot product both sides of the identity with an arbitrary vector
$\mathbf{u}$, we arrive at
\[
\mathbf{u}\cdot\operatorname{div}(\mathbf{v}\otimes\mathbf{w})
=\mathbf{u}\cdot\left(\mathbf{v}\operatorname{div}\mathbf{w}+(\nabla\mathbf{v}
)\mathbf{w}\right) \qquad \forall\mathbf{u}\in\mathbb{R}^3
\]
Expanding out the left hand side using
$\mathbf{a}\cdot\operatorname{div}(\mathbf{A})=\operatorname{div}(\mathbf{A}
^T\mathbf{a})$ and
$(\mathbf{a}\otimes\mathbf{b})\mathbf{c}=(\mathbf{b}\cdot\mathbf{c})\mathbf{a}$:
\begin{align*}
\mathbf{u}\cdot\operatorname{div}(\mathbf{v}\otimes\mathbf{w})
&=\operatorname{div}((\mathbf{v}\otimes\mathbf{w})^T\mathbf{u})\\
&=\operatorname{div}(\mathbf{w}\otimes\mathbf{v}\mathbf{u})\\
&=\operatorname{div}((\mathbf{v}\cdot\mathbf{u})\mathbf{w})\\
\end{align*}
If we replace the scalar $\psi=(\mathbf{v}\cdot\mathbf{u})$, we get
\[
\operatorname{div}((\mathbf{v}\cdot\mathbf{u})\mathbf{w})=\operatorname{div}
(\psi\mathbf{w})
\]
Now we can use our identity from Part 2.
\begin{align*}
\operatorname{div}(\psi\mathbf{w})
&=\psi\operatorname{div}\mathbf{w}+\mathbf{w}\cdot\nabla\psi
\end{align*}
Now, examining the second term on the left hand side:
\begin{align*}
\mathbf{w}\cdot\nabla\psi
&=\mathbf{w}\cdot\nabla(\mathbf{v}\cdot\mathbf{u})\\
&=w_i\partial_i(v_j u_j)\\
&=w_i v_j\cancel{\partial_i u_j} + w_i u_j\partial_i v_j\\
&\mathbf{u}\text{ is a constant vector}\\
&=u_j(\partial_i v_j)w_i\\
&=\mathbf{u}\cdot(\nabla\mathbf{v})\mathbf{w}
\end{align*}
Putting everything back together:
\begin{align*}
\mathbf{u}\cdot\operatorname{div}(\mathbf{v}\otimes\mathbf{w})
&=(\mathbf{v}\cdot\mathbf{u})\operatorname{div}\mathbf{w}+\mathbf{w}
\cdot\nabla(\mathbf{v}\cdot\mathbf{u})\\
&=(\mathbf{v}\cdot\mathbf{u})\operatorname{div}\mathbf{w}+\mathbf{u}
\cdot(\nabla\mathbf{v})\mathbf{w}\\
&=\mathbf{u}\cdot\left(\mathbf{v}\operatorname{div}\mathbf{w}
+(\nabla\mathbf{v})\mathbf{w}\right)
\end{align*}
Therefore, the identity holds.

\subsection*{Problem 9}
Let $\varphi$ and $\mathbf{u}$ be $\mathit{C}^2$ scalar and vector fields. Show
that
\begin{align*}
\operatorname{curl} \nabla\varphi &= \mathbf{0}\\
\operatorname{div}\operatorname{curl}\mathbf{v}&=0
\end{align*}
\subsubsection*{Solution}
\textbf{Part 1}:\newline
First note that
\[
(\nabla\varphi)_j=(\mathbf{e}_i\partial_i\varphi)_j=\partial_j\varphi
\]
Now
\begin{align*}
\operatorname{curl}\nabla\varphi
&=\varepsilon_{ijk}(\partial_i(\nabla\varphi)_j)\mathbf{e}_k\\
&=\varepsilon_{ijk}(\partial_i\partial_j\varphi)\mathbf{e}_k
\end{align*}
If we choose any $k=\hat k$, the remaining indices must be unique to produce a
non-zero $\varepsilon_{ijk}$ term. This only leaves two non-zero terms in the
sum: $\varepsilon_{ij\hat k}(\partial_i\partial_j\varphi)$ and
$\varepsilon_{ji\hat k}(\partial_j\partial_i\varphi)$. Since spatial
derivatives are commutative,
$\partial_i\partial_j\varphi=\partial_j\partial_i\varphi$, and since
$\varepsilon_{ji\hat k}$ is an odd permutation of $\varepsilon_{ij\hat k}$, the
terms cancel. Thus we get for any
$k$, $\varepsilon_{ijk}(\partial_i\partial_j\varphi)=0$, and
$\operatorname{curl}\nabla\varphi=\mathbf{0}$.

\textbf{Part 2}:\newline
First note that
\[
(\varepsilon_{ijm}(\partial_i v_j)\mathbf{e}_m)_k
=\varepsilon_{ijk}(\partial_i v_j)
\]
Now
\begin{align*}
\operatorname{div}\operatorname{curl}\mathbf{v}
&=\partial_k(\varepsilon_{ijm}(\partial_i v_j)\mathbf{e}_m)_k\\
&=\partial_k(\varepsilon_{ijk}(\partial_i v_j)\\
&=\varepsilon_{ijk}\partial_k\partial_i v_j
\end{align*}
By a similar argument from before. Choose any $j=\hat j$, the remaining indices
must be unique to produce a non-zero $\varepsilon_{ijk}$ term. This only leaves
two non-zero terms in the sum:
$\varepsilon_{i\hat jk}(\partial_k\partial_i v_{\hat j})$ and
$\varepsilon_{k\hat jk}(\partial_i\partial_k v_{\hat j})$. Since spatial
derivatives are commutative,
$\partial_k\partial_i v_{\hat j}=\partial_i\partial_k v_{\hat j}$, and since
$\varepsilon_{k\hat ji}$ is an odd permutation of $\varepsilon_{i\hat jk}$, the
terms cancel. Thus we get for any
$j$, $\varepsilon_{ijk}(\partial_k\partial_i v_j)=0$, and
$\operatorname{div}\operatorname{curl}\mathbf{v}=0$.

\subsection*{Problem 10}
Let $\Omega$ be an open, connected, smooth domain in $\mathbb{R}^3$ with
boundary $\partial\Omega$. Let $\mathbf{n}$ be a unit exterior normal to
$\partial\Omega$. Show that
\[
\int_\Omega
\operatorname{div}\mathbf{A}\,dx=\int_{\partial\Omega}\mathbf{An}\,dA\,.
\]

\subsubsection*{Solution}
Note that for any arbitrary vector $\mathbf{u}$,
\[
\mathbf{u}\cdot\int_{\partial\Omega}\mathbf{An}\,dA=\int_{\partial\Omega}
(\mathbf{A}^T\mathbf{u})\cdot\mathbf{n}\,dA
\qquad\forall\mathbf{u}\in\mathbb{R}^3
\]
If we use the divergence theorem:
\[
\int_\Omega\operatorname{div}\mathbf{v}\,dx
=\int_{\partial\Omega}\mathbf{v}\cdot\mathbf{n}\,dA\,,
\]
We get
\begin{align*}
\mathbf{u}\cdot\int_{\partial\Omega}\mathbf{An}\,dA
&=\int_{\partial\Omega}
(\mathbf{A}^T\mathbf{u})\cdot\mathbf{n}\,dA\\
&=\int_\Omega\operatorname{div}(\mathbf{A}^T\mathbf{u})\,dx
\end{align*}
Now using the identity
\[
\operatorname{div}(\mathbf{A}^T\mathbf{u})=\mathbf{u}\cdot\operatorname{div}(
\mathbf{A})
\]
\begin{align*}
\mathbf{u}\cdot\int_{\partial\Omega}\mathbf{An}\,dA
&=\int_\Omega\mathbf{u}\cdot\operatorname{div}(\mathbf{A})\,dx\\
&=\mathbf{u}\cdot\int_\Omega\operatorname{div}(\mathbf{A})\,dx
\end{align*}
since $\mathbf{u}$ is constant. Therefore, the identity holds.

\subsection*{Problem 11}
If $(x,y,z)$ is a Cartesian coordinate system with origin at the corner of a
cube $\Omega_0=(0,1)^3$ and axes along edges of the cube, compute
\[
\int_{\partial\Omega_0}\mathbf{v}\cdot\mathbf{n}\,dA_0\,,
\]
where $\partial\Omega_0$ is the exterior surface of $\Omega_0$, $\mathbf{n}$ is
the unit exterior normal vector, and $\mathbf{v}$ is the field,
$\mathbf{v}=x\mathbf{e}_1+y\mathbf{e}_2+3z\mathbf{e}_3$.
\subsubsection*{Solution}
Using the divergence theorem
\begin{align*}
\int_{\partial\Omega_0}\mathbf{v}\cdot\mathbf{n}\,dA_0
&=\int_{\Omega_0}\operatorname{div}\mathbf{v}\,dx\\
&=\int_{\Omega_0}\frac{\partial x}{\partial x}+\frac{\partial y}{\partial
y}+\frac{\partial 3z}{\partial z}\,dx\\
&=1+1+3=5
\end{align*}

\end{document}


\subsection*{Problem 8}
Let $\Omega$ be an open set in $\mathbb{R}^3$ and $\varphi$ be a smooth
function mapping $\Omega$ into $\mathbb{R}$. The vector $\mathbf{g}$ with the
property
\[
 D\varphi(\mathbf{x})\cdot\mathbf{v}=\mathbf{g}(\mathbf{x})\cdot\mathbf{v}
\qquad \forall\mathbf{v}\in\mathbb{R}^3\,,
\]
is the gradient of $\mathbf{v}$ at point $\mathbf{x}\in\Omega$. We use the
classical notation,
\[
 \mathbf{g}(\mathbf{x})=\nabla\varphi(\mathbf{x})\,.
\]
Show that
\begin{align*}
\nabla(\varphi\mathbf{v})
&=\varphi\nabla\mathbf{v}+\mathbf{v}\otimes\nabla\varphi\,,\\
\operatorname{div}(\phi\mathbf{v})
&=\varphi\operatorname{div}\mathbf{v}+\mathbf{v}\cdot\nabla\varphi\,,\\
\operatorname{div}(\mathbf{v}\otimes\mathbf{w})
&=\mathbf{v}\operatorname{div}\mathbf{w}+(\nabla\mathbf{v})\mathbf{w}\,.
\end{align*}

\subsubsection*{Solution}
\textbf{Part 1}:\newline
First note that
\[
\varphi\nabla\mathbf{v} = \varphi\partial_j u_i\mathbf{e}_i\otimes\mathbf{e}_j
\]
and
\[
\mathbf{v}\otimes\nabla\varphi
=v_i\mathbf{e}_i\otimes\partial_j\varphi\mathbf{e}_j
\]
Then
\begin{align*}
\nabla(\varphi\mathbf{v})
&=\partial_j\varphi v_i\mathbf{e}_i\otimes\mathbf{e}_j\\
&=\varphi\partial_j v_i\mathbf{e}_i\otimes\mathbf{e}_j
+v_i\mathbf{e}_i\otimes\partial_j\varphi\mathbf{e}_j\\
&=\varphi\nabla\mathbf{v}+\mathbf{v}\otimes\nabla\varphi
\end{align*}

\textbf{Part 2}:\newline
First note that
\[
\varphi\operatorname{div}\mathbf{v}
=\varphi\partial_i v_i
\]
and
\[
\mathbf{v}\cdot\nabla\varphi
=v_i\partial_i\varphi
\]
Then
\begin{align*}
\operatorname{div}(\varphi\mathbf{v})
&=\partial_i\varphi v_i\\
&=\varphi\partial_i v_i+v_i\partial_i\varphi\\
&=\varphi\operatorname{div}\mathbf{v}+v\otimes
\end{align*}

\textbf{Part 3}:\newline
First note that
\[
\mathbf{v}\operatorname{div}\mathbf{w}
=v_i\mathbf{e}_i\partial_j w_j
\]
and
\[
(\nabla\mathbf{v})w
=(\partial_k v_i\mathbf{e}_i\otimes\mathbf{e}_k)w_j\mathbf{e}_j
=w_j\partial_k v_i\mathbf{e}_i\delta_{jk}
=w_j\partial_j v_i\mathbf{e}_i
\]
Then
\begin{align*}
\operatorname{div}(\mathbf{v}\otimes\mathbf{w})
&=\partial_j v_i w_j \mathbf{e}_j\\
&=v_i\partial_j w_j\mathbf{e}_i+w_j\partial_j v_i\mathbf{e}_i\\
&=\mathbf{v}\operatorname{div}\mathbf{w}+(\nabla\mathbf{v})w
\end{align*}