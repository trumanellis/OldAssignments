\documentclass[letterpaper,10pt]{article}
\usepackage{amsfonts,amsmath,amssymb,amsthm}
\usepackage{cancel}
\usepackage{graphicx}
\usepackage{multirow}
\usepackage{fullpage}
\usepackage{algorithmic}

\usepackage{ifthen}
\usepackage[dotinlabels]{titletoc}
\usepackage{colortbl}
\usepackage{array}

\newcolumntype{I}{!{\vrule width 1.5pt}}
\newlength\savedwidth
\newcommand\whline{\noalign{\global\savedwidth\arrayrulewidth
                            \global\arrayrulewidth 1.5pt}%
           \hline
           \noalign{\global\arrayrulewidth\savedwidth}}


% 
% A FLAME algorithm is bracketed by
%
%  \begin{FlameAlg}
%    <statements>
%  \end{FlameAlg}

\newenvironment{FlameAlg}{
\begin{tabbing}
in \= in \= in  \= in  \= in  \= \kill
}
{
\end{tabbing}
}

\newboolean{IsWide}
\setboolean{IsWide}{true}

\newenvironment{FlameAlgNarrow}{
\setboolean{IsWide}{false}
\begin{tabbing}
in \= in \= in  \= in  \= in  \= \kill
}
{
\end{tabbing}
\setboolean{IsWide}{true}
}

%
% A loop in a FLAME algorithm is bracketed by
%
% \FlaDoUntil{ <condition> }
%   Body of the loop
% \FlaEndDo
%
% The body is indented

\newcommand{\FlaDoUntil}[1]{
{ \bf do until #1 } \+ 
}

\newcommand{\FlaDoWhile}[1]{{\bf while #1 do} \+ 
}

% Note: because of the \kill, it is important
% to have a \\ before the \FlaEndDo
% since otherwise the last line before the
% \FlaEndDo will not show

\newcommand{\FlaEndDo}{\- \kill
{\bf endwhile }
}

% In math mode, 
% \FlaTwoByTwo{A}{B}
%             {C}{D}
% creates the picture
%   / A || B \
%   | ==  == |
%   \ C || D /

\newcommand{\FlaTwoByTwo}[4]{
\left( 
\begin{array}{c I c}
#1 & #2 \\ \whline
#3 & #4 
\end{array} 
\right)
}

\newcommand{\FlaTwoByTwoNoPar}[4]{
\begin{array}{c I c}
#1 & #2 \\ \whline
#3 & #4 
\end{array} 
}

\newcommand{\FlaTwoByTwoSingleLineNoPar}[4]{
\begin{array}{c | c}
#1 & #2 \\ \hline
#3 & #4 
\end{array} 
}

\newcommand{\FlaTwoByTwoSingleLine}[4]{
\left(  
\begin{array}{c | c}
#1 & #2 \\ \hline
#3 & #4 
\end{array} 
\right)
}

\newcommand{\FlaTwoByTwoDoubleLine}[4]{
\left(  
\begin{array}{c || c}
#1 & #2 \\ \hline\hline
#3 & #4 
\end{array} 
\right)
}


\newcommand{\FlaTwoByTwoNoLine}[4]{
\left(  
\begin{array}{c  c}
#1 & #2 \\ 
#3 & #4 
\end{array} 
\right)
}

% In math mode, 
% \FlaTwoByOne{A}
%             {C}
% creates the picture
%   / A  \
%   | == |
%   \ C  /

\newcommand{\FlaTwoByOne}[2]{
\left( 
\begin{array}{c}
#1 \\ \whline
#2 
\end{array} 
\right)
}

\newcommand{\FlaTwoByOneNoPar}[2]{
\begin{array}{c}
#1 \\ \whline
#2 
\end{array} 
}

% In math mode, 
% \FlaTwoByOneSingleLine{A}
%                       {C}
% creates the picture
%   / A  \
%   | -- |
%   \ C  /

\newcommand{\FlaTwoByOneNoParSingleLine}[2]{
\begin{array}{c}
#1 \\ \hline
#2 
\end{array} 
}

\newcommand{\FlaTwoByOneSingleLine}[2]{
\left( 
\begin{array}{c}
#1 \\ \hline
#2 
\end{array} 
\right)
}

\newcommand{\FlaTwoByOneDoubleLine}[2]{
\left( 
\begin{array}{c}
#1 \\ \hline\hline
#2 
\end{array} 
\right)
}


\newcommand{\FlaTwoByOneNoLine}[2]{
\left( 
\begin{array}{c}
#1 \\
#2 
\end{array} 
\right)
}

% In math mode, 
% \FlaOneByTwo{A}{B}
% creates the picture
%   ( A || B )

\newcommand{\FlaOneByTwo}[2]{
\left( 
\begin{array}{c I c}
#1 & #2 
\end{array} 
\right)
}

\newcommand{\FlaOneByTwoNoPar}[2]{
\begin{array}{c I c}
#1 & #2 
\end{array} 
}

\newcommand{\FlaOneByTwoSingleLine}[2]{
\left( 
\begin{array}{c | c}
#1 & #2
\end{array} 
\right)
}

\newcommand{\FlaOneByTwoNoLine}[2]{
\left( 
\begin{array}{c c}
#1 & #2
\end{array} 
\right)
}

% In math mode, 
% \FlaThreeByThreeTL{A}{B}{C}
%                   {D}{E}{F}
%                   {G}{H}{I}
% creates the picture
%   / A | B || C \
%   | -- ---  -- |
%   | D | E || F |
%   | ==  ==  == |
%   \ G | H || I /
% Notice: the TL means that the
% center block (E) is part of the
% TL quadrant, where quadrants are
% partitioned by the double lines.

\newcommand{\FlaThreeByThreeTL}[9]{
\left( 
\begin{array}{c | c I c}
#1 & #2 & #3 \\ \hline
#4 & #5 & #6 \\ \whline
#7 & #8 & #9
\end{array} 
\right) 
}

% In math mode, 
% \FlaThreeByThreeBR{A}{B}{C}
%                   {D}{E}{F}
%                   {G}{H}{I}
% creates the picture
%   / A || B | C \
%   | ==  ==  == |
%   | D || E | F |
%   | -- ---  -- |
%   \ G || H | I /
% Notice: the BR means that the
% center block (E) is part of the
% BR quadrant, where quadrants are
% partitioned by the double lines.

\newcommand{\FlaThreeByThreeBR}[9]{
\left( 
\begin{array}{c I c | c}
#1 & #2 & #3 \\ \whline
#4 & #5 & #6 \\ \hline
#7 & #8 & #9
\end{array} 
\right)
}

\newcommand{\FlaThreeByThreeSingleLine}[9]{
\left( 
\begin{array}{c | c | c}
#1 & #2 & #3 \\ \hline
#4 & #5 & #6 \\ \hline
#7 & #8 & #9
\end{array} 
\right)
}

\newcommand{\FlaThreeByThreeNoLine}[9]{
\left( 
\begin{array}{c c c}
#1 & #2 & #3 \\ 
#4 & #5 & #6 \\ 
#7 & #8 & #9
\end{array} 
\right)
}

% In math mode, 
% \FlaThreeByThreeTR{A}{B}{C}
%                   {D}{E}{F}
%                   {G}{H}{I}
% creates the picture
%   / A || B | C \
%   | -- ---  -- |
%   | D || E | F |
%   | ==  ==  == |
%   \ G || H | I /
% Notice: the TR means that the
% center block (E) is part of the
% TR quadrant, where quadrants are
% partitioned by the double lines.

\newcommand{\FlaThreeByThreeTR}[9]{
\left( 
\begin{array}{c I c | c}
#1 & #2 & #3 \\ \hline
#4 & #5 & #6 \\ \whline
#7 & #8 & #9
\end{array} 
\right)
}


% In math mode, 
% \FlaThreeByThreeBL{A}{B}{C}
%                   {D}{E}{F}
%                   {G}{H}{I}
% creates the picture
%   / A | B || C \
%   | ==  ==  == |
%   | D | E || F |
%   | -- ---  -- |
%   \ G | H || I /
% Notice: the BL means that the
% center block (E) is part of the
% BL quadrant, where quadrants are
% partitioned by the double lines.

\newcommand{\FlaThreeByThreeBL}[9]{
\left( 
\begin{array}{c | c I c}
#1 & #2 & #3 \\ \whline
#4 & #5 & #6 \\ \hline 
#7 & #8 & #9
\end{array} 
\right)
}

% In math mode, 
% \FlaOneByThreeR{A}{B}{C}
% creates the picture
%   ( A || B | C )
% Notice: the R means that the
% center block (B) is part of the
% R(ight) submatrix, where 
% submatrices are % partitioned 
% by the double lines.

\newcommand{\FlaOneByThreeR}[3]{
\left( 
\begin{array}{c I c | c}
#1 & #2 & #3 
\end{array} 
\right)
}

% In math mode, 
% \FlaOneByThreeL{A}{B}{C}
% creates the picture
%   ( A | B || C )
% Notice: the R means that the
% center block (B) is part of the
% R(ight) submatrix, where 
% submatrices are % partitioned 
% by the double lines.

\newcommand{\FlaOneByThreeL}[3]{
\left( 
\begin{array}{c | c I c}
#1 & #2 & #3 
\end{array} 
\right)
}

% In math mode, 
% \FlaThreeByOneT{A}
%                {D}
%                {G}
% creates the picture
%   / A  \
%   | == |
%   | B  |
%   | -- |
%   \ C  /
% Notice: the T means that the
% center block (C) is part of the
% T(op) submatrix where submatrices
% are % partitioned by the double 
% lines.

\newcommand{\FlaThreeByOneT}[3]{
\left( 
\begin{array}{c}
#1 \\ \hline
#2 \\ \whline
#3 
\end{array} 
\right) 
}

\newcommand{\FlaThreeByOneTNoPar}[3]{
\begin{array}{c}
#1 \\ \hline
#2 \\ \whline
#3 
\end{array} 
}

% In math mode, 
% \FlaThreeByOneB{A}
%                {D}
%                {G}
% creates the picture
%   / A  \
%   | -- |
%   | B  |
%   | == |
%   \ C  /
% Notice: the B means that the
% center block (C) is part of the
% T(op) submatrix where submatrices
% are % partitioned by the double 
% lines.

\newcommand{\FlaThreeByOneB}[3]{
\left( 
\begin{array}{c}
#1 \\ \whline
#2 \\ \hline
#3 
\end{array} 
\right) 
}

\newcommand{\FlaThreeByOneBNoPar}[3]{
\begin{array}{c}
#1 \\ \whline
#2 \\ \hline
#3 
\end{array} 
}

\newcommand{\FlaThreeByOneSingleLine}[3]{
\left( 
\begin{array}{c}
#1 \\ \hline
#2 \\ \hline
#3 
\end{array} 
\right) 
}

\newcommand{\FlaThreeByOneNoLine}[3]{
\left( 
\begin{array}{c}
#1 \\ 
#2 \\ 
#3 
\end{array} 
\right) 
}

% Various key words

% The following is a typical use of 
% \FlaPartition:
% 
% \FlaPartition {
%    $ A \rightarrow \FlaTwoByTwo{ A_{TL} }{ A_{TR} }
%                                { A_{BL} }{ A_{BR} } $ 
% }
% {
%    where $ A_{TL} $ is $ 0 \times 0 $
% }
% 
% Creates something like
% Partition A -> / A_TL || A_TR \
%                | =====  ===== |
%                \ A_BL || A_BR /
% where A_TL is 0 x 0
%

\newcommand{\FlaPartition}[2]{
\ifthenelse{\boolean{IsWide}}{{\bf partition } \hspace{-1em} #1 \hspace{-1em} #2}
{{\bf partition } \+ \\ #1 \+ \\ #2 \- \-}
}

% The following is a typical use of 
% \FlaRepartition:
% 
% \FlaRepartition{
% $ 
% \FlaTwoByTwo{ A_{TL} }{ A_{TR} }
%             { A_{BL} }{ A_{BR} } \rightarrow
% \FlaThreeByThreeBR{ A_{00} }{ A_{01} }{ A_{02} }
%                   { A_{10} }{ A_{11} }{ A_{12} }
%                   { A_{20} }{ A_{21} }{ A_{22} } 
% $ 
% }
% {
%    \FlaWhere{$ A_{11} $ is $ b \times b $}
%  }
% 
% Creates something like
% Repartition  
%   / A_TL || A_TR \    / A_00 || A_01 | A_02 \
%   | =====  ===== | -> | =====  ====== ===== |
%   \ A_BL || A_BR /    | A_10 || A_11 | A_12 |
%                       | -----  ------ ----- |
%                       \ A_20 || A_21 | A_22 /
% where A_11 is b x b
%

\newcommand{\FlaRepartition}[2]{
\ifthenelse{\boolean{IsWide}}{{\bf repartition } \hspace{-1em} #1 \hspace{-1em} #2}
{{\bf repartition } \+ \\ #1 \+ \\ #2 \- \-}
}

% The following is a typical use of 
% \FlaContinue:
% 
% \FlaContinue{
% $ 
%    \FlaTwoByTwo{ A_{TL} }{ A_{TR} }
%                { A_{BL} }{ A_{BR} } \leftarrow
%    \FlaThreeByThreeTL{ A_{00} }{ A_{01} }{ A_{02} }
%                      { A_{10} }{ A_{11} }{ A_{12} }
%                      { A_{20} }{ A_{21} }{ A_{22} } 
% $ 
% }
% 
% Creates something like
% Continue with
%   / A_TL || A_TR \    / A_00 | A_01 || A_02 \
%   | =====  ===== | <- | ----- ------  ----- |
%   \ A_BL || A_BR /    | A_10 | A_11 || A_12 |
%                       | ===== ======  ===== |
%                       \ A_20 | A_21 || A_22 /

\newcommand{\FlaContinue}[1]{
\ifthenelse{\boolean{IsWide}}{{\bf continue with } #1
}
{{\bf continue with } \+ \\ #1 \-
}
}
\newcommand{\FlaWhere}[1]{
\hspace{-1em} { \bf where #1 } 
}

\newcommand{\undetermined}{ \star }

\newcommand{\FlaStartCompute}{%
\setlength{\unitlength}{1.60in}%
\begin{picture}(2,0.01)
\put(0,0){\line(1,0){2}}
\put(0,0.01){\line(1,0){2}}
\end{picture}%
}

\newcommand{\FlaEndCompute}{%
\noindent%
\setlength{\unitlength}{1.60in}%
\begin{picture}(2,0.01)
\put(0,0){\line(1,0){2}} 
\put(0,0.01){\line(1,0){2}} 
\end{picture}%
}

\newcommand{\FlaStartComputeShort}{
\setlength{\unitlength}{1.30in}
\begin{picture}(3,0.01)
\put(0,0){\line(1,0){3}}
\put(0,0.01){\line(1,0){3}}
\end{picture} 
}

\newcommand{\FlaStartComputeShorter}{
\setlength{\unitlength}{0.5in}
\begin{picture}(3,0.01)
\put(0,0){\line(1,0){3}}
\put(0,0.01){\line(1,0){3}}
\end{picture} 
}

\newcommand{\FlaEndComputeShort}{
\setlength{\unitlength}{1.30in}
\begin{picture}(3,0.01)
\put(0,0){\line(1,0){3}} 
\put(0,0.01){\line(1,0){3}} 
\end{picture} 
}

\newcommand{\FlaEndComputeShorter}{
\setlength{\unitlength}{0.5in}
\begin{picture}(3,0.01)
\put(0,0){\line(1,0){3}} 
\put(0,0.01){\line(1,0){3}} 
\end{picture} 
}

\newcommand{\FlaUpLo}[2]{
{#1} \backslash {#2}
}

\newcommand{\FlaInverse}[1]{
{ #1 }^{-1}
}

\newcommand{\Square}{ \mbox{Square} }
\newcommand{\Symm}{ \mbox{Symm} }
\newcommand{\SameSize}{ \mbox{SameSize} }
\newcommand{\RowDim}{\mbox{RowDim}}
\newcommand{\ColDim}{\mbox{ColDim}}
\newcommand{\LowTr}{\mbox{LowTr}}
\newcommand{\UpTr}{\mbox{UpTr}}
\newcommand{\UpperTr}{\mbox{UpTr}}
\newcommand{\UppTr}{\mbox{UpTr}}
\newcommand{\operation}{ [ D, E, F, \ldots ] \becomes {\rm op}( A, B, C, D, \ldots ) }
\newcommand{\routinename}{ [ D, E, F, \ldots ] \becomes {\rm op}( A, B, C, D, \ldots ) }
\newcommand{\routinecost}{ X }
\newcommand{\precondition}{ Q }
\newcommand{\postcondition}{ R }
\newcommand{\invariant}{ P }
\newcommand{\costinv}{ \  }
\newcommand{\costbefore}{ \  }
\newcommand{\costafter}{ \  }
\newcommand{\updatecost}{ \  }
\newcommand{\guard}{ R }
\newcommand{\partitionings}{
\begin{minipage}{3in}
$ S_I $
\end{minipage}
}
\newcommand{\partitionsizes}{ \hspace{ 3.25in} }

\newcommand{\blocksize}{1}

\newcommand{\repartitionings}{
\begin{minipage}[t]{3in}
\ \\
\ \\
\ \\
\end{minipage}
}

\newcommand{\repartitionsizes}{ \hspace{ 3.25in} }
\newcommand{\moveboundaries}{
\begin{minipage}[t]{3in}
\ \\
\ \\
\ \\
\end{minipage}
}
\newcommand{\beforeupdate}{
$ \QBefore $
}
\newcommand{\afterupdate}{
$ \QAfter $
}
\newcommand{\update}{%
\begin{minipage}[t]{4in}
$ S_U $
\end{minipage}
}

\newcommand{\resetsteps}{
\renewcommand{\blocksize}{1}
\renewcommand{\operation}{ [ D, E, F, \ldots ] \becomes {\rm op}( A, B, C, D, \ldots ) }
\renewcommand{\routinename}{ [ D, E, F, \ldots ] \becomes {\rm op}( A, B, C, D, \ldots ) }
\renewcommand{\routinecost}{ 0 }
\renewcommand{\precondition}{ \PPre }
\renewcommand{\postcondition}{ \PPost }
\renewcommand{\invariant}{ \PInv }
\renewcommand{\costinv}{ \  }
\renewcommand{\guard}{ G }
\renewcommand{\partitionings}{ %
\begin{minipage}[t]{3in}
\ \\
\end{minipage}
}
\renewcommand{\partitionsizes}{ \hspace{ 3.25in} }
\renewcommand{\repartitionings}{%
\begin{minipage}[t]{3in}
\ \\
\end{minipage}
}
\renewcommand{\repartitionsizes}{ \hspace{ 3.25in} }
\renewcommand{\moveboundaries}{%
\begin{minipage}[t]{3in}
\ \\
\end{minipage}
}
\renewcommand{\beforeupdate}{
\QBefore
}
\renewcommand{\afterupdate}{
\QAfter
}
\renewcommand{\update}{
$ S_U $
}
}

\newenvironment{FlaBox}{ 
\footnotesize
\begin{framebox}
}
{
\end{framebox}
}

\newcommand{\WSoperation}{ 
$ \operation $
}

\newcommand{\WSprecondition}{
$ \left\{ \precondition \right\} $
}

\newcommand{\WSpostcondition}{
$ \left\{ \postcondition \right\} $
}

\newcommand{\WSinvariant}{
$ \left\{ \invariant \right\} $
}

\newcommand{\WScostinv}{
$ \Csf = \costinv $ flops
}

\newcommand{\WScostzero}{
$ \Csf = 0 $ flops
}

\newcommand{\WScostbu}{
$ \Csf = \costbefore $ flops
}

\newcommand{\WScostau}{
$ \Csf = \costafter $ flops
}

\newcommand{\WSupdatecost}{
{\bf Cost:} $ \updatecost $ flops
}

\newcommand{\WStopofloop}{
$ \left\{ \left( \invariant \right) \wedge \left( \guard \right) \right\} $
}

\newcommand{\WStopofloopnarrow}{
$ \begin{array}{r}
  \left\{ \left( \invariant \right) \right. \\
  \left. \wedge \left( \guard \right) \right\}
\end{array} $
}

\newcommand{\WSafterloop}{
$ \left\{ \left( \invariant \right) \wedge \neg \left( \guard \right) \right\} $
}

\newcommand{\WSguard}{
$ \guard $
}

\newcommand{\WSpartition}{
\begin{minipage}[t]{3in}
\begin{tabbing}
ind \= ind \= \kill
{\bf Partition} 
\partitionings \+ \\
{\bf where } \hspace*{-2ex} \partitionsizes 
\end{tabbing}
\end{minipage}
}

\newcommand{\WSpartitionNarrow}{
\begin{minipage}[t]{2.0in}
\begin{tabbing}
id \= id \= \kill
{\bf Partition} 
\partitionings \+ \\
{\bf where } 
\begin{minipage}[t]{1.5in}
\partitionsizes 
\end{minipage}
\end{tabbing}
\end{minipage}
}

\newcommand{\WSrepartition}{
\begin{minipage}[t]{3in}
\ifthenelse{ \equal{\blocksize}{1} }{}
{%
\ifthenelse{ \equal{\blocksize}{blank} }{~}
{{\bf Determine block size $ \blocksize $}} \\
}
{\bf Repartition}
\begin{tabbing}
in \= in \= \+ \kill
\repartitionings \+ \\
{\bf where } \hspace*{-2ex} \repartitionsizes 
\end{tabbing}
\end{minipage}
}

\newcommand{\WSrepartitionNarrow}{
\begin{minipage}[t]{2.05in}
\ifthenelse{ \equal{\blocksize}{1} }{}
{%
\ifthenelse{ \equal{\blocksize}{blank} }{~}
{{\bf Determine block size $ \blocksize $}} \\
}
{\bf Repartition}
\begin{tabbing}
i \= i \= \+ \kill
\repartitionings \+ \\
{\bf where }
\begin{minipage}[t]{1.5in}
\repartitionsizes
\end{minipage}
\end{tabbing}
\end{minipage}
}

\newcommand{\WSmoveboundary}{
\begin{minipage}[t]{4in}
{\bf Continue with}
\begin{tabbing}
ind \= \+ \kill
\moveboundaries 
\end{tabbing}
\end{minipage}
}

\newcommand{\WSmoveboundaryNarrow}{
\begin{minipage}[t]{2.05in}
{\bf Continue with}
\begin{tabbing}
i \= \+ \kill
\moveboundaries 
\end{tabbing}
\end{minipage}
}

\newcommand{\WSbeforeupdate}{
$ \left\{ \beforeupdate \right\} $
}

\newcommand{\WSafterupdate}{
$ \left\{ \afterupdate \right\} $
}

\newcommand{\WSupdate}{
\update
}

\newcommand{\worksheet}{
\begin{center}
\begin{tabular}{| c I l |} \hline
\footnotesize {\bf Step} & 
{\bf Annotated Algorithm:} $\operation$ 
\\ \whline 
\rowcolor[gray]{0.93}
1a & 
\WSprecondition \\ \hline
4 & 
\WSpartition \\ \hline
\rowcolor[gray]{0.93}
2 & 
\WSinvariant \\ \hline
3 & 
{\bf while} \WSguard { \bf do} \\ \hline
\rowcolor[gray]{0.93}
2,3 &
\ \hspace{0.15in} \WStopofloop \\ \hline
5a & 
\ \hspace{0.15in} 
\WSrepartition
\\ \hline
\rowcolor[gray]{0.93}
6 &
\ \hspace{0.15in} \WSbeforeupdate \\ \hline
8 & \ \hspace{0.15in} \WSupdate \\ \hline
5b &
\ \hspace{0.15in} 
\WSmoveboundary
\\ \hline
\rowcolor[gray]{0.93}
7 &
\ \hspace{0.15in} \WSafterupdate \\ \hline 
\rowcolor[gray]{0.93}
2 &
\ \hspace{0.15in} \WSinvariant \\ \hline
& {\bf endwhile} \\ \hline
\rowcolor[gray]{0.93}
2,3 &
\WSafterloop \\ \hline
\rowcolor[gray]{0.93}
1b &
\WSpostcondition \\ \hline
\end{tabular}
\end{center}
}

\newcommand{\worksheetnarrow}{
\begin{center}
\begin{tabular}{| c I l |} \hline
\footnotesize {\bf Step} & 
{\bf Annotated Algorithm:} $\routinename$ 
\\ \whline
\rowcolor[gray]{0.93}
1a & 
\WSprecondition \\ \hline
4 & 
\WSpartition \\ \hline
\rowcolor[gray]{0.93}
2 & 
\WSinvariant \\ \hline
3 & 
{\bf while} \WSguard { \bf do} \\ \hline
\rowcolor[gray]{0.93}
2,3 &
\ \hspace{0.15in} \WStopofloopnarrow \\ \hline
5a & 
\ \hspace{0.15in} 
\WSrepartition
\\ \hline
\rowcolor[gray]{0.93}
6 &
\ \hspace{0.15in} \WSbeforeupdate \\ \hline
8 & \ \hspace{0.15in} \WSupdate \\ \hline
5b &
\ \hspace{0.15in} 
\WSmoveboundary
\\ \hline
\rowcolor[gray]{0.93}
7 &
\ \hspace{0.15in} \WSafterupdate \\ \hline 
\rowcolor[gray]{0.93}
2 &
\ \hspace{0.15in} \WSinvariant \\ \hline
& {\bf endwhile} \\ \hline
\rowcolor[gray]{0.93}
2,3 &
\WSafterloop \\ \hline
\rowcolor[gray]{0.93}
1b &
\WSpostcondition \\ \hline
\end{tabular}
\end{center}
}

\newcommand{\worksheetwithcost}{
\begin{center}
\begin{tabular}{| c I l  r | } \hline
\footnotesize {\bf Step} & 
\multicolumn{2}{l|}
{{\bf Cost Analysis:} $\routinename$} 
\\ \whline
\rowcolor[gray]{0.93}
1a & 
% \WSprecondition 
& \WScostzero \\ \hline
4 & \multicolumn{2}{l|}{
\WSpartition} \\ \hline
\rowcolor[gray]{0.93}
2 & 
% \WSinvariant 
& \WScostinv \\ \hline
3 & \multicolumn{2}{l|}{
{\bf while} \WSguard { \bf do} }\\ \hline
\rowcolor[gray]{0.93}
2,3 &
% \ \hspace{0.15in} \WStopofloop 
& \WScostinv \\ \hline
5a & \multicolumn{2}{l|}{
\ \hspace{0.15in} 
\WSrepartition}
\\ \hline
\rowcolor[gray]{0.93}
6 &
% \ \hspace{0.15in} \WSbeforeupdate
& 
\WScostbu
\\ \hline
8 & 
\hspace{0.15in} \WSupdate
&
\WSupdatecost \\ \hline
5b &\multicolumn{2}{l|}{
\ \hspace{0.15in} 
\WSmoveboundary}
\\ \hline
\rowcolor[gray]{0.93}
7 &
% \ \hspace{0.15in} \WSafterupdate 
& 
\WScostau
\\ \hline 
\rowcolor[gray]{0.93}
2 &
% \ \hspace{0.15in} \WSinvariant
&
\WScostinv \\ \hline
& \multicolumn{2}{l|}{{\bf endwhile}} \\ \hline
\rowcolor[gray]{0.93}
2,3 &
% \WSafterloop
&
\WScostinv
\\ \hline
\rowcolor[gray]{0.93}
1b &
% \WSpostcondition
\multicolumn{2}{r|}
{{\bf Total Cost:} $ \routinecost $ flops}
\\ \hline
\end{tabular}
\end{center}
}

\newcommand{\worksheetnosteps}{
\begin{center}
\begin{tabular}{| l |} \hline
Annotated Algorithm: \WSoperation \\ \whline
\rowcolor[gray]{0.93}
\WSprecondition \\ \hline
\WSpartition \\ \hline
\rowcolor[gray]{0.93}
\WSinvariant \\ \hline
{\bf while} \WSguard { \bf do} \\ \hline
\rowcolor[gray]{0.93}
\ \hspace{0.15in} \WStopofloop \\ \hline
\ \hspace{0.15in} 
\WSrepartition
\\ \hline
\rowcolor[gray]{0.93}
\ \hspace{0.15in} \WSbeforeupdate \\ \hline
\hspace{0.15in} \WSupdate \\ \hline
\ \hspace{0.15in} 
\WSmoveboundary
\\ \hline
\rowcolor[gray]{0.93}
\ \hspace{0.15in} \WSafterupdate \\ \hline 
\rowcolor[gray]{0.93}
\ \hspace{0.15in} \WSinvariant \\ \hline
{\bf endwhile} \\ \hline
\rowcolor[gray]{0.93}
\WSafterloop \\ \hline
\rowcolor[gray]{0.93}
\WSpostcondition \\ \hline
\end{tabular}
\end{center}
}

\newcommand{\worksheetupdate}{
\begin{center}
\begin{tabular}{| c I l |} \hline
\footnotesize Step & Annotated Algorithm: \WSoperation \\ \whline
3 & 
{\bf while} \WSguard { \bf do} \\ \hline
2,3 &
\ \hspace{0.15in} \WStopofloop \\ \hline
5a & 
\ \hspace{0.15in} 
\WSrepartition
\\ \hline
6 &
\ \hspace{0.15in} \WSbeforeupdate \\ \hline
8 & \hspace{0.15in} \WSupdate \\ \hline
5b &
\ \hspace{0.15in} 
\WSmoveboundary
\\ \hline
7 &
\ \hspace{0.15in} \WSafterupdate \\ \hline 
2 &
\ \hspace{0.15in} \WSinvariant \\ \hline
& {\bf endwhile} \\ \hline
\end{tabular}
\end{center}
}

\newcommand{\FlaAlgorithm}{
\begin{center}
\begin{tabular}{|l |} \hline
{\bf Algorithm:} $\routinename$ 
\\ \whline
{\WSpartition} \\[0.3in]
{\bf while} \WSguard { \bf do} \\
\ \hspace{0.15in} \WSrepartition \\
{\hspace{0.15in} \FlaStartComputeShort} \\
{\hspace{0.15in} \WSupdate} \\
{\hspace{0.15in} \FlaEndComputeShort} \\
{\ \hspace{0.15in} \WSmoveboundary} \\
{{\bf endwhile}} \\ \hline
\end{tabular}
\end{center}
}


\newcommand{\FlaAlgorithmNarrow}{
\begin{center}
\begin{tabular}{|l |} \hline
{\bf Algorithm:} $\routinename$ 
\\ \whline
{\WSpartitionNarrow} \\
{\bf while} \WSguard { \bf do} \\
\ \hspace{0.0in} \WSrepartitionNarrow \\
{\hspace{0.0in} \FlaStartComputeShorter} \\
{\hspace{0.0in} \WSupdate} \\
{\hspace{0.0in} \FlaEndComputeShorter} \\
{\ \hspace{0.0in} \WSmoveboundaryNarrow} \\
{{\bf endwhile}} \\ \hline
\end{tabular}
\end{center}
}


\newcommand{ \PPre }{ P_{\it pre} }
\newcommand{ \PPost }{ P_{\it post} }
\newcommand{ \PInv }{ P_{\it inv} }
\newcommand{ \PInvone }{ P_{\it inv-1} }
\newcommand{ \PInvtwo }{ P_{\it inv-2} }
\newcommand{ \becomes }{:=}
\newcommand{ \QBefore }{ P_{\it before} }
\newcommand{ \QAfter }{ P_{\it after} }
\newcommand{ \Qbefore }{ P_{\it before} }
\newcommand{ \Qafter }{ P_{\it after} }
\newcommand{ \PCons }{ P_{\it cons} }
\newcommand{ \PCost }{ P_{\it cost} }
\newcommand{ \PStr }{ P_{\it struct} }
\newcommand{ \Csf }{ C_{\it sf} }



\setlength{\parindent}{0in}
\setlength{\parskip}{.4ex}

\renewcommand*\arraystretch{1.2}
\providecommand{\abs}[1]{\left\lvert#1\right\rvert}
\providecommand{\norm}[1]{\lVert#1\rVert}

\def\mbf{\mathbf}
\def\mbb{\mathbb}


%opening
\title{Numerical Analysis: Linear Algebra Final}
\author{Truman Ellis}

\begin{document}

\maketitle

\section*{Problem 1}
\paragraph*{a)} The following results provide the underpinnings of a proof that
the algorithm completes and computes the Cholesky factorization.
\subparagraph*{i} Partition
  $
  A \rightarrow
  \FlaTwoByTwo{A_{TL}}{\star}
              {A_{BL}}{A_{BR}}
  $
and
  $
  L \rightarrow
  \FlaTwoByTwo{L_{TL}}{0}
              {L_{BL}}{L_{BR}}
  $
, where the $\star$ indicates the symmetric part of $A$ that is not stored,
$A_{00}$ and $L_{00}$ are square and $L$ is lower triangular.
\subparagraph*{ii} Assume that $A_{00}$ has a Cholesky factorization
$A_{00}=L_{00}L_{00}^T$.
\subparagraph*{iii} Argue that $l_{10}^T:=a_{10}^TL_{00}^{-T}$ is well-defined.
\subsubsection*{Proof}
With the assumption that $A_{00}$ has a Cholesky factorization, we have $L_00$,
and since $L_{00}$ is the Cholesky factorization of a symmetric positive
definite matrix, it is invertible.
Multiplying both sides by $L_{00}^{-T}$ on the right, we can rewite this as
\[
l_{10}^TL_{00}^T=a_{10}^T
\]
or
\[
L_{00}l_{10}=a_{10}\,.
\]
Since $L_{00}$ is lower triangular, we can trivially solve this for $l_{10}$ at
each iteration. Therefore this operation is well-defined.
\subparagraph*{iv} Prove that $\alpha_{11}-l{10}^Tl_{10}$ is positive.
\subsubsection*{Proof}
First let's expand our expression: $l_{10}^T=a_{10}^TL_{00}^{-T}$. So
\[
\alpha_{11}-l_{10}^Tl_{10}=\alpha_{11}-a_{10}^TL_{00}^{-T}L_{00}^{-1}a_{10}
=\alpha_{11}-a_{10}^TA_{00}^{-1}a_{10}\,.
\]

To the contrary, assume $\alpha_{11}-a_{10}^TA_{00}^{-1}a_{10} < 0$, then there
exists a positive constant $C$ such that,
$\alpha_{11}=a_{10}^TA_{00}^{-1}a_{10}-C$. Then
we can substitute this equality into $A$ and compute $x^TAx$ with
\[
x=\FlaThreeByOneB{x_0}
                 {\chi_1}
                 {0}\,.
\]
Then we get
\begin{align*}
x^TAx&=\FlaOneByThreeR{x_0^T}
                 {\chi_1}
                 {0}
\FlaThreeByThreeBR{A_{00}}{a_{01}}{A_{02}}
                    {a_{10}^T}{a_{10}^TA_{00}^{-1}a_{10}-C}{a_{12}^T}
                    {A_{20}}{a_{21}}{A_{22}}
\FlaThreeByOneB{x_0}
                 {\chi_1}
                 {0}\\
&=x_{0}^TA_{00}x_0+\chi_1a_{10}^Tx_0+x_0^Ta_{10}\chi_1
+\chi_1(a_{10}^TA_{00}^{-1}a_{10}-C)\chi_1\\
&=x_{0}^TA_{00}x_0+\chi_1a_{10}^T(x_0+A_{00}^{-1}a_{10}\chi_1)
+x_0^Ta_{10}\chi_1-\chi_1C\chi_1\,.
\end{align*}
Now let $x_0=-A_{00}^{-1}a_{10}\chi_1$, then
\begin{align*}
x^TAx&=\cancel{\chi_1^2a_{10}^TA_{00}^{-T}a_{10}}
-\cancel{\chi_1^2a_{10}^TA_{00}^{-T}a_{10}}-C\chi_1^2\\
&=-C\chi_1^2\,.
\end{align*}
Since $C>0$ by definition, the result is negative for $\chi_1\neq1$. This is a
contradiction since $A$ is symmetric positive definite, thus we must have
$\alpha_11-l_{10}^Tl_{10}$ positive.

\paragraph*{b)}
Use the above insights to provide and inductive proof of the Cholesky
factorization theorem.
\subsubsection*{Proof}
Proof by induction.
\paragraph*{Base case:} $n=1$. Clearly the result is true for a $1\times1$
matrix $A=\alpha_{11}$: Since $A$ is symmetric positive definite, $\alpha_{11}$
is real and positive and the Cholesky factorization is simply given by
$\lambda_{11}=\sqrt{\alpha_{11}}$, and is unique if we insist that
$\lambda_{11}$ be positive.

\paragraph*{Inductive step:} Assume the result is true for symmetric positive
definite matrix $A\in\mathbb{C}^{n\times n}$. We need to show that it holds for
$A\in\mathbb{C}^{(n+1)\times(n+1)}$. Let $A\in\mathbb{C}^{(n+1)\times(n+1)}$ be
symmetric positive definite. Partition $A$ and $L$,
\[
  \FlaTwoByTwo{A_{TL}}{\star}
              {A_{BL}}{A_{BR}}
  \rightarrow
  \FlaThreeByThreeBR{A_{00}}{\star}{\star}
                    {a_{10}^T}{\alpha_{11}}{\star}
                    {A_{20}}{a_{21}}{A_{22}}
\]
\[
  \FlaTwoByTwo{L_{TL}}{0}
              {L_{BL}}{L_{BR}}
  \rightarrow
  \FlaThreeByThreeBR{L_{00}}{0}{0}
                    {l_{10}^T}{\lambda_{11}}{0}
                    {L_{20}}{l_{21}}{L_{22}}\,.
\]
Let $l_{10}^T=a_{10}^TL_{00}^{-T}$, which we just showed is well-defined and
$A_{00}=L_{00}L_{00}^T$ by the inductive hypothesis. Then let
$\lambda_{11}=\sqrt{\alpha_{11}-l_{10}^Tl_{10}}$, where
$\alpha_{11}-l_{10}^Tl_{10}$ is positive. Finally,
$L_{22}=\textsc{Chol}(A_{22})$, whic exists by the inductive hypothesis. Then
$L$ is the desired Cholesky factor of $A$.

By the principle of mathematical induction, the theorem holds.

\section*{Problem 2}
Given a matrix $A\in\mbb{R}^{m\times n}$. A \emph{communication avoiding}
algorithm for the QR factorization follows:
\begin{itemize}
\item View matrix A as two submatrices $B\,,C\in\mbb{R}^{\frac{m}{2}\times n}$
so that $\displaystyle A=\FlaTwoByOneDoubleLine{B}{C}$.
\item Compute the QR factorizations of $B$ and $C$: $B=Q_BR_B$ and $C=Q_CR_C$.
\item Compute the QR factorization of
$\FlaTwoByOneDoubleLine{R_B}{R_C}=Q_{BC}R_{BC}$.
\end{itemize}
\paragraph*{a)} Prove that, under mild conditions, the $R_{BC}$ that results
from the above procedure is exactly the $R$ that would come out of a QR
factorization of $A$. What are those mild conditions?

\subsubsection*{Proof}
Expanding according to the process described above,
\begin{align*}
A&=\FlaTwoByOneDoubleLine{B}{C}=\FlaTwoByOneDoubleLine{Q_BR_B}{Q_CR_C}\\
&=\FlaTwoByTwoDoubleLine{Q_B}{0}{0}{Q_C}\FlaTwoByOneDoubleLine{R_B}{R_C}\\
&=\FlaTwoByTwoDoubleLine{Q_B}{0}{0}{Q_C}Q_{BC}R_{BC}\,,
\end{align*}
where $Q_B$ and $Q_C$ are $\frac{m}{2}\times n$, $R_B$, $R_C$, and $R_{BC}$ are
$n\times n$, and $Q_{BC}$ is $2n\times n$.

The matrix
$
\FlaTwoByTwoDoubleLine{Q_B}{0}{0}{Q_C}
$
is unitary, since $Q_B$ and $Q_C$ are unitary and
\begin{align*}
&\FlaTwoByTwoDoubleLine{Q_B}{0}{0}{Q_C}^H\FlaTwoByTwoDoubleLine{Q_B}{0}{0}{Q_C}
=\FlaTwoByTwoDoubleLine{Q_B^HQ_B}{0}{0}{Q_C^HQ_C}\\
&=I=\FlaTwoByTwoDoubleLine{Q_BQ_B^H}{0}{0}{Q_CQ_C^H}
=\FlaTwoByTwoDoubleLine{Q_B}{0}{0}{Q_C}
\FlaTwoByTwoDoubleLine{Q_B}{0}{0}{Q_C}^H\,.
\end{align*}
Since a unitary matrix represents a length-preserving transformation, the
product of two unitary matrices is also unitary, and the final $R_{BC}$
preserves the length information of the original matrix. Provided $A$ has
linearly independent columns, it has a unique QR factorization and
\[
Q=\FlaTwoByTwoDoubleLine{Q_B}{0}{0}{Q_C}Q_{BC}
\]
and
\[
R=R_{BC}\,,
\]
where $A=QR$.

\paragraph*{b)} Propose an algorithm for computing the QR factorization of $
\FlaTwoByOneDoubleLine{R_B}{R_C}$ via Householder transformations that takes
advantage of the fact that $R_B$ and $R_C$ are upper triangular.


\resetsteps      % Reset all the commands to create a blank worksheet  

% Define the operation to be computed

\renewcommand{\routinename}{ \left[ R \right] := \mbox{QR}( R, U ) }

% Step 3: Loop-guard 

\renewcommand{\guard}{
  m( R_{TL} ) < m( R )
}

% Step 4: Define Initialize 

\renewcommand{\partitionings}{
  $
  \FlaTwoByOneDoubleLine{R}{U} \rightarrow
  \FlaTwoByOneDoubleLine{
  \FlaTwoByTwo{R_{TL}}{R_{TR}}
              {0}{R_{BR}}}{
  \FlaTwoByTwo{U_{TL}}{U_{TR}}
              {0}{U_{BR}}}
  $
}

\renewcommand{\partitionsizes}{
$ R_{TL} $ and $ U_{TL} $ are $ 0 \times 0 $,
}

% Step 5a: Repartition the operands 

\renewcommand{\repartitionings}{
$
  \FlaTwoByOneDoubleLine{
  \FlaTwoByTwo{R_{TL}}{R_{TR}}
              {0}{R_{BR}}}{
  \FlaTwoByTwo{U_{TL}}{U_{TR}}
              {0}{U_{BR}}}
  \rightarrow
  \FlaTwoByOneDoubleLine{
  \FlaThreeByThreeBR{R_{00}}{r_{01}}{R_{02}}
                    {0}{\rho_{11}}{r_{12}^T}
                    {0}{0}{R_{22}}}{
  \FlaThreeByThreeBR{U_{00}}{u_{01}}{U_{02}}
                    {0}{\upsilon_{11}}{u_{12}^T}
                    {0}{0}{U_{22}}}
$
}

\renewcommand{\repartitionsizes}{
  $ \rho_{11} $ and $ \upsilon_{11} $ are $ 1 \times 1 $}

% Step 5b: Move the double lines 

\renewcommand{\moveboundaries}{
$
  \FlaTwoByOneDoubleLine{
  \FlaTwoByTwo{R_{TL}}{R_{TR}}
              {0}{R_{BR}}}{
  \FlaTwoByTwo{U_{TL}}{U_{TR}}
              {0}{U_{BR}}}
  \leftarrow
  \FlaTwoByOneDoubleLine{
  \FlaThreeByThreeTL{R_{00}}{r_{01}}{R_{02}}
                    {0}{\rho_{11}}{r_{12}^T}
                    {0}{0}{R_{22}}}{
  \FlaThreeByThreeTL{U_{00}}{u_{01}}{U_{02}}
                    {0}{\upsilon_{11}}{u_{12}^T}
                    {0}{0}{U_{22}}}
$
}

% Step 8: Insert the updates required to change the 
%         state from that given in Step 6 to that given in Step 7
% Note: The below needs editing!!!

\renewcommand{\update}{
$
  \begin{array}{l}
    \mbox{$\left[
    \FlaTwoByOneSingleLine{\rho_{11}}{x_{21}},\tau_1\right]
    :=\textsc{Housev}\FlaTwoByOneDoubleLine{\rho_{11}}{
    \FlaTwoByOneNoPar{u_{01}}{\upsilon_{11}}}$} \\
    \mbox{Update $\displaystyle
    \FlaTwoByOneDoubleLine{r_{12}^T}{
    \FlaTwoByOneNoPar{U_{02}}{u_{12}^T}}
    :=\left(I-\frac{1}{\tau}\FlaTwoByOneSingleLine{1}{x_{21}}
    \FlaOneByTwoSingleLine{1}{x_{21}}\right)
    \FlaTwoByOneDoubleLine{r_{12}^T}{
    \FlaTwoByOneNoPar{U_{02}}{u_{12}^T}}
    $} \\
    \mbox{via the steps} \\
    \mbox{$w_{12}^T:=\left(r_{12}^T+x_{21}^H
    \FlaTwoByOne{U_{02}}{u_{12}^T}\right)/\tau_1$} \\
%     \mbox{$v_{12}^T:=(u_{12}^T+x_{21}^HU_{22})/\tau_1$} \\
    \mbox{$\FlaTwoByOneDoubleLine{r_{12}^T}{
    \FlaTwoByOneNoPar{U_{02}}{u_{12}^T}}
    :=\FlaTwoByOneDoubleLine{r_{12}^T-w_{12}^T}{
    \FlaTwoByOne{U_{02}}{u_{12}^T}-x_{21}w_{12}^T}$} \\
%     \mbox{$\FlaTwoByOneSingleLine{u_{12}^T}{U_{22}}
%     :=\FlaTwoByOneSingleLine{u_{12}^T-v_{12}^T}{U_{22}-x_{21}v_{12}^T}$} \\
  \end{array}
$
}



  % Load commands from file algorithm.tex
\begin{figure}[tbp]
\begin{center}     % center the algorithm on the page
\FlaAlgorithm      % this command typesets the algorithm
\end{center}
\caption{Algorithm for computing
$\left[ R \right] := \mbox{QR}( R, U )$.}
\label{fig:label}
\end{figure}

\paragraph*{c)} Compute the approximate operation count of your algorithm and
compare it to the operation count had you used a standard Householder
transformation based algorithm on $\FlaTwoByOneDoubleLine{R_B}{R_C}$ without
taking any advantage of the special structure of $R_B$ and $R_C$.

\subsubsection*{Solution}
The majority of the computation goes into
\[
w_{12}^T:=\left(r_{12}^T+x_{21}^H\FlaTwoByOne{U_{02}}{u_{12}^T}\right)/\tau_1
\]
and
\[
\FlaTwoByOne{U_{02}}{u_{12}^T}-x_{21}w_{12}^T\,.
\]
During the $k$th iteration this means a matrix-vector multiplication
$x_{21}^H\FlaTwoByOne{U_{02}}{u_{12}^T}$ and a rank-1 update with matrix
$\FlaTwoByOne{U_{02}}{u_{12}^T}$ which is of size $(k+1)\times(n-k-1)$, for a
cost of $4(k+1)(n-k)$ flops. Thus the total cost is approximately
\begin{align*}
\sum_{k=0}^{n-1}4(k+1)(n-k-1)&\approx4\int_0^n(k+1)(n-k-1)dk\\
&=\frac{2}{3}n^3-4n\approx\frac{2}{3}n^3\,.
\end{align*}
The standard Householder QR algorithm would have cost
\[
2(2n)n^2-\frac{2}{3}n^3=\frac{10}{3}n^3\,,
\]
which is five times slower.

\section*{Problem 3}
\paragraph*{a)} Prove that $\hat V^{(k)}$ and $\hat A^{(k)}$ from the subspace
iteration are the same as $V^{(k)}$ and $A^{(k)}$ from the QR algorithm.

\subsubsection*{Proof}
For reference, let us reprint the two algorithms side by side.
\begin{figure}[ht]
\begin{minipage}[t]{0.5\linewidth}

\underline{Subspace iteration}
\begin{algorithmic}
\STATE $\hat A^{(0)}:=A$
\STATE $\hat V^{(0)}:=I$
\FOR{$k=0$ to convergence}
\STATE $A\hat V^{(k)}\rightarrow\hat V^{(k+1)}\hat R^{(k+1)}$ (QR factorization)
\STATE $\hat A^{(k+1)}:=\hat V^{(k+1)T}A\hat V^{(k+1)}$
\ENDFOR
\end{algorithmic}

\end{minipage}
\begin{minipage}[t]{0.5\linewidth}

\underline{QR algorithm}
\begin{algorithmic}
\STATE $A^{(0)}:=A$
\STATE $V^{(0)}:=I$
\FOR{$k=0$ to convergence}
\STATE $A^{(k)}\rightarrow Q^{(k+1)}R^{(k+1)}$ (QR factorization)
\STATE $A^{(k+1)}:=R^{(k+1)}Q^{(k+1)}$
\STATE $V^{(k+1)}:=V^{(k)}Q^{(k+1)}$
\ENDFOR
\end{algorithmic}

\end{minipage}
\caption{Basic subspace iteration and basic QR algorithm.}
\end{figure}

\paragraph*{Base case:} $k=0$. Both algorithms start identically with $\hat
A^{(0)}=A^{(0)}=A$ and $\hat V^{(0)}=V^{(0)}=I$, so both algorithms enter the
for loop identically. Our first computation in the subspace iteration is to
compute
\[
[\hat V^{(1)},\hat R^{(1)}]=\mathrm{QR}(A\hat V^{(0)})=\mathrm{QR}(A)\,,
\]
since $\hat V^{(0)}=I$. The first computation of the QR algorithm is to obtain
\[
[\hat Q^{(1)},\hat R^{(1)}]=\mathrm{QR}(A^{(0)})=\mathrm{QR}(A)\,,
\]
since $A^{(0)}=A$. So at this point we have $\hat V^{(1)}=Q^{(1)}$ and $\hat
R^{(1)}=R^{(1)}$. In the next step of the subspace iteration we compute
\begin{align*}
\hat A^{(1)}&:=\hat V^{(1)T}A\hat V^{(1)}
=\cancel{\hat V^{(1)T}\hat V^{(1)}}\hat R^{(1)}\hat V^{(1)}
=\hat R^{(1)}\hat V^{(1)}\\
&=R^{(1)}Q^{(1)}=A^{(1)}\,,
\end{align*}
since we just factored $A=\hat V^{(1)}\hat R^{(1)}$, $\hat V^{(1)}=Q^{(1)}$,
and $\hat R^{(1)}=R^{(1)}$, which is exactly the second step of the QR
algorithm. Finally, the QR algorithm computes
\[
V^{(1)}:=V^{(0)}Q^{(1)}=Q^{(1)}\,,
\]
since $V^{(0)}=I$, which is exactly the result we got from the subspace
iteration. So after one iteration, we get the desired result: $\hat
V^{(1)}=V^{(1)}$ and $\hat A^{(1)}=A^{(1)}$.

\paragraph*{Inductive step:} Assume that $\hat
V^{(k)}=V^{(k)}$ and $\hat A^{(k)}=A^{(k)}$ holds for some $k$, we need to show
that it holds for $k+1$. From the inductive hypothesis, we can assume the
following at step $k+1$:
\begin{enumerate}
\item $AV^{(k-1)}=V^{(k)}R^{(k)}$
\item $A^{(k)}=V^{(k)T}AV^{(k)}$
\item $A^{(k-1)}=Q^{(k)}R^{(k)}$
\item $A^{(k)}=R^{(k)}Q^{(k)}$
\item $V^{(k)}=V^{(k-1)}Q^{(k)}=Q^{(1)}Q^{(2)}\cdots Q^{(k)}$
\end{enumerate}
From these assumptions, we can derive some helpful identities. Combining the
second and last properties yields,
\begin{equation}\label{eq:base1}
A^{(k)}=Q^{(k)T}\cdots Q^{(1)T}AQ^{(1)}\cdots Q^{(k)}\,.
\end{equation}
Moving on to our $k+1$ step, anywhere we see a $(k)$ term, we can apply our
assumptions, thus step 1 of the QR algorithm becomes (via Eq. \ref{eq:base1})
\begin{equation}
Q^{(k)T}\cdots Q^{(1)T}AQ^{(1)}\cdots Q^{(k)}=Q^{(k+1)}R^{(k+1)}\,.
\end{equation}
From assumption 5, step 1 of the subspace iteration becomes
\begin{equation}\label{eq:sub1}
AQ^{(1)}\cdots Q^{(k)}=\hat V^{(k+1)}\hat R^{(k+1)}\,.
\end{equation}
There is a strong similarity in structure between these two equations, and in
fact, if we multiply Eq. \ref{eq:sub1} on the left by $V^{(k)T}$, we get
\begin{equation}
Q^{(k)T}\cdots Q^{(1)T}AQ^{(1)}\cdots Q^{(k)}=A^{(k)}
=\hat V^{(k+1)}\hat R^{(k+1)}\,,
\end{equation}
which is just the QR factorization of $A^{(k)}$. Assuming $A^{(k)}$ has
linearly independent columns, the QR factorization is unique and we must have
\[
Q^{(k)T}\cdots Q^{(1)T}\hat V^{(k+1)}\hat R^{(k+1)}=Q^{(k+1)}R^{(k+1)}\,.
\]
Multiplying both sides on the left by $V^{(k)}$,
\[
\hat V^{(k+1)}\hat R^{(k+1)}=Q^{(1)}\cdots Q^{(k+1)}R^{(k+1)}
=V^{(k+1)}R^{(k+1)}\,.
\]
Since the QR factorization is unique, we must have
\[
\hat V^{(k+1)}=V^{(k+1)}\,,
\]
and
\[
\hat R^{(k+1)}=R^{(k+1)}\,.
\]
Finally, we need to prove that $\hat A^{(k+1)}=A^{(k+1)}$. Recall that
$A^{(k)}=Q^{(k+1)}R^{(k+1)}$. Looking at step 2 of the subspace iteration,
\begin{align*}
\hat A^{(k+1)}&=Q^{(k+1)T}\cdots Q^{(1)T}AQ^{(1)}\cdots Q^{(k+1)}\\
&=Q^{(k+1)T}\cdots \cancel{Q^{(1)T}Q^{(1)}}\underbrace{R^{(1)}Q^{(1)}}_
{A^{(1)}=Q^{(2)}R^{(2)}}
\cdots
Q^{(k+1)}\\
&=Q^{(k+1)T}\cdots \cancel{Q^{(2)T}Q^{(2)}}\underbrace{R^{(2)}Q^{(2)}}_
{A^{(2)}=Q^{(3)}R^{(3)}}
\cdots
Q^{(k+1)}\\
&\vdots \\
&=\cancel{Q^{(k+1)T}Q^{(k+1)}}R^{(k+1)}Q^{(k+1)}\\
&=R^{(k+1)}Q^{(k+1)}=A^{(k+1)}\,.
\end{align*}
So by the principle of mathematical induction, $\hat R^{(k)}=R^{(k)}$,
$\hat V^{(k)}=V^{(k)}$, and $\hat A^{(k)}=A^{(k)}$ for all $k$.

\paragraph*{Extending to shifted QR algorithm} We would follow a very similar
process to extend the results to a shifted QR algorithm, printed below for
reference.

\begin{figure}[ht]
\begin{minipage}[t]{0.5\linewidth}

\underline{Subspace iteration}
\begin{algorithmic}
\STATE $\hat A^{(0)}:=A$
\STATE $\hat V^{(0)}:=I$
\FOR{$k=0$ to convergence}
\STATE $\hat\mu_k:=v_{n-1}^{(k)T}A_{n-1}^{(k)}$
\STATE $(A-\hat\mu_kI)\hat V^{(k)}\rightarrow\hat V^{(k+1)}\hat R^{(k+1)}$ (QR
factorization)
\STATE $\hat A^{(k+1)}:=\hat V^{(k+1)T}A\hat V^{(k+1)}$
\ENDFOR
\end{algorithmic}

\end{minipage}
\begin{minipage}[t]{0.5\linewidth}

\underline{QR algorithm}
\begin{algorithmic}
\STATE $A^{(0)}:=A$
\STATE $V^{(0)}:=I$
\FOR{$k=0$ to convergence}
\STATE $\mu_k:=\alpha_{n-1,n-1}^{(k)}$
\STATE $A^{(k)}-\mu_kI\rightarrow Q^{(k+1)}R^{(k+1)}$ (QR factorization)
\STATE $A^{(k+1)}:=R^{(k+1)}Q^{(k+1)}+\mu_kI$
\STATE $V^{(k+1)}:=V^{(k)}Q^{(k+1)}$
\ENDFOR
\end{algorithmic}

\end{minipage}
\caption{Basic shifted subspace iteration and basic shifted QR algorithm.}
\end{figure}

The first step would be to follow the algorithms through for $k=0$ which would
show that
\[
\hat\mu_0=\mu_0=\alpha_{n-1,n-1}^{(0)}\,,
\]
\[
\hat V^{(1)}=V^{(1)}=Q^{(1)}\,,
\]
and
\[
\hat A^{(1)}=A^{(1)}=R^{(1)}Q^{(1)}+\mu_0I\,.
\]
Then we would need to propose the inductive hypothesis and compose a list of
assumptions as before:
\begin{enumerate}
\item $\mu_{k-1}=v_{n-1}^{(k-1)T}A_{n-1}^{(k-1)}$
\item $(A-\mu_{k-1}I)V^{(k-1)}=V^{(k)}R^{(k)}$
\item $A^{(k)}=V^{(k)T}AV^{(k)}$
\item $\mu_{k-1}=\alpha_{n-1,n-1}^{(k-1)}$
\item $A^{(k-1)}-\mu_{k-1}I=Q^{(k)}R^{(k)}$
\item $A^{(k)}=R^{(k)}Q^{(k)}+\mu_{k-1}I$
\item $V^{(k)}=V^{(k-1)}Q^{(k)}=Q^{(1)}Q^{(2)}\cdots Q^{(k)}$
\end{enumerate}

Again, we could combine terms to get
\begin{equation}
A^{(k)}=Q^{(k)T}\cdots Q^{(1)T}AQ^{(1)}\cdots Q^{(k)}\,.
\end{equation}
We could substitute this into step 2 of the shifted QR algorithm and compare the
results with step 2 of the shifted subspace iteration. We could multiply step
two of the subspace iteration on the left by $V^{(k)T}$ and use the uniqueness
of QR factorization to argue that
\[
\hat V^{(k+1)}=V^{(k+1)}\,,
\]
and
\[
\hat R^{(k+1)}=R^{(k+1)}\,.
\]
This would also allow us to see that $\hat \mu_k=\mu_k=\alpha_{n-1,n-1}^{(k)}$.
Finally, we could collapse the third step of the shifted subspace iteration to
the third step of the shifted QR algorithm, proving that
\[
\hat A^{(k+1)}=A^{(k+1)}\,.
\]
Thus, the theorem would hold by the principle of mathematical
induction, and we will have proved our point.

\section{Problem 4}
With $\lambda_{11}=1$, the main source of error in our calculations is from
\begin{equation}\label{eq:errorsource}
\chi_1:=\beta_1-l_{10}^Tx_0\,.
\end{equation}
According to Theorem 4.1, we wish to push all of the error into the $L$ matrix
for easy comparison between results. Theorem 3.12 however, does not enumerate
exactly the correct result to allow us to do this. Equation
\ref{eq:errorsource} fits the pattern of ``$\mu:=\alpha-(x^Ty)$.'' R1-B puts
would make it possible to put the error into both $L$ and the $b$ vector,
while R1-F and R2-F could put the error into either $x$ or $b$ depending on how
they were considered. The remaining results fit the pattern
``$\nu:=(\alpha-(x^Ty))/\lambda$.''  Of course, in our case, $\lambda=1$, so we
don't need that final divide. Omitting the division operation, we can modify
R4-B to fit our purposes, in this case we get
\[
\check\nu=\frac{\alpha-(x+\delta x)^Ty}{\lambda+\delta\lambda}\text{, where }
\abs{\delta\lambda}\leq\gamma_1\abs{\lambda}
\text{ and }\abs{\delta x}\leq\gamma_n\abs{x}.
\]
Note, the division operation is not committed, it is only shown here to show
the error accumulation in the ``ones'' on the diagonal in lambda. The other
alternative was to throw all of the error on $x$, the part of $L$ below the
diagonal, but this method makes for a more direct comparison of results.

With this new error pattern established, we can modify our results to compute a
new error bound. Clearly the first change we can make is in Fig. 4.1 where we
can replace $\chi_1:=(\beta_1-l_{10}^Tx_0)\lambda_{11}$ with simply
$\chi_1:=(\beta_1-l_{10}^Tx_0)$. This correction should be made anywhere this
result is computed. In step 1 of Section 4.2, our backward error analysis
becomes
\[
(L+\Delta L)\check x=b\text{ with }\abs{\Delta L}\leq\gamma_{n-1}\abs{L}\,,
\]
where the $\max(\gamma_2,\gamma_{n-1})$ term was replaced with simply
$\gamma_{n-1}$. Also, the error postcondition becomes
\[
{(L+\Delta L)\check x=b\wedge\abs{\Delta L}\leq\gamma_{n-1}\abs{L}\wedge m(x)=n}
\,.
\]
Similarly, the error invariant becomes,
\[
{(L_{TL}+\Delta L_{TL})\check x=b_T\wedge\abs{\Delta
L_{TL}}\leq\gamma_{n-1}\abs{L_{TL}}\wedge m(x_T)=k}\,.
\]
The bottom half of Equation (4.2) becomes
\[
\abs{\FlaTwoByTwoSingleLine{\Delta L_{00}}{0}{\delta
l_{10}^T}{\delta\lambda_{11}}}\leq\gamma_{k}
\abs{\FlaTwoByTwoSingleLine{L_{00}}{0}{l_{10}^T}{1}}\,.
\]
Now, our error variables must satisfy
\[
\begin{array}{ll}
b_0=(L_{00}+\Delta L_{00})\check x_0
& \wedge\abs{\Delta L_{00}}\leq\gamma_k\abs{L_{00}} \\
\beta_1=(l_{10}^T+\delta l_{10}^T)\check x_0+(1+\delta\lambda_{11})\check\chi_1
& \wedge\abs{(\delta l_{10}^T|\delta\lambda_{11})}\leq\gamma_k
\abs{(l_{10}^T|1)} \\
\end{array}
\]

Looking now at Figure 4.2, we need to make the following changes. The error
postcondition in line 1b should be replaced with the one described above. The
error invariants in 2a, 2b, 2c, and 2d should be replaced with the one described
above. Lines 5a, 6, 7, and 8 should be modified to reflect the fact that
$\lambda_{11}=1$. In lines 6, 7, and 8 replace $\max(\gamma_2,\gamma_{k-1})$
with $\gamma_{k-1}$ and $\max(\gamma_2,\gamma_{k})$ with $\gamma_{k}$.
\end{document}