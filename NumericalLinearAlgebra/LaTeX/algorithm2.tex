
\resetsteps      % Reset all the commands to create a blank worksheet  

% Define the operation to be computed

\renewcommand{\routinename}{ \left[ y \right] := \mbox{\sc SLAP_Gemv\_unb}(
\alpha, A, x, y ) }

% Step 3: Loop-guard 

\renewcommand{\guard}{
  n( A_L ) < n( A )
}

% Step 4: Define Initialize 

\renewcommand{\partitionings}{
  $
  A \rightarrow
  \FlaOneByTwo{A_L}{A_R}
  $
,
  $
  x \rightarrow
  \FlaTwoByOne{x_{T}}
              {x_{B}}
  $
}

\renewcommand{\partitionsizes}{
$ A_L $ has $ 0 $ columns,
$ x_{T} $ has $ 0 $ rows
}

% Step 5a: Repartition the operands 

\renewcommand{\repartitionings}{
$
  \FlaOneByTwo{A_L}{A_R}
\rightarrow  \FlaOneByThreeR{A_0}{a_1}{A_2}
$
,
$
  \FlaTwoByOne{ x_T }
              { x_B }
\rightarrow
  \FlaThreeByOneB{x_0}
                 {\chi_1}
                 {x_2}
$
}

\renewcommand{\repartitionsizes}{
$ a_1 $ has $1$ column,
$ \chi_1 $ has $ 1 $ row}

% Step 5b: Move the double lines 

\renewcommand{\moveboundaries}{
$
  \FlaOneByTwo{A_L}{A_R}
\leftarrow  \FlaOneByThreeL{A_0}{a_1}{A_2}
$,
$
  \FlaTwoByOne{ x_T }
              { x_B }
\leftarrow
  \FlaThreeByOneT{x_0}
                 {\chi_1}
                 {x_2}
$
}

% Step 8: Insert the updates required to change the
%         state from that given in Step 6 to that given in Step 7
% Note: The below needs editing!!!

\renewcommand{\update}{
$
  \begin{array}{l}
    \mbox{update line 1} \\
    \mbox{\ \ \ \ :} \\
    \mbox{update line n} \\
  \end{array}
$
}
